\chapter{Framework} \label{chap:framework}

    Building upon the ideas presented in Chapter \ref{chap:design}, we outline how an prototype Python framework was implemented. In this chapter, we demonstrate the construction process, focusing on a problem-agnostic approach. To maintain generality, some classes have been mocked, allowing us to emphasize the framework's core design principles rather than problem-specific details.
    The implementation of the framework is available on GitHub\footnote{\url{https://github.com/RafaelGranza/Matching-Optimization-Core}} \cite{myframework}.

    \section{Representing Relations}
        % To represent the relations, we provide the class         \textit{GroupingRule}, which stores each classes that can be part of a Grouping, the specifications, and also represents the geometry of the Grouping, i.e. how many relations each object can be part of, can ahve and so on. 
        To represent the rules of a grouping, as defined in Section \ref{sec:relation_rules}, we provide the class \textit{GroupRule} (see Script \ref{script:groupingrule}).
        The \textit{GroupRule} provides methods to define the classes that can be part of a \textit{Group}, the specifications, and also represents the rules of a specific \textit{Group}, as defined in Script \ref{script:groupingrule}.
        %
        In Section \ref{sec:examples}, we will provide examples of how to create these rules using the \textit{GroupRule} class.

        \begin{lstlisting}[language=Python, caption={\textit{GroupRule} class that defines valid \textit{Groups} and its specifications. Such as min/max cardinality of each relation, group validation functions, statistic functions, and object function.}, label={script:groupingrule}]
class GroupRule:
    # Initializes the grouping rule object.
    def __init__(self):

    # Sets the objective function (string or callable).
    def set_objective_function(self, function)

    # Defines min/max cardinality for a class in the grouping.
    def set_cardinality(self, cls, min_count, max_count)

    # Adds a validator function for group validation.
    def add_validator(self, validator_fn)

    # Adds a statistic function for the grouping.
    def add_statistic(self, statistic)

    # Validates group, raises exception if invalid.
    def validate_or_raise(self, group)

    # Validates group(s), returns True if valid.
    def validate(self, groups)

    # Enables or disables stable matching.
    def set_stable_match(self, stable_match)
\end{lstlisting}

    %One can see some examples of how to create a \textit{GroupingRule} at Section \ref{sec:examples}.
    The \textit{Group} class serves as a flexible container for objects, as demonstrated in Script \ref{script:grouping}. It represents the elements (instances) that are grouped together, i.e., members of a group. Importantly, these objects can be instances of any class, and they are organized by their class type.

    \begin{lstlisting}[language=Python, caption={\textit{Group} Class. It represents the elements (instances) that are grouped together.}, label={script:grouping}]
class Group:
    """
    A container of members, which can be any type of object.
    Members are organized by their class type.
    """

    # Initializes the Group object with an empty members dictionary.
    def __init__(self):

    # Adds a member to the group, allowing multiple instances.
    def add_member(self, *instances):

    # Removes a member from the group, allowing multiple instances.
    def remove_member(self, *instances):

    # Retrieves a dictionary of members divided by class type, optionally filtered by only one class type.
    def get_members(self, cls=None):

    # Retrieves members of the group as a list, optionally filtered by class type.
    def get_members_as_list(self, cls=None):

    # Returns a string representation of the group, showing all members and their types.
    def __repr__(self):
    \end{lstlisting}

    \section{Solvers}
        As described in Section \ref{sec:solvers}, the framework is designed to accommodate various solvers, each tailored to specific grouping problems. These solvers are implemented as classes that inherit from the abstract class \textit{Solver} (see Script \ref{script:solver}).
        Different \textit{Solvers} work with very different problems, and each problem may be solved by more than one available solver.
        Therefore, we provide an assigner to select the suitable solver for each problem.
        %and for the purpose of this work, they are enough to demonstrate the efficiency of the design.

        \subsection{Assigning Solvers}
            The Script \ref{script:assinger} implements a simple solver assignment strategy equivalent to the flow-chart of Figure \ref{fig:flow_chart}.
            Returns the first solver capable of handling the problem from a priority list.
            The high-priority algorithms are the deterministic ones.
            %The assigner gives the responsability to the solvers. 
            %They ask whether each solver can solve and specific problem as shown in Script \ref{script:assinger}.
            
    \begin{lstlisting}[language=Python, caption={Assigner Class.}, label={script:assinger}]
class Assigner:

    solvers: List[Solver] = [
        HungarianAlgorithm,
        BinSearchHungarianAlgorithm,
        StableMarriage,
        GeneticAlgorithm, # Metaheuristic solver
        # Add more solvers here as needed (e.g., SimulatedAnnealing)
    ]

    def choose_solver(self, group_rule: GroupRule) -> Solver:
        """
        Choose the best solver based on the group members and their cardinality.
        """
        for solver in self.solvers:
            if solver.can_solve(group_rule):
                return solver
        raise ValueError("No suitable solver found for the given group.")

    def add_solver(self, solver: Type[Solver]):
        """
        Add a new solver to the list of available solvers.
        """
        self.solvers.append(solver)
    
    def remove_solver(self, solver: Type[Solver]):
        """
        Remove a solver from the list of available solvers.
        """
        self.solvers.remove(solver)
\end{lstlisting}

            The proposed design ensures a seamless integration of new solvers, as the responsibility of verifying whether a solver can handle a given problem lies within the solver itself. The order in which solvers are added, and consequently the sequence in which they are evaluated, dictates the preference when multiple solvers are capable of addressing the same problem.

            The assigner iterates through a predefined list of solvers, selecting the first one that successfully resolves the given problem. The Genetic Algorithm is positioned as the last resort, ensuring that it is only employed when no specialized solver is available.

        \subsection{Adding Solvers}
            Integrating new solvers into the framework is straightforward.
            %
            A new solver can be implemented by creating a class that inherits from \textit{Solver} (see Script \ref{script:solver}) and by overriding the necessary virtual functions.
            %
            Once implemented, the solver must be added to the assigner's list, as demonstrated in Script \ref{script:assinger}.

            In Section \ref{sec:implemented_solvers}, four default solvers are introduced.

\begin{lstlisting}[language=Python, caption={Solver Class.}, label={script:solver}]
class Solver:
    """
    Abstract class for solvers. All solvers should inherit from this class.
    """

    @staticmethod
    def can_solve(group_rule: GroupRule):
        """
        Check if the solver can solve the given group_rule.
        """
        raise NotImplementedError("Subclasses should implement this method.")

    @staticmethod
    def solve(group_rule: GroupRule, instances=List[object]):
        """
        Solve the given group.
        """
        raise NotImplementedError("Subclasses should implement this method.")
    
    @staticmethod
    def solve(group_rule: GroupRule, groups: List[Group]):
        """
        Solve the given group.
        """
        raise NotImplementedError("Subclasses should implement this method.")
    \end{lstlisting}

    \subsection{Implemented Solvers}
    \label{sec:implemented_solvers}
    In this section, we describe the different solvers that were implemented to address the grouping problems. Each solver is chosen based on its suitability for the problem at hand, ranging from optimization-based techniques like the Hungarian Algorithm to more heuristic-based approaches such as Genetic Algorithms. We also delve into a solver designed for specific cases like the Stable Marriage problem, which ensures a stable matching between two groups.

    It is important to notice that for this work, it is enough to demonstrate possibilities, indicating an arbitrary amount of solvers can be implemented and added to this framework in the future.

    \subsubsection{Metaheuristic}
        To implement the genetic algorithm, we utilized the DEAP library \cite{deap}, which provides a flexible framework for evolutionary algorithms in Python.
        In this implementation, individuals are represented as lists of booleans, and for matrix-based problems, we use a compressed matrix representation as a flat list.

        DEAP offers customizable genetic operators and supports user-defined fitness functions, allowing constraints to be incorporated directly. The default mutation and crossover operators from DEAP are used, enabling efficient exploration of large search spaces. This solver is particularly suitable for complex matching problems where traditional optimization algorithms are impractical and approximate solutions are sufficient.
    \subsubsection{Hungarian Algorithm}
    The Hungarian Algorithm, also known as the Kuhn-Munkres algorithm, was implemented to solve assignment problems where the goal is to find an optimal, minimum-cost matching between two sets of objects. The algorithm operates in polynomial time, making it an efficient choice for solving balanced bipartite matching problems \cite{kuhn1955hungarian}. 
    
    In this implementation, we followed the standart approach of constructing a cost matrix, where each element represents the "cost" or "penalty" of matching a pair of objects. The algorithm then attempts to minimize the total cost by identifying the optimal set of pairings that leads to the lowest overall value. This is particularly suited for scenarios where the goal is not only to match but to optimize a single objective, such as minimizing total distance or maximizing total compensation in the distribution of resources. The Hungarian Algorithm is deterministic, guaranteeing an optimal solution for these types of problems.

    \subsubsection{Binary Search and Hungarian Algorithm}

        The combination of binary search and the Hungarian Algorithm provides an efficient approach to solving matching problems where constraints on edge rankings play a crucial role. This method is particularly useful when the objective is to find a maximum matching while ensuring that the lowest-ranked edge in the matching is as high as possible.
        
        The approach consists of performing a binary search on the minimum allowable edge ranking. For a given threshold value \( T \), all edges with rankings lower than \( T \) are temporarily removed from consideration, leaving only edges with rankings \(\geq T\). The Hungarian Algorithm is then applied to find the optimal matching under these constraints. The binary search then adjusts the threshold until the maximum possible minimum ranking is achieved while maintaining a valid matching.
        
        More formally, the algorithm follows these steps:
        
        \begin{enumerate}
            \item Define the search space by setting an initial range \([L, R]\), where \( L \) is the lowest ranking in the graph and \( R \) is the highest ranking.
            \item Perform binary search on \( T \):
            \begin{itemize}
                \item Set \( T = \frac{L + R}{2} \);
                \item Remove all edges with rankings lower than \( T \);
                \item Apply the Hungarian Algorithm to determine the maximum matching under the new graph constraints;
                \item If a valid matching exists, adjust \( L \) to try increasing \( T \); otherwise, adjust \( R \).
            \end{itemize}
            \item Repeat until convergence, returning the best feasible matching where the lowest-ranked edge is maximized.
        \end{enumerate}
        
        This technique ensures that the worst-case edge ranking in the matching is as high as possible, making it particularly useful in applications where fairness or quality guarantees are required. By leveraging binary search, the solution efficiently narrows down the feasible rankings, ensuring that the Hungarian Algorithm is executed only on relevant subsets of the graph, thereby improving computational efficiency in large-scale instances.
        
    
    \subsubsection{Stable Marriage Algorithm}
    The Stable Marriage Algorithm, based on the Gale-Shapley Deferred Acceptance algorithm, was implemented to solve matching problems where stability between two groups is the primary concern. Stability in this context refers to a matching where no pair of individuals would prefer each other over their current matches, thus preventing any potential disruptions to the arrangement.
    
    For the implementation, we focused on creating an efficient solution for the stable matching problem, where each group provides a preference list, and the algorithm iteratively matches individuals based on their rankings. We applied the "men-proposing" variant of the Stable Marriage Algorithm, ensuring that once a match is formed, it is either retained or improved upon without destabilizing previous matches.
    
    This algorithm is particularly effective in applications where stability are more important than optimization, such as job candidate assignment or college admissions. The result is a matching that respects individual preferences while maintaining stability across the entire system.

    \section{Optimizer}
    For the purpose of this framework, we provide a simple optimizer that selects the appropriate solver based on the problem's characteristics.
    The optimizer is implemented in the \textit{optimize} function, which provides a direct interface for solving grouping problems.
    It accepts either a \textit{GroupRule} and a list of valid \textit{Groups}, or a \textit{GroupRule} and a list of instances.
    The optimizer internally uses the \textit{Assigner} to select the appropriate solver and applies it to the problem, abstracting away the manual selection and invocation of solvers.
    See Script \ref{script:optimizer} for details.

    \begin{lstlisting}[language=Python, caption={Optimizer.}, label={script:optimizer}]
def optimize(gr: GroupRule, instances: List[object]):
    assigner = Assigner()
    solver = assigner.choose_solver(gr)

    if all(isinstance(inst, Group) for inst in instances):
        return solver.solve_from_valid_groups(gr, instances)

    return solver.solve_from_instances(gr, instances)
\end{lstlisting}

    \section{Problem Examples} \label{sec:examples}
To illustrate the flexibility of the framework in solving various grouping problems, this section presents three distinct problems, each one with its unique requirements.

\subsection{Job Assignment}
In this example, the goal is to match workers to jobs while minimizing a cost metric (e.g., travel distance or time) as described at Section \ref{subsec:job_assignment}. In Script \ref{script:job_assignement}, we configure the framework with a cost function that dynamically evaluates the expense of assigning specific workers to jobs, enforcing a one-to-one assignment.

\begin{lstlisting}[language=Python, caption={Creating and solving the Job Assignment Problem using this framework.}, label={script:job_assignement}]
from src.group import Group, GroupRule
import random
from typing import List, Type
import itertools
from src.optimizer import optimize

# Define the Classes for the Example
class Worker:
    def __init__(self, name):
        self.name = name
        self.skills = random.randint(1, int(1e6)) # Simulating worker skill level
    def __repr__(self): return f"Worker({self.name})"

class Job:
    def __init__(self, title):
        self.title = title
        self.skills = random.randint(1, int(1e6)) # Simulating skills required
    def __repr__(self): return f"Job({self.title})"

# Define the Statistic Function
def skill_allignment(members: dict[Type, List]):
    workers = members.get(Worker, [None])[0]
    jobs = members.get(Job, [None])[0]

    return abs(workers.skills - jobs.skills)

# Create instances
workers = [
    Worker("Alice"), Worker("Bob"), Worker("Charlie"),
    Worker("Diana"), Worker("Eve"), Worker("Frank"),
    Worker("Grace"), Worker("Hank"), Worker("Ivy"),
    Worker("Jack"), Worker("Karen")
]

jobs = [
    Job("Painting"), Job("Electrical"), Job("Plumbing"),
    Job("Cleaning"), Job("Gardening")
]

# Create Groups
groups = [Group().add_member(*comb) for comb in itertools.product(workers, jobs)]

# Define the Group Rule
gr = GroupRule()
gr.set_cardinality(Worker, 1, 1)   # exactly 1 worker per job
gr.set_cardinality(Job, 1, 1)      # exactly 1 job per worker
gr.set_objective_function("minimize_sum_of_single_statistic")
gr.add_statistic(skill_allignment) # adding the skill alignment statistic

# Optimizing from a list of Groups
print("Answer: ", optimize(gr, groups))
\end{lstlisting}


\subsection{Stable Marriage}
This example in Script \ref{script:stable_marriage} demonstrates stable pairing, where two groups (e.g., males and females) are matched based on mutual preferences, further explanation can be found at Section \ref{subsec:stable_matching}. The framework allows defining preference-based rules to ensure a stable solution where no pairs would prefer a different match.
The fitness function of the following function is shown in Algorithm \ref{algo:stbl_marriage_obj_func} in Section \ref{sec:obj_func}, and its implementation is builtin, there is only need to configure the preferences and set stable matching.

\begin{lstlisting}[language=Python, caption={Creating and solving the Stable Marriage Problem using this framework.}, label={script:stable_marriage}]
from src.group import Group, GroupRule
import random
from typing import List, Type
from src.optimizer import optimize

# Define the Classes for the Example
class Man:
    def __init__(self, name, preferences=None):
        self.name = name
        self.preferences = preferences if preferences is not None else []
    def add_preference(self, preference):
        if preference not in self.preferences:
            self.preferences.append(preference)
    def __repr__(self): return f"Man({self.name})"

class Woman:
    def __init__(self, name, preferences=None):
        self.name = name
        self.preferences = preferences if preferences is not None else []
    def add_preference(self, preference):
        if preference not in self.preferences:
            self.preferences.append(preference)
    def __repr__(self): return f"Woman({self.name})"


# Define a function to validate if the preferences are repected:
# The Stabel Matching does not allow statistics, so this is a validator function.
def stable_matching_validator(members):
    males = members.get(Man, [])
    females = members.get(Woman, [])
    male_partner = {m: f for m, f in zip(males, females)}
    female_partner = {f: m for m, f in zip(males, females)}

    for m in males:
        for w_name in preferences[m.name]:
            w = next((f for f in females if f.name == w_name), None)
            if w is None:
                continue
            if female_partner[w] != m:
                current_w = male_partner[m]
                current_m = female_partner[w]

                m_prefers_w = preferences[m.name].index(w.name) < preferences[m.name].index(current_w.name)
                w_prefers_m = preferences[w.name].index(m.name) < preferences[w.name].index(current_m.name)

                if m_prefers_w and w_prefers_m:
                    return False
    return True


# Create instances
men = {
    "John":   Man("John"),   "Paul":  Man("Paul"),  "Mike":    Man("Mike"),
    "George": Man("George"), "Ringo": Man("Ringo"), "Pete":    Man("Pete"),
    "Brian":  Man("Brian"),  "Roger": Man("Roger"), "Freddie": Man("Freddie")
}

women = {
    "Mary":  Woman("Mary"),  "Patricia": Woman("Patricia"), "Susan":   Woman("Susan"),
    "Linda": Woman("Linda"), "Karen":    Woman("Karen"),    "Jessica": Woman("Jessica"),
    "Sarah": Woman("Sarah"), "Jennifer": Woman("Jennifer"), "Nancy":   Woman("Nancy")
}

# Define preferences
men_preferences = {
    "John":    [women["Mary"],     women["Linda"],    women["Susan"]   ],
    "Paul":    [women["Linda"],    women["Mary"],     women["Susan"]   ],
    "Mike":    [women["Susan"],    women["Mary"],     women["Linda"]   ],
    "George":  [women["Patricia"], women["Jennifer"], women["Jessica"] ],
    "Ringo":   [women["Jennifer"], women["Patricia"], women["Jessica"] ],
    "Pete":    [women["Jessica"],  women["Patricia"], women["Jennifer"]],
    "Brian":   [women["Sarah"],    women["Karen"],    women["Nancy"]   ],
    "Roger":   [women["Karen"],    women["Sarah"],    women["Nancy"]   ],
    "Freddie": [women["Nancy"],    women["Sarah"],    women["Karen"]   ],
}

woman_preferences = {
    "Mary":     [men["John"],    men["Paul"]   ],
    "Linda":    [men["Paul"],    men["Mike"]   ],
    "Susan":    [men["Mike"],    men["George"] ],
    "Patricia": [men["George"],  men["Ringo"]  ],
    "Jennifer": [men["Ringo"],   men["Pete"]   ],
    "Jessica":  [men["Pete"],    men["Brian"]  ],
    "Sarah":    [men["Brian"],   men["Roger"]  ],
    "Karen":    [men["Roger"],   men["Freddie"]],
    "Nancy":    [men["Freddie"], men["John"]   ],
}

for name, man in woman_preferences.items():
    women[name].add_preference(man)

for name, woman in men_preferences.items():
    men[name].add_preference(woman)

instances = list(men.values()) + list(women.values())


# Create the Group Rule
gr = GroupRule()
gr.set_cardinality(Woman, 1, 1)
gr.set_cardinality(Man, 1, 1)
gr.set_stable_match(True)                   # Enables stable matching
gr.add_validator(stable_matching_validator) # Adds the validator function

# Optimizing from a list of Instances, the groups are built automatically.
print("Answer: ", optimize(gr, instances))
\end{lstlisting}

\subsection{Timetable Scheduling}
For complex scheduling scenarios, such as assigning professors, rooms, lectures, and times, we use a genetic algorithm. This example defines a fitness function to optimize resource allocation while meeting scheduling constraints. 

\begin{lstlisting}[language=Python, caption={Creating and solving the Timetable Scheduling Problem using this framework.}, label={script:time_table}]
from src.group import Group, GroupRule
import itertools
from src.optimizer import optimize


# Define the Classes for the Example
class Professor:
    def __init__(self, name): self.name = name
    def __repr__(self): return f"Professor({self.name})"

class Room:
    def __init__(self, name): self.name = name
    def __repr__(self): return f"Room({self.name})"

class Cohort:
    def __init__(self, name): self.name = name
    def __repr__(self): return f"Cohort({self.name})"

class TimeWindow:
    def __init__(self, slot): self.slot = slot
    def __repr__(self): return f"TimeWindow({self.slot})"

class Subject:
    def __init__(self, name): self.name = name
    def __repr__(self): return f"Subject({self.name})"

# Define an objective function to evaluate the groups
def objective_function(groups):
    professors_by_time = dict()
    rooms_by_time = dict()
    score = 0
    for group in groups:
        elements = group.get_members()
        time = elements[TimeWindow][0]
        professor = elements[Professor][0]
        room = elements[Room][0]

        if time not in professors_by_time:
            professors_by_time[time] = set()
        if time not in rooms_by_time:
            rooms_by_time[time] = set()

        # Check for conflicts in professors and rooms at the same time
        if professor in professors_by_time[time]:
            score += -1e6 # Conflict penalty for professor
        else:
            score += 1e6 # Reward for only one professor at a given time
        if room in rooms_by_time[time]:
            score += -1e6 # Conflict penalty for room
        else:
            score += 1e6 # Reward for only one room at a given time

        professors_by_time[time].add(professor)
        rooms_by_time[time].add(room)

    return score


# Create instances and groups for the problem
professors =   [Professor("ProfA"), Professor("ProfB"), Professor("ProfC")]
rooms =        [Room("Room1"), Room("Room2"), Room("Room3")]
cohorts =      [Cohort("Cohort1"), Cohort("Cohort2"), Cohort("Cohort3")]
time_windows = [TimeWindow("8h"), TimeWindow("10h"), TimeWindow("12h")]
subjects =     [Subject("Math"), Subject("History"), Subject("Science")]

instances =  [professors, rooms, cohorts, time_windows, subjects]
all_groups = [Group().add_member(*comb) for comb in itertools.product(*instances)]

# Define the Group Rule
gr = GroupRule()
gr.set_cardinality(Professor, 1, 1)
gr.set_cardinality(Room, 1, 1)
gr.set_cardinality(Cohort, 1, 1)
gr.set_cardinality(TimeWindow, 1, 1)
gr.set_cardinality(Subject, 1, 1)
gr.set_objective_function(objective_function)


# Optimize from a list of Groups
print(optimize(gr, all_groups))
\end{lstlisting}

To see more information about resource allocation problems, see Section \ref{subsec:resource_allocation}.