% ---
% RESUMOS
% ---

% resumo em português
\setlength{\absparsep}{18pt} % ajusta o espaçamento dos parágrafos do resumo
\begin{resumo}[RESUMO]
    \begin{otherlanguage*}{brazil}
        Os problemas de \textit{matching} são críticos em diversas aplicações industriais, como alocação de tarefas, agendamento e distribuição de recursos.
        %
        No entanto, as soluções de otimização existentes são frequentemente complexas, rígidas ou inacessíveis para profissionais sem experiência especializada.
        %
        Prara mitigar este problema, esta dissertação propõe o design de um modelo flexível para definir e resolver problemas de \textit{matching} e agrupamento, permitindo que os usuários configurem dados relacionais e restrições de otimização sem a necessidade de um conhecimento profundo em técnicas de otimização.
        %
        O método proposto opera sobre o modelo OMR (Mapeamento Relacional de Objetos, do Inglês \textit{Object-Relational Mapping}) da aplicação, o que potencializa a adoção do método, haja visto a atual adoção do ORM pela indústria.
        %
        Além do design de um modelo, este trabalho apresenta um protótipo de uma implementação e introduz um algoritmo de matching inovador sob restrições de cotas.
        %
        Esse algoritmo garante a equidade e a viabilidade em cenários de alocação de recursos onde as cotas desempenham um papel crucial. Ao preencher a lacuna entre os modelos teóricos de matching e suas aplicações práticas, esta pesquisa oferece uma abordagem estruturada, porém adaptável, para resolver problemas reais de \textit{matching}.
    \textbf{Palavras-chave}: Matching. Agrupamento. Algoritmo. \textit{Framework}. Otimização.
    \end{otherlanguage*}
\end{resumo}
