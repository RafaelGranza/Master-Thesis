% ---
% Abstract
% ---

% resumo em inglês
\begin{resumo}[Abstract]
  \begin{otherlanguage*}{english}
    Matching problems are critical in various industrial applications, such as task allocation, scheduling, and resource distribution.
    %
    However, existing optimization solutions are often complex, rigid, or inaccessible to professionals without specialized expertise.
    %
    To mitigate this issue, this dissertation proposes the design of a flexible model to define and solve matching and grouping problems, enabling users to configure relational data and optimization constraints without requiring deep knowledge of optimization techniques.
    %
    The proposed method operates on the application's ORM (Object-Relational Mapping) model, which enhances the method's adoption given the current widespread use of ORM in the industry.
    %
    In addition to the model design, this work presents a prototype implementation and introduces an innovative matching algorithm under quota constraints.
    %
    This algorithm ensures fairness and feasibility in resource allocation scenarios where quotas play a crucial role. By bridging the gap between theoretical matching models and their practical applications, this research offers a structured yet adaptable approach to solving real-world matching problems.  
    \textbf{Keywords}: Matching. Grouping. Algorithm. Framework. Optimization.
  \end{otherlanguage*}
\end{resumo}
