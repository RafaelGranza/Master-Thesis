\begin{figure}[!ht]
    \centering
    \resizebox{0.9\textwidth}{!}{%
    \begin{tikzpicture}[scale=0.4]
        \begin{class}{Class A}{0, 6}
            \attribute {id : String}
            
        \end{class}
        \begin{class}{Class B}{15, 6}
            \attribute {id : String}
            
        \end{class}
        \begin{class}{Class C}{30, 6}
            \attribute {id : String}
            
        \end{class}
        \begin{class}{Grouping}{15, -2}
            \attribute {id : String}
            \attribute {classAId : String}
            \attribute {classBId : String}
            \attribute {classCId : String}
        \end{class}
        \association {Class A}{}{1:1}{Grouping}{0:1}{}
        \association {Grouping}{}{0:1}{Class B}{1:1}{}
        \association {Class C}{}{1:1}{Grouping}{0:1}{}
    \end{tikzpicture}
    }
    \caption{Representation of the relation between 3 differents Classes.}
    \label{fig:3_classes}
\end{figure}

\begin{figure}[!ht]
    \centering
    \resizebox{0.279\textwidth}{!}{%
    \begin{tikzpicture}[scale=0.4]
        \begin{class}{Class A}{0, 15}
            \attribute {id : String}
                        
        \end{class}
        \begin{class}{Grouping}{0, 6}
            \attribute {id : String}
            \attribute {ids : Set<String>}
        \end{class}
        \association {Class A}{}{1:N}{Grouping}{0:N}{}
    \end{tikzpicture}
    }
    \caption{Representation of the relation between elements of a same Class.}
    \label{fig:1_class}
\end{figure}