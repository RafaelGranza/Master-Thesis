
\begin{figure}[!ht]
    \centering
    \begin{subfigure}[b]{0.47\textwidth}
    \centering
    \resizebox{1\textwidth}{!}{%
    \begin{tikzpicture}[scale=0.5]
        \begin{class}{Person}{0, 8}
            \attribute {id : String}
            
        \end{class}
        \begin{class}{Teacher}{0, 0}
            \attribute {id : String}
            
        \end{class}
        \begin{class}{Subject}{15, 0}
            \attribute {id : String}
            
        \end{class}
        \begin{class}{Course}{15, 8}
            \attribute {id : String}
            
        \end{class}
        \association {Teacher}{}{0..N}{Subject}{0..N}{}
        \association {Teacher}{}{0..1}{Person}{0..1}{}
        \association {Course}{}{0..1}{Subject}{0..N}{}
    \end{tikzpicture}
    }
    \caption{Representation of simple relations between entities of a school.}
    \label{fig:uml_teacher_subject_implicit_example}
    \end{subfigure}
    \hfill
    \begin{subfigure}[b]{0.47\textwidth}
    \centering
    \resizebox{1\textwidth}{!}{%
    \begin{tikzpicture}[scale=0.5]
        \begin{class}{Person}{0, 8}
            \attribute {id : String}
            
        \end{class}
        \begin{class}{Teacher}{0, 0}
            \attribute {id : String}
            
        \end{class}
        \begin{class}{Subject}{15, 0}
            \attribute {id : String}
            
        \end{class}
        \begin{class}{Course}{15, 8}
            \attribute {id : String}
            
        \end{class}
        \begin{class}{Assignment}{7.5, -8}
            \attribute {id : String}
            \attribute {professorId : String}
            \attribute {classId : String}
        \end{class}
        \association {Teacher}{}{0..1}{Assignment}{0..1}{}
        \association {Assignment}{}{0..1}{Subject}{0..1}{}
        \association {Teacher}{}{0..1}{Person}{0..1}{}
        \association {Course}{}{0..1}{Subject}{0..N}{}
    \end{tikzpicture}
    }
    \caption{New Class \textit{Assignment} to represent the many-to-many relation of Figure \ref{fig:uml_teacher_subject_implicit_example}.}
    \label{fig:uml_teacher_new_class}
    \end{subfigure}
\end{figure}