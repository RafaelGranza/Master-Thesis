\begin{figure}[!ht]
    \centering
    \begin{subfigure}[b]{0.45\textwidth}
        \centering
        \scalebox{0.6}{%
        \begin{tikzpicture}[scale=0.4]
            \begin{class}{User}{0, 6}
                \attribute {id : String}
            \end{class}
            \begin{class}{Item}{15, 6}
                \attribute {id : String}
            \end{class}

            \begin{class}{Grouping}{7.5, -2}
                \attribute {id : String}
                \attribute {UserId : String}
                \attribute {ItemId : String}
            \end{class}
            \association {User}{}{1..1}{Grouping}{0..1}{}
            \association {Grouping}{}{0..1}{Item}{1..1}{}
        \end{tikzpicture}
        }
        \caption{An user has to be assigned to a single item.}
        \label{fig:1_!}
    \end{subfigure}
    \hfill
    \begin{subfigure}[b]{0.45\textwidth}
        \centering
        \scalebox{0.6}{%
        \begin{tikzpicture}[scale=0.4]
            \begin{class}{User}{0, 6}
                \attribute {id : String}
            \end{class}
            \begin{class}{Item}{15, 6}
                \attribute {id : String}
            \end{class}

            \begin{class}{Grouping}{7.5, -2}
                \attribute {id : String}
                \attribute {UserId : String}
                \attribute {ItemId : String}
            \end{class}
            \association {User}{}{1..1}{Grouping}{0..1}{}
            \association {Grouping}{}{0..1}{Item}{1..N}{}
        \end{tikzpicture}
        }
        \caption{An user can be assigned to multiple items.}
        \label{fig:1_N}
    \end{subfigure}
    
    \caption{The User-Item Recommendation Problem showcases how different dimensions on the relations result in different restrictions.
            In both examples a grouping must have an user, but not all users have an assignment.}
    \label{fig:relation_geometry}
\end{figure}