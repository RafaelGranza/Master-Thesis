\begin{figure}[!ht]
    \centering
    \begin{subfigure}[b]{0.45\textwidth}
        \centering
        \scalebox{0.6}{%
        \begin{tikzpicture}[scale=0.4]
            \begin{class}{Worker}{0, 6}
                \attribute {id : String}
            \end{class}
            \begin{class}{Job}{15, 6}
                \attribute {id : String}
            \end{class}

            \begin{class}{Grouping}{7.5, -2}
                \attribute {id : String}
                \attribute {JobId : String}
                \attribute {WorkerId : String}
            \end{class}
            \association {Worker}{}{1:1}{Grouping}{0:1}{}
            \association {Grouping}{}{1:1}{Job}{1:1}{}
        \end{tikzpicture}
        }
        \caption{Job Assignment Problem.}
        \label{fig:uml_job_matching_problem}
    \end{subfigure}
    \hfill
    \begin{subfigure}[b]{0.45\textwidth}
        \centering
        \scalebox{0.6}{%
        \begin{tikzpicture}[scale=0.4]
            \begin{class}{Room}{0, 6}
                \attribute {id : String}
            \end{class}
            \begin{class}{Fund}{16, 6}
                \attribute {id : String}
            \end{class}
            \begin{class}{Project}{0, -3.5}
                \attribute {id : String}
            \end{class}
            \begin{class}{Grouping}{16, -2}
                \attribute {id : String}
                \attribute {roomId : String}
                \attribute {fundId : String}
                \attribute {projectId : String}
            \end{class}
            \association {Room}{}{1:N}{Grouping}{0:1}{}
            \association {Grouping}{}{0:N}{Fund}{0:N}{}
            \association {Project}{}{1:1}{Grouping}{1:1}{}
        \end{tikzpicture}
        }
        \caption{Resource Allocation Problem.}
        \label{fig:uml_resorce_allocation_problem}
    \end{subfigure}
    \hfill
    \begin{subfigure}[b]{0.45\textwidth}
        \centering
        \scalebox{0.6}{%
        \begin{tikzpicture}[scale=0.4]
            \begin{class}{User}{0, 6}
                \attribute {id : String}
            \end{class}
            \begin{class}{Item}{15, 6}
                \attribute {id : String}
            \end{class}

            \begin{class}{Grouping}{7.5, -2}
                \attribute {id : String}
                \attribute {UserId : String}
                \attribute {ItemrId : String}
            \end{class}
            \association {User}{}{1:1}{Grouping}{1:1}{}
            \association {Grouping}{}{1:1}{Item}{1:N}{}
        \end{tikzpicture}
        }
        \caption{User-Item recommendation Problem.}
        \label{fig:uml_recomendation_problem}
    \end{subfigure}
    \hfill
    \begin{subfigure}[b]{0.45\textwidth}
        \centering
        \scalebox{0.6}{%
        \begin{tikzpicture}[scale=0.4]
            \begin{class}{Image}{0, 6}
                \attribute {id : String}
            \end{class}

            \begin{class}{Grouping}{0, -2}
                \attribute {id : String}
                \attribute {image1Id : String}
                \attribute {image2Id : String}
            \end{class}
            \association {Image}{}{2..2}{Grouping}{0:1}{}
        \end{tikzpicture}
        }
        \caption{Image Matching Problem.}
        \label{fig:uml_image_matching_problem}
    \end{subfigure}
    
    \caption{Visualizing Grouping Problems as a relational diagram.}
    \label{fig:examples_grouping_as_relations}
\end{figure}