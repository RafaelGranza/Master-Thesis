%% abtex2-modelo-trabalho-academico.tex, v-1.9.6 laurocesar
%% Copyright 2012-2016 by abnTeX2 group at http://www.abntex.net.br/ 
%%
%% This work may be distributed and/or modified under the
%% conditions of the LaTeX Project Public License, either version 1.3
%% of this license or (at your option) any later version.
%% The latest version of this license is in
%%   http://www.latex-project.org/lppl.txt
%% and version 1.3 or later is part of all distributions of LaTeX
%% version 2005/12/01 or later.
%%
%% This work has the L PPL maintenance status `maintained'.
%% 
%% The Current Maintainer of this work is the abnTeX2 team, led
%% by Lauro César Araujo. Further information are available on 
%% http://www.abntex.net.br/
%%
%% This work consists of the files abntex2-modelo-trabalho-academico.tex,
%% abntex2-modelo-include-comandos and abntex2-modelo-references.bib
% ------------------------------------------------------------------------
% ------------------------------------------------------------------------
% abnTeX2: Modelo de Trabalho Academico (tese de doutorado, dissertacao de
% mestrado e trabalhos monograficos em geral) em conformidade com 
% ABNT NBR 14724:2011: Informacao e documentacao - Trabalhos academicos -
% Apresentacao
% ------------------------------------------------------------------------
% ------------------------------------------------------------------------
% Personalização para o modelo Udesc 2020 7. ed. revisada e modificada
% MANUAL_2020_09_07_1599489825065_12510.pdf
% Autor: Felipe Joel Zimann (felipezimann@hotmail.com)
% Data: 02/12/2020 v1.0
% Data: 13/02/2021 v1.0.1 alterado tamanho numeração da página para 10pt
% ------------------------------------------------------------------------
% ------------------------------------------------------------------------

\documentclass[
	12pt,					% tamanho da fonte
	openright,				% capítulos começam em pág ímpar (insere página vazia caso preciso)
	oneside,				% para impressão em recto e verso (twoside). Oposto a (oneside)
	a4paper,				% tamanho do papel. 
	chapter=TITLE,			% títulos de capítulos convertidos em letras maiúsculas
	section=TITLE,			% títulos de seções convertidos em letras maiúsculas
	sumario=abnt-6027-2012,
	english,				% idioma adicional para hifenização
	brazil,					% o último idioma é o principal do documento
	fleqn,					% equações alinhadas a esquerda (UDESC/CCT)+
	]{abntex2}

% ----------------------------------------------------------
% Pacotes básicos 
% ----------------------------------------------------------
\usepackage{amsmath}							% Pacote matemático
\usepackage{amssymb}							% Pacote matemático
\usepackage{pdfpages}

\usepackage{amsfonts}							% Pacote matemático
%\usepackage{lmodern}							% Usa a fonte Latin Modern		
\usepackage{mathptmx} 							% Usa a fonte Times New Roman	 (UDESC/CCT)
\usepackage{tocloft}

\usepackage[T1]{fontenc}						% Selecao de codigos de fonte.
\usepackage[utf8]{inputenc}						% Codificacao do documento (conversão automática dos acentos)
\usepackage{lastpage}							% Usado pela Ficha catalográfica
\usepackage{indentfirst}						% Indenta o primeiro parágrafo de cada seção.
\usepackage{ragged2e} % Para \RaggedRight
\usepackage{multirow}
\usepackage{listings}

\renewcommand\lstlistingname{Script}
\renewcommand\lstlistlistingname{Scripts}

\lstset{ 
  language=Python,
  backgroundcolor=\color{white},   
  basicstyle=\ttfamily\tiny,
  breakatwhitespace=false,        
  breaklines=true,                 
  captionpos=b,                    
  commentstyle=\color{red},    
  keywordstyle=\color{blue},       
  stringstyle=\color{orange},     
  frame=single,                    
  rulecolor=\color{black},         
  showspaces=false,                
  showstringspaces=false,          
  showtabs=false,                  
  tabsize=2,
  title=\lstname                    
}

\usepackage{amssymb,amsmath,amsthm}

\newtheorem{lemma}{Lemma}
\newtheorem{theorem}{Theorem}



\usepackage{graphicx}							% Inclusão de gráficos
\usepackage{microtype} 							% para melhorias de justificação
\usepackage{lipsum}								% para geração de dummy text
\usepackage[english,hyperpageref]{backref}	% Paginas com as citações na bibl
%\usepackage[alf,abnt-emphasize=bf,abnt-full-initials=yes]{abntex2cite}					% Citações padrão ABNT
\usepackage[num]{abntex2cite}					% Citações padrão ABNT numérica


\usepackage{adjustbox}							% Pacote de ajuste de boxes
\usepackage{pgf-umlcd}
\newcommand{\doubleassossiation}[4]
{
    \draw[umlcd style] (#1.east) to[out=0, in=0, looseness=1.5] (#4.east)
    node[pos=0.75, above]{#2}
    node[pos=0.25, below]{#3};
    \draw[umlcd style] (#1.west) to[out=180, in=180, looseness=1.5] (#4.west)
    node[pos=0.75, above]{#2}
    node[pos=0.25, below]{#3};
}
\renewcommand{\umldrawcolor}{black}
\renewcommand{\umlfillcolor}{white}


\usepackage{subcaption}							% Inclusão de Subfiguras e sublegendas	
\usepackage{enumitem}							% Personalização de listas
\usepackage{siunitx}							% Grandezas e unidades
\usepackage[section]{placeins}					% Manter as figuras delimitadas na respectiva seção com a opção [section]
\usepackage{multirow}							% Multi colunas nas tabelas
\usepackage{array,tabularx} 					% Pacotes de tabelas
\usepackage{booktabs}							% Pacote de tabela profissonal
\usepackage{rotating}							% Rotacionar figuras e tabelas
\usepackage{xfrac}								% Fazer frações n/d em linha
\usepackage{bm}									% Negrito em modo matemático
\usepackage{xstring}							% Manipulação de strings
\usepackage{pgfplots}							% Pacote de Gráficos
\usepackage{tikz}								% Pacote de Figuras
\usepackage{graphicx}
\usepackage{array}
\usepackage{longtable}
\usepackage{algorithm}
\usepackage{algorithmic}


\usepackage[american, cuteinductors,smartlabels, fulldiode, siunitx, americanvoltages, oldvoltagedirection, smartlabels]{circuitikz}						% Pacote de circuitos elétricos
\usepackage{chemformula}						% Pacote para fórmulas químicas
\usepackage{chngcntr}							% Pacte usado para deixar numeração de equações sequencial (UDESC/CCT)
\counterwithout{equation}{chapter}
% fonte: https://latex.org/forum/viewtopic.php?t=15392

% Comando para deixar numeração das equações contínua (1), (2), (3)... ao invés de organizar por capítulos (1.1)(1.2)... (2.1)(2.2)
%\renewcommand{\theequation}{\arabic{equation}}

%\numberwithin{equation}{section}
\usepackage{hyperref}


% Cabecalho cabeçalho somente com numeração de página 10pt
\makepagestyle{PagNumReduzida}
\makeevenhead{PagNumReduzida}{\ABNTEXfontereduzida\thepage}{}{}
\makeoddhead{PagNumReduzida}{}{}{\ABNTEXfontereduzida\thepage}
%fonte: https://github.com/abntex/abntex2/wiki/HowToCustomizarCabecalhoRodape
%fonte: Manual memoir seção 7.3 pg. 111 pdf http://linorg.usp.br/CTAN/macros/latex/contrib/memoir/memman.pdf 

% Personalização das opções das listas
\setlist[itemize]{leftmargin=\parindent}

% Citação online --- MODIFICAR ---
% \newcommand{\citeshort}[1]{\citeauthoronline{#1}~(\citeyear{#1})}

% \newcommand{\me}[1]{Elaborado pelo autor (#1).}

% Configuração do pgfplots
\pgfplotsset{compat=newest} %compat=1.14
\pgfplotsset{plot coordinates/math parser=false} 
\newlength\figureheight 
\newlength\figurewidth 

% Libraries do TiKz
\usetikzlibrary{quotes,angles,arrows}
\usetikzlibrary{through,calc,math}
\usetikzlibrary{graphs,backgrounds,fit}
\usetikzlibrary{shapes,positioning,patterns,shadows}
\usetikzlibrary{decorations.pathreplacing}
\usetikzlibrary{shapes.geometric}
\usetikzlibrary{arrows.meta}
\usetikzlibrary{external}

%\tikzexternalize[]
%\tikzexternalenable
%\tikzexternalize
%\tikzexternaldisable
%\tikzset{external/force remake}
%\tikzexternalize[shell escape=-enable-write18]

% Configurações do CircuiTiKz
\ctikzset{bipoles/thickness=1}
%\ctikzset{bipoles/length=1.2cm}
\ctikzset{monopoles/ground/width/.initial=.2}
\ctikzset{bipoles/resistor/height=0.25}
\ctikzset{bipoles/resistor/width=0.6}
\ctikzset{bipoles/capacitor/height=0.5}
\ctikzset{bipoles/capacitor/width=0.15}
\ctikzset{bipoles/generic/height=0.25}
\ctikzset{bipoles/generic/width=0.6}
%\ctikzset{bipoles/capacitor polar/length=0.5}
%\ctikzset{bipoles/diode/height=.375}
%\ctikzset{bipoles/diode/width=.3}
%\ctikzset{tripoles/thyristor/height=.8}
%\ctikzset{tripoles/thyristor/width=1}
\ctikzset{bipoles/vsourcesin/height=.5}
\ctikzset{bipoles/vsourcesin/width=.5}
\ctikzset{bipoles/cvsourceam/height=.6}
\ctikzset{bipoles/cvsourceam/width=.6}
%\ctikzset{tripoles/european controlled voltage source/width=.4}

\tikzstyle{every node}=[font=\footnotesize]
\tikzstyle{every path}=[line width=0.25pt,line cap=round,line join=round]
%\tikzstyle{every path}=[line cap=round,line join=round]


% Definição de cores MATLAB
\definecolor{matlab_blue}{rgb}	{         0,    0.4470,    0.7410}
\definecolor{matlab_orange}{rgb}{    0.8500,    0.3250,    0.0980}
\definecolor{matlab_yellow}{rgb}{    0.9290,    0.6940,    0.1250}
\definecolor{matlab_violet}{rgb}{    0.4940,    0.1840,    0.5560}
\definecolor{matlab_green}{rgb}	{	 0.4660,    0.6740,    0.1880}
\definecolor{matlab_lblue}{rgb}	{    0.3010,    0.7450,    0.9330}
\definecolor{matlab_red}{rgb}	{    0.6350,    0.0780,    0.1840}

% Personalização das legendas
\usepackage[format = plain, %hang
			justification = centering,
			labelsep = endash,
			singlelinecheck = false,
			skip = 6pt,
			listformat = simple]{caption}	

% Personalização das unidades
\sisetup{output-decimal-marker = {,}}
\sisetup{exponent-product = \cdot, output-product = \cdot}
\sisetup{tight-spacing=true}
\sisetup{group-digits = false}

% Personalizações de tipo de colunas de tabelas
\newcolumntype{L}[1]{>{\raggedright\let\newline\\\arraybackslash\hspace{0pt}}m{#1}}
\newcolumntype{C}[1]{>{\centering\let\newline\\\arraybackslash\hspace{0pt}}m{#1}}
\newcolumntype{R}[1]{>{\raggedleft\let\newline\\\arraybackslash\hspace{0pt}}m{#1}}

% Personalizações de cores da UDESC
\definecolor{CapaAmareloUDESC}{RGB}{243,186,83}		% Especializacao
\definecolor{CapaVerdeUDESC}{RGB}{0,112,52}			% Mestrado
\definecolor{CapaVermelhoUDESC}{RGB}{171,35,21}		% Doutorado
\definecolor{CapaAzulUDESC}{RGB}{38,54,118} 		% Pós-Doutorado

% CONFIGURAÇÕES DE PACOTES
% Configurações do pacote backref
% Usado sem a opção hyperpageref de backref
\renewcommand{\backrefpagesname}{Citado na(s) página(s):~}
% Texto padrão antes do número das páginas
\renewcommand{\backref}{}
% Define os textos da citação
\renewcommand*{\backrefalt}[4]{
	\ifcase #1 %
	Nenhuma citação no texto.%
	\or
	Citado na página #2.%
	\else
	Citado #1 vezes nas páginas #2.%
	\fi}%

% alterando o aspecto da cor azul
%\definecolor{blue}{RGB}{41,5,195}

% informações do PDF
\makeatletter
\hypersetup{
	%pagebackref=true,
	pdftitle={\@title}, 
	pdfauthor={\@author},
	pdfsubject={\imprimirpreambulo},
	pdfcreator={LaTeX with abnTeX2},
	pdfkeywords={abnt}{latex}{abntex}{abntex2}{trabalho academico}, 
	colorlinks=true,       		% false: boxed links; true: colored links
	linkcolor=black,          	% color of internal links
	citecolor=black,        	% color of links to bibliography
	filecolor=black,      		% color of file links
	urlcolor=black,
	bookmarksdepth=4
}
\makeatother


\makeatletter
\newcommand{\includetikz}[1]{%
	\tikzsetnextfilename{#1}%
	\input{#1.tex}%
}
\makeatother

% % ---
% % Possibilita criação de Quadros e Lista de quadros.
% % Ver https://github.com/abntex/abntex2/issues/176
% %
% \newcommand{\quadroname}{Quadro}
% \newcommand{\listofquadrosname}{Lista de quadros}

% \newfloat[chapter]{quadro}{loq}{\quadroname}
% \newlistof{listofquadros}{loq}{\listofquadrosname}
% \newlistentry{quadro}{loq}{0}

% % configurações para atender às regras da ABNT
% \setfloatadjustment{quadro}{\centering}
% \counterwithout{quadro}{chapter}
% \renewcommand{\cftquadroname}{\quadroname\space} 
% \renewcommand*{\cftquadroaftersnum}{\hfill--\hfill}

% \setfloatlocations{quadro}{hbtp} % Ver https://github.com/abntex/abntex2/issues/176
% % ---


% Espaçamento depois do título
\setlength{\afterchapskip}{0.7\baselineskip}
% O tamanho do parágrafo é dado por:
\setlength{\parindent}{1.25cm}
% Controle do espaçamento entre um parágrafo e outro:
\setlength{\parskip}{0.0cm}  % tente também \onelineskip
%\SingleSpacing % Espaçamento simples 
\OnehalfSpacing % Espaçamento 1,5 (UDESC/CCT)
%\DoubleSpacing	% Espaçamento duplo

% ---
% Margens - NBR 14724/2011 - 5.1 Formato
% ---
\setlrmarginsandblock{3cm}{2cm}{*}
\setulmarginsandblock{3cm}{2cm}{*}
\checkandfixthelayout[fixed]
% ---


% To use externalize consider
%https://tex.stackexchange.com/questions/182783/tikzexternalize-not-compatible-with-miktex-2-9-abntex2-package
%Lauro Cesar digged into the problem until he came with a solution for me to test. And it Works!
%
%According to this link:
%
%The package calc changed the commands \setcounter and friends to be fragile. So you have to make them robust. The example below uses etoolbox with \robustify:
%
\usepackage{etoolbox}
\robustify\setcounter
\robustify\addtocounter
\robustify\setlength
\robustify\addtolength


%% How to silence memoir class warning against the use of caption package?
%% https://tex.stackexchange.com/questions/391993/how-to-silence-memoir-class-warning-against-the-use-of-caption-package
%\usepackage{silence}
%\WarningFilter*{memoir}{You are using the caption package with the memoir class}
%\WarningFilter*{Class memoir Warning}{You are using the caption package with the memoir class}

% --------------------------------------------------------
% INICIO DAS CUSTOMIZACOES PARA A UDESC
% --------------------------------------------------------

% --------------------------------------------------------
% Fontes padroes de part, chapter, section, subsection e subsubsection
% --------------------------------------------------------
% --- Chapter ---
\renewcommand{\ABNTEXchapterfont}{\fontseries{b}} %\bfseries
\renewcommand{\ABNTEXchapterfontsize}{\normalsize}
% --- Part ---
\renewcommand{\ABNTEXpartfont}{\ABNTEXchapterfont}
\renewcommand{\ABNTEXpartfontsize}{\LARGE}
% --- Section ---
\renewcommand{\ABNTEXsectionfont}{\normalfont}
\renewcommand{\ABNTEXsectionfontsize}{\normalsize}
% --- SubSection ---
\renewcommand{\ABNTEXsubsectionfont}{\fontseries{b}} %\bfseries
\renewcommand{\ABNTEXsubsectionfontsize}{\normalsize}
% --- SubSubSection ---
\renewcommand{\ABNTEXsubsubsectionfont}{\itshape}
\renewcommand{\ABNTEXsubsubsectionfontsize}{\normalsize}

\renewcommand{\ABNTEXsubsubsubsectionfont}{\normalfont}
\renewcommand{\ABNTEXsubsubsubsectionfontsize}{\normalsize}
% ---

% --------------------------------------------------------
% Fontes das entradas do sumario
% --------------------------------------------------------

\renewcommand{\cftpartfont}{\ABNTEXpartfont\selectfont}
\renewcommand{\cftpartpagefont}{\normalsize\selectfont}

\renewcommand{\cftchapterfont}{\ABNTEXchapterfont\selectfont}
\renewcommand{\cftchapterpagefont}{\normalsize\selectfont}

\renewcommand{\cftsectionfont}{\ABNTEXsectionfont\selectfont}
\renewcommand{\cftsectionpagefont}{\normalsize\selectfont}

\renewcommand{\cftsubsectionfont}{\ABNTEXsubsectionfont\selectfont}
\renewcommand{\cftsubsectionpagefont}{\normalsize\selectfont}

\renewcommand{\cftsubsubsectionfont}{\normalfont\itshape\selectfont}
\renewcommand{\cftsubsubsectionpagefont}{\normalsize\selectfont}

\renewcommand{\cftparagraphfont}{\normalfont\selectfont}
\renewcommand{\cftparagraphpagefont}{\normalsize\selectfont}

% --------------------------------------------------------
% Usando os pacotes hyperref, uppercase... 
% Para deixar a section do toc uppercase precisa de:
% --------------------------------------------------------
\usepackage{textcase}

\makeatletter

\let\oldcontentsline\contentsline
\def\contentsline#1#2{%
	\expandafter\ifx\csname l@#1\endcsname\l@section
	\expandafter\@firstoftwo
	\else
	\expandafter\@secondoftwo
	\fi
	{%
		\oldcontentsline{#1}{\MakeTextUppercase{#2}}%
	}{%
		\oldcontentsline{#1}{#2}%
	}%
}
\makeatother

% --------------------------------------------------------
% Renomenando as entradas de APÊNDICES E ANEXOS
% --------------------------------------------------------

\renewcommand{\apendicesname}{AP\^ENDICES}
\renewcommand{\anexosname}{ANEXOS}


% Manipulação de Strings
%\RequirePackage{xstring}

% Comando para inverter sobrenome e nome
\newcommand{\invertname}[1]{%
	\StrBehind{#1}{{}}, \StrBefore{#1}{{}}%
}%


% --------------------------------------------------------
% Alterando os estilos de Caption e Fonte
% --------------------------------------------------------
\makeatletter
% Define o comando \fonte que respeita as configurações de caption do memoir ou do caption
\renewcommand{\fonte}[2][\fontename]{%
	\M@gettitle{#2}%
	\memlegendinfo{#2}%
	\par
	\begingroup
	\@parboxrestore
	\if@minipage
	\@setminipage
	\fi
	\ABNTEXfontereduzida
	\configureseparator
	\captiondelim{\ABNTEXcaptionfontedelim}
	\@makecaption{#1}{\ignorespaces #2}\par
	\endgroup}


\captionstyle[\raggedright]{\raggedright}

\makeatother

\setlength{\cftbeforechapterskip}{0pt plus 0pt}
\renewcommand*{\insertchapterspace}{}

\newlength{\mylen}	% New length to use with spacing
\setlength{\mylen}{1pt}

\setlength{\cftbeforechapterskip}{\mylen}
\setlength{\cftbeforesectionskip}{\mylen}
\setlength{\cftbeforesubsectionskip}{\mylen}
\setlength{\cftbeforesubsubsectionskip}{\mylen}
\setlength{\cftbeforesubsubsubsectionskip}{\mylen}


% ---
% Ajuste das listas de abreviaturas e siglas ; e símbolos [Personalizada para UDESC com espaçamento 1,5]
% ---

% ---
% Redefinição da Lista de abreviaturas e siglas [Personalizada para UDESC com espaçamento 1,5]
\renewenvironment{siglas}{%
	\pretextualchapter{\listadesiglasname}
	\begin{symbols} 
		\setlength{\itemsep}{0pt}	% Ajuste para Espaçamento 1,5 (UDESC/CCT)
	}{% 
	\end{symbols}
	\cleardoublepage
}
% ---

% ---
% Redefinição da Lista de símbolos [Personalizada para UDESC com espaçamento 1,5]
\renewenvironment{simbolos}{%
	\pretextualchapter{\listadesimbolosname}
	\begin{symbols}
		\setlength{\itemsep}{0pt}	% Ajuste para Espaçamento 1,5 (UDESC/CCT)
	}{%
	\end{symbols}
	\cleardoublepage
}
% ---

\captionsetup{labelformat=simple, labelsep=colon}
\renewcommand{\figurename}{Figure}
\renewcommand{\tablename}{Table}


\usetikzlibrary{patterns}



% ---
% FIM DAS CUSTOMIZACOES PARA A  Universidade do Estado de Santa Catarina - UDESC/CCT
% ---





	% Incliu pacotes básicos 

% -----------------------------------------------------------------
% Você pode adicionar seus pacotes a partir desta linha;
% -----------------------------------------------------------------

%\usepackage[showframe,pass]{geometry}
%\usepackage[11,12]{pagesel}

% -----------------------------------------------------------------
% Informações de dados para CAPA e FOLHA DE ROSTO
% -----------------------------------------------------------------
% \titulo{Ferramenta para facilitação de otimização de Emparaelhamento com foco em desenvolvedores web sem conhecimento prévio em emparelhamento}%

\titulo{Categorised Grouping: A framework for industry applications}%

\autor{Rafael{} Granza de Mello}%
\orientador{Yuri{} Kaszubowski Lopes}%
\coorientador{Rui Jorge{} Tramontin Junior}%

% ATENÇÃO: O símbolo {} indica o sobrenome para a ficha catalográfica.
% Exemplo: Sherlock Holmes {}da Silva para sobrenomes compostos;
% Exemplo: Arnold Alois {}Schwarzenegger para sobrenome simples.

\instituicao{Universidade do Estado de Santa Catarina, Centro de Ciências Tecnológicas, Programa de Pós--Graduação em Computação Aplicada}%

%\tipotrabalho{Tese (Doutorado)}
\tipotrabalho{Dissertação (Mestrado)}

%\preambulo{Tese apresentada ao Programa de Pós--Graduação em Computação Aplicada do Centro de Ciências Tecnológicas da Universidade do Estado de Santa Catarina, como requisito parcial para a obtenção do grau de Doutor em Computação Aplicada.}

\preambulo{Master Dissertation submitted to the Graduate Program in Applied Computing of the Centro de Ciências Tecnológicas at the Universidade do Estado de Santa Catarina, as a partial requirement for obtaining the degree of Master in Applied Computing.}

\local{Joinville}%

\data{\the\year}%
% ---

% compila o indice
\makeindex

% -----------------------------------------------------------------
% Início do documento
% -----------------------------------------------------------------
\begin{document}

\selectlanguage{english}
\frenchspacing  % Retira espaço extra obsoleto entre as frases.

% -----------------------------------------------------------------
% ELEMENTOS PRÉ-TEXTUAIS
% -----------------------------------------------------------------
\pretextual

% Você pode comentar os elementos que não deseja em seu trabalho;

% A capa pode ser escolhida dentro do arquivo Capa.tex (TCC, Master, Doc, ...)
% ---
% Capa
% ---


% --------------------------------------------------------
% Capa Padrão
% --------------------------------------------------------
\renewcommand{\imprimircapa}{%
	\begin{capa}%
		\center

		{\fontseries{b}\selectfont\MakeTextUppercase{UNIVERSIDADE DO ESTADO DE SANTA CATARINA -- UDESC}}
		
		{\fontseries{b}\selectfont\MakeTextUppercase{CENTRO DE CIÊNCIAS TECNOLÓGICAS -- CCT  }}
		
		{\fontseries{b}\selectfont\MakeTextUppercase{PROGRAMA DE PÓS-GRADUAÇÃO -- PPGCAP  }}
		
		\vfill
		
		{\fontseries{b}\selectfont\MakeTextUppercase{\normalsize\imprimirautor}}
		
		\vfill
		\begin{center}
			{\fontseries{b}\selectfont\MakeTextUppercase{\imprimirtitulo}}
		\end{center}
		\vfill
		
		\vfill
		
		{\fontseries{b}\selectfont\MakeTextUppercase{\imprimirlocal}}
		\par
		{\fontseries{b}\selectfont \imprimirdata}
		\vspace*{1cm}
	\end{capa}
}



\imprimircapa				% Capa padrão

					% Elemento Obrigatório
% ---
% Folha de rosto
% ---








% --------------------------------------------------------
% folha de rosto 
% --------------------------------------------------------

\makeatletter

\renewcommand{\folhaderostocontent}{
	\begin{center}
		
		{\fontseries{b}\selectfont\MakeTextUppercase{\imprimirautor}}
		
		\vfill
		
		\begin{center}
			{\fontseries{b}\selectfont\MakeTextUppercase{\imprimirtitulo}}
		\end{center}
	
		\vspace*{1.5cm}

		\abntex@ifnotempty{\imprimirpreambulo}{%
			\hspace{.45\textwidth}
			{\begin{minipage}{.5\textwidth}
					\SingleSpacing
					\imprimirpreambulo\par
					\vspace*{4pt}
					{\imprimirorientadorRotulo~\imprimirorientador\par}
					\abntex@ifnotempty{\imprimircoorientador}{%
						{\imprimircoorientadorRotulo~\imprimircoorientador}%
					}%
			\end{minipage}}%
		}%
	
		
		\vfill
		
	{\fontseries{b}\selectfont\MakeTextUppercase{\imprimirlocal}}
	\par
	{\fontseries{b}\selectfont \imprimirdata}
	\vspace*{1cm}
	\end{center}
}


% (o * indica que haverá a ficha bibliográfica)
% ---
\imprimirfolhaderosto*
% ---


			% Elemento Obrigatório
% Caso não utilize a Ficha Catalográfica entre na folha de rosto e retire o * de dentro do arquivo FolhadeRosto

% ---
% Inserir a ficha bibliografica
% ---

% Isto é um exemplo de Ficha Catalográfica, ou ``Dados internacionais de
% catalogação-na-publicação''. Você pode utilizar este modelo como referência. 
% Porém, provavelmente a biblioteca da sua universidade lhe fornecerá um PDF
% com a ficha catalográfica definitiva após a defesa do trabalho. Quando estiver
% com o documento, salve-o como PDF no diretório do seu projeto e substitua todo
% o conteúdo de implementação deste arquivo pelo comando abaixo:



% \begin{fichacatalografica}
%     \includepdf{fig_ficha_catalografica.pdf}
% \end{fichacatalografica}


%	\setlength{\parindent}{0cm}
%	\setlength{\parskip}{0pt}
\begin{fichacatalografica}
	%\sffamily
	%\rmfamily
	\ttfamily \hbadness=10000
	\vspace*{\fill}					% Posição vertical
	\begin{center}					% Minipage Centralizado
	Para gerar a ficha catalográfica de teses e \\ 
	dissertações acessar o link:  \\
	https://www.udesc.br/bu/manuais/ficha
	
	\vspace*{8pt}
	
%	\begin{minipage}[c]{8cm}
%	\centering \sffamily
%	 Ficha catalográfica elaborada pelo(a) autor(a), com auxílio do programa de geração automática da Biblioteca Setorial do CCT/UDESC
%	\end{minipage}
	\fbox{\begin{minipage}[c]{12.5cm}		% Largura
	\flushright
	{\begin{minipage}[c]{10.5cm}		% Largura
	\vspace{1.25cm}
	%\footnotesize
	\setlength{\parindent}{1.5em}
	\noindent \invertname{\imprimirautor} \par
	\imprimirtitulo{ }/{ }\imprimirautor. -- \imprimirlocal, \imprimirdata .\par
	\pageref{LastPage} p. : il. ; 30 cm.\par
	\vspace{1.5em}
	\imprimirorientadorRotulo~\imprimirorientador.\par
	\imprimircoorientadorRotulo~\imprimircoorientador.\par
	\imprimirtipotrabalho~--~\imprimirinstituicao, \imprimirlocal, \imprimirdata.\par
	\vspace{1.5em}
		1. Palavra-chave.
		2. Palavra-chave.
		3. Palavra-chave.
 		4. Palavra-chave.
		5. Palavra-chave.
		I. \invertname{\imprimirorientador}.
		II. \invertname{\imprimircoorientador}.
		III. \imprimirinstituicao.
		IV. Título. %
	\vspace{1.25cm}	%		
	\end{minipage}%
	}% 
	\end{minipage}}%
	
	\vspace*{0.5cm}
	
	\end{center}
\end{fichacatalografica}


%\begin{fichacatalografica}
%	\sffamily
%	\vspace*{\fill}					% Posição vertical
%	\begin{center}					% Minipage Centralizado
%	\fbox{\begin{minipage}[c][8cm]{13.5cm}		% Largura
%	\small
%	\imprimirautor
%	%Sobrenome, Nome do autor
%	
%	\hspace{0.5cm} \imprimirtitulo  / \imprimirautor. --
%	\imprimirlocal, \imprimirdata-
%	
%	\hspace{0.5cm} \pageref{LastPage} p. : il. (algumas color.) ; 30 cm.\\
%	
%	\hspace{0.5cm} \imprimirorientadorRotulo~\imprimirorientador\\
%	
%	\hspace{0.5cm}
%	\parbox[t]{\textwidth}{\imprimirtipotrabalho~--~\imprimirinstituicao,
%	\imprimirdata.}\\
%	
%	\hspace{0.5cm}
%		1. Palavra-chave1.
%		2. Palavra-chave2.
%		3. Palavra-chave3.
% 		4. Palavra-chave4.
%		5. Palavra-chave5.
%		I. Orientador.
%		II. Universidade xxx.
%		III. Faculdade de xxx.
%		IV. Título 			
%	\end{minipage}}
%	\end{center}
%\end{fichacatalografica}
% ---

	% Elemento Obrigatório (Verso da Folha)

% % ---
% % Inserir errata
% % ---
% \begin{errata}
% Elemento opcional. 

% Exemplo:

% \vspace{\onelineskip}

% SOBRENOME, Prenome do Autor. Título de obra: subtítulo (se houver). Ano de depósito. Tipo do trabalho (grau e curso) - Vinculação acadêmica, local de apresentação/defesa, data.

% \begin{table}[htb]
% \center
% \begin{tabular}{|p{2.4cm}|p{2cm}|p{3cm}|p{3cm}|}
%   \hline
%    \textbf{Folha} & \textbf{Linha}  & \textbf{Onde se lê}  & \textbf{Leia-se}  \\
%     \hline
%     1 & 10 & auto-conclavo & autoconclavo\\
%    \hline
% \end{tabular}
% \end{table}

% \end{errata}
% % ---				% Elemento Opcional

% ---
% Inserir folha de aprovação
% ---

% Isto é um exemplo de Folha de aprovação, elemento obrigatório da NBR
% 14724/2011 (seção 4.2.1.3). Você pode utilizar este modelo até a aprovação
% do trabalho. Após isso, substitua todo o conteúdo deste arquivo por uma
% imagem da página assinada pela banca com o comando abaixo:
%
% \includepdf{folhadeaprovacao_final.pdf}
%
\begin{folhadeaprovacao}



	\begin{center}
		{\fontseries{b}\selectfont\MakeTextUppercase{\normalsize\imprimirautor}}
	\end{center}
    \vfill
    
	\vfill
	\begin{center}
		{\fontseries{b}\selectfont\MakeTextUppercase{\imprimirtitulo}}
	\end{center}
	\vfill

    
\abntex@ifnotempty{\imprimirpreambulo}{%
	\hspace{.45\textwidth}
	{\begin{minipage}{.5\textwidth}
			\SingleSpacing
			\imprimirpreambulo\par
			\vspace*{4pt}
			{\imprimirorientadorRotulo~\imprimirorientador\par}
			\abntex@ifnotempty{\imprimircoorientador}{%
				{\imprimircoorientadorRotulo~\imprimircoorientador}%
			}%
	\end{minipage}}%
}%


\vfill
        
	 \begin{center}
        {\bfseries EXAMINING COMMITTEE: }
        \vspace*{1.75cm}
        
        Yuri Kaszubowski Lopes PhD. \par
        Universidade do Estado de Santa Catarina
        
        \vspace*{1.25cm}
        Rui Jorge Tramontim Junior PhD. \par
        Universidade do Estado de Santa Catarina
    \end{center}
        
    {Members:} 
        
    \begin{center}
        \vspace*{1.25cm}
        Fabiano Baldo PhD. \par
        Universidade do Estado de Santa Catarina
        
        \vspace*{1.25cm}
        Gilmário Barbosa do Santos PhD. \par
        Universidade do Estado de Santa Catarina
        
        \vspace*{1.25cm}
        Ricardo José Pfitscher PhD. \par
        Universidade Federal de Santa Catarina
    \end{center}
    
    
    \vspace*{\fill}  
    \begin{center}
    {\imprimirlocal, February 5, \imprimirdata}
	\end{center}
    \vspace*{0.25cm}  
\end{folhadeaprovacao}
% ---




%\textbf{	{Orientador: \vspace{-16pt} }
%	\assinatura{\textbf{Prof. \imprimirorientador , Dr.} \\ Univ. XXX} 
%	{Coorientador: \vspace{-16pt}}   
%	\assinatura{\textbf{Prof. \imprimircoorientador , Dr.} \\ Univ. XXX}
%	
%	{Membros: \vspace{-16pt} } 
%	
%	% --- Exemplo de assinaturas em sequência ---       
%	\setlength{\ABNTEXsignwidth}{8.5cm}
%	
%	\assinatura{\textbf{Prof. Professor, Dr.} \\ Univ. XXX}
%	\assinatura{\textbf{Prof. Professor, Dr.} \\ Univ. XXX}
%	\assinatura{\textbf{Prof. Professor, Dr.} \\ Univ. XXX}
%	
%	% --- Exemplo de assinaturas lado a lado ---
%	\setlength{\ABNTEXsignwidth}{7.5cm}
	%
	%    \noindent\hfill\assinatura*{\textbf{Prof. Professor, Dr.} \\ Univ. XXX}%
	%    \hfill%
	%    \assinatura*{\textbf{Prof. Professor, Dr.} \\ Univ. XXX}%
	%    \hfill
	%    
	%    \noindent\hfill\assinatura*{\textbf{Prof. Professor, Dr.} \\ Univ. XXX}%
	%    \hfill%
	%    \assinatura*{\textbf{Prof. Professor, Dr.} \\ Univ. XXX}%
	%    \hfill}		% Elemento Obrigatório
% ---
% Dedicatória
% ---
\begin{dedicatoria}
   \vspace*{\fill}
%   \begin{flushright}
%   \noindent
%	Este trabalho é dedicado às crianças adultas que,\\
%	quando pequenas, sonharam em se tornar cientistas. 
%   \end{flushright}

{%
	\noindent\hspace{.5\textwidth}
	{\begin{minipage}{.5\textwidth}
			\begin{flushleft}
				To my family, for putting up with me during this period. And to my friends, for putting up with me after it!
			\end{flushleft}
	\end{minipage}}%
\vspace*{3cm}
}%

\end{dedicatoria}
% ---
			% Elemento Opcional
% ---
% Agradecimentos
% ---
\begin{agradecimentos}
I would like to thank my advisor for accepting to guide my research work.

To all my professors at the Universidade do Estado de Santa Catarina – UDESC, for their excellence and technical expertise.

To my parents, who have always been by my side, supporting me throughout my entire journey. I am grateful to Júlia and my family for the unwavering support they have given me throughout my life.

A special thank you to my advisor for the encouragement and for dedicating his scarce time to my research project.


\end{agradecimentos}
% ---		% Elemento Opcional
% ---
% Epígrafe
% ---
\begin{epigrafe}
    \vspace*{\fill}
%	\begin{flushright}
%		\textit{``Eu não falhei, encontrei 10 mil soluções que não davam certo.'' (EDISON, [19--])}
%	\end{flushright}
{%
	\noindent\hspace{.5\textwidth}
	{\begin{minipage}{.5\textwidth}
		\begin{flushright}
			I am not procrastinating; I am minimizing the area under the effort curve.
		\end{flushright}
	\end{minipage}}%
	\vspace*{3cm}
}%
\end{epigrafe}
% ---				% Elemento Opcional
% ---
% RESUMOS
% ---

% resumo em português
\setlength{\absparsep}{18pt} % ajusta o espaçamento dos parágrafos do resumo
\begin{resumo}[RESUMO]
 \begin{otherlanguage*}{brazil}
Os problemas de \textit{matching} são críticos em diversas aplicações industriais, como alocação de tarefas, agendamento e distribuição de recursos. No entanto, as soluções de otimização existentes são frequentemente complexas, rígidas ou inacessíveis para profissionais sem experiência especializada. Esta dissertação propõe o design de um modelo flexível para definir e resolver problemas de \textit{matching} e agrupamento, permitindo que os usuários configurem dados relacionais e restrições de otimização sem a necessidade de um conhecimento profundo em técnicas de otimização.

Além do design de um modelo, este trabalho apresenta uma implementação inicial e introduz um algoritmo de matching inovador sob restrições de cotas. Esse algoritmo garante a equidade e a viabilidade em cenários de alocação de recursos onde as cotas desempenham um papel crucial. Ao preencher a lacuna entre os modelos teóricos de matching e suas aplicações práticas, esta pesquisa oferece uma abordagem estruturada, porém adaptável, para resolver problemas reais de \textit{matching}.

\textbf{Palavras-chave}: Matching. Agrupamento. Algoritmo. \textit{Framework}. Otimização.
\end{otherlanguage*}
\end{resumo}
				% Elemento Obrigatório
% ---
% Abstract
% ---

% resumo em inglês
\begin{resumo}[Abstract]
 \begin{otherlanguage*}{english}
   Matching problems are critical in various industrial applications, such as job assignment, scheduling, and resource allocation. However, existing optimization solutions are often complex, rigid, or inaccessible to practitioners without specialized expertise. This dissertation proposes the design of a flexible framework for defining and solving matching and grouping problems, enabling users to configure relational data and optimization constraints without requiring deep knowledge of optimization techniques.

    Beyond the framework design, this work presents an initial implementation and introduces a novel matching algorithm under quota constraints. This algorithm ensures fairness and feasibility in resource allocation scenarios where quotas play a crucial role. By bridging the gap between theoretical matching models and practical applications, this research provides a structured yet adaptable approach to solving real-world matching problems.
    
   \textbf{Keywords}: Matching. Grouping. Algorithm. Framework. Optimization.
 \end{otherlanguage*}
\end{resumo}
				% Elemento Obrigatório

% ---
% inserir lista de ilustrações
% ---
\pdfbookmark[0]{\listfigurename}{lof}
\listoffigures*
\cleardoublepage
% ---

% ---
% inserir lista de quadros
% ---
%\pdfbookmark[0]{\listofquadrosname}{loq}
%\listofquadros*
%\cleardoublepage
% ---


% ---
% inserir lista de tabelas
% ---
\pdfbookmark[0]{\listtablename}{lot}
\listoftables*
\cleardoublepage
% ---

% ---
% inserir lista de abreviaturas e siglas
% ---
\begin{siglas}
	\item[AI] Artificial Inteligence
	\item[GNN] Graph Neural Networks
	\item[MCMF] Minimum Cost Maximum Flow
    \item[NLP] Natural Language Processing
	\item[ORM] Object Relational Mapping
	\item[SCF] Set Coverage Framework
	\item[UDESC] Universidade do Estado de Santa Catarina
\end{siglas}
% ---

% ---
% inserir lista de símbolos
% ---


% \begin{simbolos}
%   \item[@] Arroba
%   \item[\%] Porcento
%   \item[$^\circ$C] Graus Celsius
%   \item[Ca] Cálcio
% \end{simbolos}

% ---
				% Elemento Opcional
% ---
% inserir o sumario
% ---
\pdfbookmark[0]{\contentsname}{toc}
\tableofcontents*
\cleardoublepage
% ---
				% Elemento Obrigatório

% -----------------------------------------------------------------
% ELEMENTOS TEXTUAIS
% -----------------------------------------------------------------
\textual

\pagestyle{PagNumReduzida}						% Comando para cabeçalho somente com numeração de página 10pt
\aliaspagestyle{chapter}{PagNumReduzida}		% Deixar numeração da primeira página com tamanho igual ao resto da numeração
% ref.: https://groups.google.com/g/abntex2/c/CP7g8ZMgi-c/m/KjfEnn5b9a4J


% ---- Mantenha está estrutura, assim você deixa o trabalho mais organizado -------

\chapter{Introduction} \label{chap:introduction}

Matching problems are fundamental in various real-world applications, including job and task assignment, scheduling, resource distribution, and recommendation systems. The efficiency of solving such problems directly impacts operational performance in multiple industries, from logistics and healthcare to finance and human resources. Despite their significance, many matching problems are computationally complex, often requiring advanced optimization techniques.

Existing approaches typically rely on specialized optimization frameworks or domain-specific heuristics\cite{ieee_survey}. While these methods can be effective, they present challenges regarding accessibility and adaptability. Many existing frameworks require extensive expertise in combinatorial optimization, making them difficult to adopt for professionals without specialized knowledge. Additionally, rigid modeling approaches may not generalize well to different industry needs, limiting their applicability.

Given these challenges, this dissertation proposes a flexible and accessible framework for solving various matching and grouping problems. The framework leverages relational data and user-defined optimization constraints, allowing for intuitive problem formulation while maintaining computational efficiency. By bridging the gap between theoretical optimization techniques and practical implementation, the proposed framework seeks to provide an effective tool for a broad audience of developers and industry professionals.

\section{Motivation}
Many matching problems belong to the class of NP-hard problems (see \cite{karp1972reducibility}), requiring sophisticated algorithms to obtain near-optimal solutions within feasible time constraints. However, the practical adoption of these techniques is hindered by the lack of accessible tools that can be adapted to different industrial contexts. Existing solutions often require dedicated teams of experts, increasing implementation costs and reducing scalability.

By developing a framework that simplifies the formulation and resolution of matching problems, this work aims to democratize access to advanced optimization techniques. The proposed approach enables professionals to efficiently configure and apply matching algorithms, reducing dependency on specialized expertise and facilitating adoption across diverse sectors.

\section{Research Objectives} 

The primary objective of this research is to develop a comprehensive framework capable of addressing a broad range of matching problems through configurable optimization strategies. Specifically, this work seeks to:

\begin{itemize}
    \item[\textbf{RO1.}]\label{ro1} Design an adaptable framework that supports various types of matching problems, including one-to-one, many-to-many, and constrained matchings.
    
    \item[\textbf{RO2.}]\label{ro2} Integrate different solution methodologies, including exact algorithms and heuristics, to balance computational efficiency and solution quality.
    
    \item[\textbf{RO3.}]\label{ro3} Provide an intuitive interface for defining problem constraints, allowing users to customize matching models without requiring deep expertise in optimization.

    \item[\textbf{RO4.}]\label{ro4} Introduce a novel approach for solving matching problems under fairness constraints, ensuring compliance with minimum quota requirements.
\end{itemize}

\section{Structure of the Dissertation}
The remainder of this dissertation is structured as follows.

Chapter~\ref{chap:background} presents fundamental concepts in matching problems, graph theory, optimization techniques, and relational diagram usage.
Chapter~\ref{chap:grouping_problems} reviews related work, highlighting existing algorithms and new tendencies.
Chapter~\ref{chap:design} describes the proposed framework’s architecture and design principles.
Chapter~\ref{chap:framework} details a preliminary implementation, illustrating the framework’s adaptability through real-world applications.
Chapter~\ref{chap:fair_matching} presents an algorithm for matching under minimum quotas.
Finally, Chapter~\ref{chap:conclusion} summarizes the contributions, discusses limitations, and outlines directions for future research.

%\chapter{Background and related work}
\chapter{Background} \label{chap:background}

    This chapter presents fundamental concepts in matching problems, graph theory, relational diagram usage, and optimization techniques. This concepts are important for the understanding of the further development of this dissertation.
    \section{Graphs}
    
        Graphs are a fundamental concept in computer science and mathematics, widely used to represent relationships between pairs of objects. A graph \( G \) is defined as a pair \( (V, E) \), where \( V \) is a set of vertices (or nodes) and \( E \) is a set of edges. Each edge in \( E \) connects a pair of vertices.

        \subsection{Vertices and Edges}
        
            The primary components of a graph are vertices and edges. A vertex (singular of vertices) represents a discrete entity in the graph, often denoted as \( v \in V \). An edge connects two vertices and represents the relationship between them. An edge is typically denoted as \( e = (u, v) \in E \), where \( u, v \in V \).
            %
            In Figure \ref{fig:base_graph}, the vertices are represented as circles and the edges are represented as lines.

            
            \begin{figure}[!ht]
                \centering
                \begin{tikzpicture}[main/.style = {draw, circle}] 
                    \node[main] (1) {$v_1$};
                    \node[main] (2) [above right of=1] {$v_2$};
                    \node[main] (3) [above right of=2] {$v_3$}; 
                    \node[main] (4) [below right of=2] {$v_4$};
                    \node[main] (5) [below of=4] {$v_5$};
                    \node[main] (6) [right of=4] {$v_6$};

                    \draw (2) -- (4);
                    \draw (1) -- (2);
                    \draw (4) -- (3);
                    \draw (4) -- (6);
                    \draw (1) to [out=90,in=180,looseness=1.5] (3);
                    \draw (2) to [out=270,in=180,looseness=1.5] (5);
                \end{tikzpicture}
                \caption{Graph $G = (V, E)$. The circles ($v_1, \ldots, v_6$) are the vertices $V$; the lines connecting the vertices are the edges $E$.} \label{fig:base_graph}
            \end{figure}
        
        \subsection{Subgraphs}
        
            A subgraph \( G' \) of a graph \( G \) is a graph formed from a subset of the vertices and edges of \( G \). Formally, \( G' = (V', E') \) is a subgraph of \( G = (V, E) \) if \( V' \subseteq V \) and \( E' \subseteq E \). Figure \ref{fig:subgraphs} provide examples of subgraphs of Graph G presented in Figure \ref{fig:base_graph}.

            \begin{figure}[!ht]
                \centering
                \begin{subfigure}{0.45\textwidth}
                    \centering
                    \begin{tikzpicture}[main/.style = {draw, circle}] 
                        \node[main] (1) {$v_1$};
                        \node[main] (2) [above right of=1] {$v_2$};
                        \node[main] (3) [above right of=2] {$v_3$}; 

                
                        \draw (1) -- (2);
                        \draw (1) to [out=90,in=180,looseness=1.5] (3);
                    \end{tikzpicture}
                    \caption{Subgraph $G'$} 
                    \label{fig:subgraph_1}
                \end{subfigure}
                \hfill
                \begin{subfigure}{0.45\textwidth}
                    \centering
                    \begin{tikzpicture}[main/.style = {draw, circle}] 
                        \node[main] (2) {$v_2$};
                        \node[main] (4) [below right of=2] {$v_4$};
                        \node[main] (5) [below of=4] {$v_5$};
                
                        \draw (2) -- (4);
                        \draw (2) to [out=270,in=180,looseness=1.5] (5);
                    \end{tikzpicture}
                    \caption{Subgraph $G''$} 
                    \label{fig:subgraph_2}
                \end{subfigure}
                \caption{Two possible subgraphs of Graph $G$ (See Figure \ref{fig:base_graph}).}
                \label{fig:subgraphs}
            \end{figure}
        
        \subsection{Subsets of Vertices}
        
            Subsets of vertices play a crucial role in various graph algorithms and properties. Given a graph \( G = (V, E) \), a subset of vertices \( S \subseteq V \) can be used to define induced subgraphs, vertex covers, and other structures. For example, the induced subgraph \( G[S] \) is formed by the vertices in \( S \) and all the edges between them in \( G \). Figure \ref{fig:subset_of_vertices} shows a subset of vertices of $G''$ (See Figure \ref{fig:subgraph_2}).

            \begin{figure}[!ht]
                \centering
                \begin{tikzpicture}[main/.style = {draw, circle}] 
                    \node[main] (2) {$v_2$};
                    \node[main] (4) [below right of=2] {$v_4$};
                    \node[main] (5) [below of=4] {$v_5$};
                \end{tikzpicture}
                \caption{Subset of vertices from Graph $G$ (Figure \ref{fig:base_graph}) which induces (generates) the Graph $G''$ (Figure \ref{fig:subgraph_2}).}
                \label{fig:subset_of_vertices}
            \end{figure}


        \subsection{Bipartite Graphs}

            A bipartite graph is a graph \( G = (V, E) \) whose vertices can be divided into two disjoint sets \( V_1 \) and \( V_2 \) such that every edge in \( E \) connects a vertex in \( V_1 \) to a vertex in \( V_2 \). Formally, a graph \( G \) is bipartite if \( V \) can be partitioned into two sets \( V_1 \) and \( V_2 \) such that \( V_1 \cap V_2 = \emptyset \) and every edge in \( E \) has one endpoint in \( V_1 \) and the other in \( V_2 \). Figure \ref{fig:G_bipartite_by_layer} depics examples of bipartite graphs and Figure \ref{fig:non_bipartite_graph} shows an example of a non-bipartite graph.
            
            Bipartite graphs have many applications in various fields, including computer science, biology, and social sciences. They are used to model relationships between two different classes of objects, such as students and courses, users and items, or proteins and genes. One common algorithmic problem on bipartite graphs is the maximum bipartite matching problem, which aims to find the largest subset of edges in the graph such that no two edges share a common vertex.
            
            % Bipartite graphs can be represented using an adjacency matrix \( A \), where \( A_{ij} = 1 \) if there is an edge between vertex \( i \) in \( V_1 \) and vertex \( j \) in \( V_2 \), and \( A_{ij} = 0 \) otherwise. Another common representation is the bipartite adjacency list, where each vertex in \( V_1 \) is associated with a list of vertices in \( V_2 \) that it is connected to.

        \begin{figure}[!ht]
            \centering
            \begin{subfigure}{0.3\textwidth}
                \centering
                \begin{tikzpicture}[main/.style = {draw, circle}] 
                    \node[main] (1) {$v_1$};
                    \node[main] (2) [below of=1] {$v_2$};
                    \node[main] (3) [below of=2] {$v_3$};
                    \node[main] (4) [below of=3] {$v_4$};
        
                    \node[main] (5) [right of=1, xshift=2cm]{$v_5$};
                    \node[main] (6) [below of=5] {$v_6$};
                    \node[main] (7) [below of=6] {$v_7$};
                    \node[main] (8) [below of=7] {$v_8$};
        
                    \draw (1) -- (8);
                    \draw (1) -- (6);
                    \draw (2) -- (5);
                    \draw (2) -- (7);
                    \draw (3) -- (5);
                    \draw (3) -- (6);
                    \draw (4) -- (7);
                    \draw (4) -- (8);
                \end{tikzpicture}
                \caption{}
                \label{fig:common_representation_of_bipartition}
            \end{subfigure}
            \hfill
            \begin{subfigure}{0.3\textwidth}
                \centering
                \begin{tikzpicture}[main/.style = {draw, circle}, rednode/.style={draw=red, thick, dashed, circle}] 
                    \node[main] (1) {$v_1$};
                    \node[rednode] (2) [above right of=1] {$v_2$};
                    \node[rednode] (3) [above right of=2] {$v_3$}; 
                    \node[main] (4) [below right of=2] {$v_4$};
                    \node[main] (5) [below of=4] {$v_5$};
                    \node[rednode] (6) [right of=4] {$v_6$};

        
                    \draw (2) -- (4);
                    \draw (1) -- (2);
                    \draw (4) -- (3);
                    \draw (4) -- (6);
                    \draw (1) to [out=90,in=180,looseness=1.5] (3);
                    \draw (2) to [out=270,in=180,looseness=1.5] (5);
                \end{tikzpicture}
                \caption{}
                \label{fig:bipartition_by_color}
            \end{subfigure}
            \hfill
            \begin{subfigure}{0.3\textwidth}
                \centering
                \begin{tikzpicture}[main/.style = {draw, circle}] 
                    \node[main] (1) {$v_1$};
                    \node[main] (4) [below of=1] {$v_4$};
                    \node[main] (5) [below of=4] {$v_5$};
                    

                    \node[main] (2) [right of=1, xshift=2cm]{$v_2$};
                    \node[main] (3) [below of=2] {$v_3$};
                    \node[main] (6) [below of=3] {$v_6$};
                    
        
                    \draw (2) -- (4);
                    \draw (1) -- (2);
                    \draw (4) -- (3);
                    \draw (1) -- (3);
                    \draw (2) -- (5);
                    \draw (4) -- (6);
                \end{tikzpicture}
                \caption{}
                \label{fig:G_bipartite_by_layer}
            \end{subfigure}
            \caption{Examples of bipartite graphs and ways to represent it. (a) Common representation of a bipartite graph (by layer). (b-c) the same Graph $G$ (See Figure \ref{fig:base_graph}), with different bipartite representations. (b) Representation by color, black / solid nodes represent one partition and red / dashed nodes represent the other partition. (c) Representation by layer.}
        \end{figure}

        It is important to note that not every graph is bipartite. A graph that contains an odd-length cycle cannot be bipartite, as it is impossible to partition the vertices into two sets without having two vertices of the same set being adjacent. For example, consider the following graph:
        
        \begin{figure}[!ht]
            \centering
            \begin{tikzpicture}[main/.style = {draw, circle}] 
                \node[main] (1) {$v_1$};
                \node[main] (2) [above right of=1] {$v_2$};
                \node[main] (3) [below right of=2] {$v_3$}; 
        
                \draw (1) -- (2);
                \draw (2) -- (3);
                \draw (3) -- (1);
            \end{tikzpicture}
            \caption{Example of a non-bipartite graph.} 
            \label{fig:non_bipartite_graph}
        \end{figure}
        
        This graph contains a cycle of length 3 (vertices \(v_1\), \(v_2\), and \(v_3\)), which is an odd-length cycle. Therefore, it is not possible to divide the vertices into two sets where all edges connect vertices from different sets, making this graph non-bipartite.
        


        \section{Matching}

            A matching in a graph \( G = (V, E) \) is a set of edges \( M \subseteq E \) such that no two edges in \( M \) share a common vertex. Matchings are used to model relationships in various applications, such as assigning tasks to workers or pairs in social networks. Figure \ref{fig:possible_matching} shows a possible matching o Graph G (see Figure \ref{fig:base_graph}).
            
            In a bipartite graph, a matching (also called bipartite matching) is a subset of edges \( M \subseteq E \) such that each vertex is incident to at most one edge in \( M \). Finding a maximum matching in a bipartite graph can be efficiently solved using algorithms like the Hopcroft-Karp algorithm \cite{west2001introduction, diestel2017graph}.


            \begin{figure}[!ht]
                \centering
                \begin{tikzpicture}[main/.style = {draw, circle}] 
                    \node[main] (1) {$v_1$};
                    \node[main] (4) [below of=1] {$v_4$};
                    \node[main] (5) [below of=4] {$v_5$};
                    

                    \node[main] (2) [right of=1, xshift=2cm]{$v_2$};
                    \node[main] (3) [below of=2] {$v_3$};
                    \node[main] (6) [below of=3] {$v_6$};
                    
        
                    \draw (1) -- (3);
                    \draw (2) -- (5);
                    \draw (4) -- (6);
                \end{tikzpicture}
                \caption{Possible matching of Graph $G$ (see Figure \ref{fig:base_graph}).}
                \label{fig:possible_matching}
            \end{figure}
            
        \subsection{Bipartite Matching}

            Bipartite Matching, also known as bipartite pairing is a fundamental problem in graph theory, with many practical applications, including resource allocation, market design, and job assignment.
            We have two distinct sets of elements, and the objective is to find corresponding pairs between them while also optimizing some criteria.
            
            In order to formalize the problem, let us consider two distinct sets, \(U\) and \(V\). The objective is to find a set of edges \(M\) that connect elements from \(U\) to \(V\) in such a manner that any pair of vertices can only be connected by at most one edge. This configuration is also known as a \textit{matching} \cite{hopcroft1973n}.
            
            It is important to note that in some cases, sets have different sizes, and thus it is not possible to match every vertex of both sets.
            
            \subsection{Fairness in Matching}
            
            Fairness in matching consists in take into account specific attributes of the set of vertices and their proportionality in the solution.
            
            Inspired by the work of \cite{sankar}, the notion of fairness adopted by this study is delegated to an external operator. This impartial entity does not directly engage with the sets being matched. Instead, it has the responsibility to judge what is fair or not in the matching context.
            
            In the search for pairing a set $U$ and a set $V$, the concept of giving priority to some specific subset of $U$ that contains specific traits is proposed. This different approach to the original problem is particularly relevant in scenarios where prioritizing different characteristics alongside minimal cost is desired.
            
            To operationalize this prioritization, this paper employs the concept of minimum quotas, which establish a minimum number of vacancies reserved for specific subsets. 
            These minimum quotas, act as instruments to promote equity in the matching process, ensuring those specific subsets receive adequate representation.
            
            The quotas refer to the predetermined allocation of a minimum number of matchings to each special subset, thereby establishing a fair distribution of pairings. This definition is crucial for ensuring representativity and preventing imbalances in the allocations.
            
            It is important to highlight that, in this study, the quotas are only applied to one set of the bipartite graph, meaning that only $U$ or $V$ is involved in the quota implementation.
            
            This definition of quotas aligns with the fairness proposed by \cite{sankar}, where the imposition of quotas becomes essential to maximize equity in the resources allocated. The next session approaches how this definition of fairness is represented in the mapping to the MCMF problem. 
            
            \subsection{Minimum Cost Maximum Flow}
            
            The minimum cost maximum flow problem refers to the search for the maximum amount of possible flow in a network, considering costs associated with the passage of flow through certain edges. In other words, the objective of such problems is to optimize the transportation of resources from one point to another while minimizing the costs involved in this process.
            
            This concept has broad applications, often used in transportation problems, network design, and linear programming, among others. Efficient resolution of Minimum-Cost Maximum Flow (MCMF) problems is crucial in various fields, contributing to the efficiency and economy of resources.
            
            \subsection{Solving Bipartite Matching with MCMF}\label{subsubsec:resolucao-fluxo-matching}
            
            To clarify the methodology of mapping a Bipartite Matching problem to a Minimum-Cost Maximum Flow (MCMF) problem, a method that is already well-known and documented \cite{ahuja1993network, edmonds1972theoretical, tarjan1997dynamic}, we present a detailed visual representation of this process in Figure~\ref{fig:mcmf}.
            
            % In this bipartite graph, the sets of nodes $U$ and $V$. 
            In this figure, the bipartite graph being matched is the set of nodes $U = \{u_1, \ldots, u_6\}$ (in red/dashed) and $V = \{v_1, \ldots, v_3\}$ (in blue/dotted). 
            %The edges connecting these sets have two parameters: the first number denotes the capacity, indicating the maximum number of flows that can be assigned to a given task, while the second number represents the cost associated with the edge. 
            The edges connecting the sets $U$ and $V$ have two parameters: the first number denotes the capacity, indicating the maximum number of flows that can be assigned to a given task, while the second number represents the cost associated with the edge. 
            
            The mapping strategy involves representing the bipartite graph as a flow network. 
            Two other nodes are added, a Source, connected to all nodes in Set $U$, and a Sink, connected to all nodes in Set $V$. 
            It is worth noting that the edges connected to the Source and Sink sets have zero cost and unit capacity to guarantee that the added connections do not affect the overall cost minimization.
            
            \begin{figure}[ht] \centering
                \resizebox{0.5\textwidth}{!}{%
 \begin{tikzpicture}[xscale=0.45,yscale=1.05]
   \tikzstyle{style} = [circle, draw, ultra thick, minimum size=45pt, inner sep=0pt, text centered]


      \draw
        (-10.909, 0.0) node[style, draw=black] (0){Source}
        (-2.909, 5.0) node[style, draw=red, dashed] (1){$u$$_1$}
        (-2.909, 3.0) node[style, draw=red, dashed] (2){$u$$_2$}
        (-2.909, 1.0) node[style, draw=red, dashed] (3){$u$$_3$}
        (-2.909, -1) node[style, draw=red, dashed] (4){$u$$_4$}
        (-2.909, -3.0) node[style, draw=red, dashed] (5){$u$$_5$}
        (-2.909, -5.0) node[style, draw=red, dashed] (6){$u$$_6$}
        (5.091, 4) node[style, draw=blue, dotted] (7){$v$$_1$}
        (5.091, -4.0) node[style, draw=blue, dotted] (8){$v$$_3$}
        (5.091, 0) node[style, draw=blue, dotted] (9){$v$$_2$}
        (13.091, 0.0) node[style, draw=black] (10){Sink};

  %   \tikzstyle{style} = [circle, draw, fill, minimum size=45pt, inner sep=0pt, text centered]


      % \draw
      %   (-10.909, 0.0) node[style, fill=cyan] (0){Source}
      %   (-2.909, 5.0) node[style, fill=brown] (1){$u$$_1$}
      %   (-2.909, 3.0) node[style, fill=brown] (2){$u$$_2$}
      %   (-2.909, 1.0) node[style, fill=brown] (3){$u$$_3$}
      %   (-2.909, -1) node[style, fill=brown] (4){$u$$_4$}
      %   (-2.909, -3.0) node[style, fill=brown] (5){$u$$_5$}
      %   (-2.909, -5.0) node[style, fill=brown] (6){$u$$_6$}
      %   (5.091, 4) node[style, fill=orange] (7){$v$$_1$}
      %   (5.091, -4.0) node[style, fill=orange] (8){$v$$_3$}
      %   (5.091, 0) node[style, fill=orange] (9){$v$$_2$}
      %   (13.091, 0.0) node[style, fill=cyan] (10){Sink};
    
  \begin{scope}[-]
    \draw[lightgray, text=black, font=\footnotesize] (0) to node[midway, sloped, ] {1 - 0} (1);
    \draw[lightgray, text=black, font=\footnotesize] (0) to node[midway, sloped, ] {1 - 0} (2);
    \draw[lightgray, text=black, font=\footnotesize] (0) to node[midway, sloped, ] {1 - 0} (3);
    \draw[lightgray, text=black, font=\footnotesize] (0) to node[midway, sloped, ] {1 - 0} (4);
    \draw[lightgray, text=black, font=\footnotesize] (0) to node[midway, sloped, ] {1 - 0} (5);
    \draw[lightgray, text=black, font=\footnotesize] (0) to node[midway, sloped, ] {1 - 0} (6);
    \draw[lightgray, text=black, font=\footnotesize] (1) to node[midway, sloped, ] {1 - 1} (7);
    \draw[lightgray, text=black, font=\footnotesize] (2) to node[midway, sloped, ] {1 - 7} (9);
    \draw[lightgray, text=black, font=\footnotesize] (3) to node[midway, sloped, ] {1 - 7} (9);
    \draw[lightgray, text=black, font=\footnotesize] (4) to node[midway, sloped, ] {1 - 1} (8);
    \draw[lightgray, text=black, font=\footnotesize] (5) to node[midway, sloped, ] {1 - 10} (9);
    \draw[lightgray, text=black, font=\footnotesize] (5) to node[midway, sloped, ] {1 - 10} (8);
    \draw[lightgray, text=black, font=\footnotesize] (6) to node[midway, sloped, ] {1 - 10} (7);
    \draw[lightgray, text=black, font=\footnotesize] (7) to node[midway, sloped, ] {1 - 0} (10);
    \draw[lightgray, text=black, font=\footnotesize] (8) to node[midway, sloped, ] {1 - 0} (10);
    \draw[lightgray, text=black, font=\footnotesize] (9) to node[midway, sloped, ] {1 - 0} (10);
  \end{scope}
\end{tikzpicture}
}%
                \caption{Mapping of bipartite matching to flow problem.}
                \label{fig:mcmf} 
            \end{figure}
            
            \begin{figure}[ht] \centering 
                
  % \begin{tikzpicture}[scale=0.7]
  %     \draw
  %       (-6.545, 0.0) node[draw=cyan, rounded corners] (0){Source}
  %       (-2.545, 1.2) node[draw=red, dashed, rounded corners] (1){$u$}
  %       (-2.545, 0.0) node[draw=red, dashed, rounded corners] (2){$u$}
  %       (-2.545, -1.2) node[draw=red, dashed, rounded corners] (3){$u$}
  %       (1.455, 3.0) node[draw=blue, dotted, rounded corners] (4){$v$}
  %       (1.455, 1.8) node[draw=blue, dotted, rounded corners] (5){$v$}
  %       (1.455, 0.6) node[draw=blue, dotted, rounded corners] (6){$v$}
  %       (1.455, -0.6) node[draw=blue, dotted, rounded corners] (7){$v$}
  %       (1.455, -1.8) node[draw=blue, dotted, rounded corners] (8){$v$}
  %       (1.455, -3.0) node[draw=blue, dotted, rounded corners] (9){$v$}
  %       (5.455, 0.0) node[draw=cyan, rounded corners] (10){Target};
  %     \begin{scope}[-]
  %       \draw[lightgray, text=black, font=\footnotesize]  (0) to node[]{0} (1);
  %       \draw[lightgray, text=black, font=\footnotesize]  (0) to  node[]{0} (2);
  %       \draw[lightgray, text=black, font=\footnotesize]  (0) to node[]{0}  (3);
  %       \draw[lightgray, text=black, font=\footnotesize]  (1) to  node[]{1} (4);
  %       \draw[lightgray, text=black, font=\footnotesize]  (2) to  node[]{2} (6);
  %       \draw[lightgray, text=black, font=\footnotesize]  (3) to  node[]{4} (9);
  %       \draw[lightgray, text=black, font=\footnotesize]  (4) to  node[]{0} (10);
  %       \draw[lightgray, text=black, font=\footnotesize]  (6) to  node[]{0} (10);
  %       \draw[lightgray, text=black, font=\footnotesize]  (9) to  node[]{0} (10);
  %     \end{scope}
  %   \end{tikzpicture}
    
\resizebox{0.5\textwidth}{!}{%
 \begin{tikzpicture}[xscale=0.45,yscale=1.05]
    \tikzstyle{style} = [circle, ultra thick, draw, draw, minimum size=45pt, inner sep=0pt, text centered]
    \tikzstyle{no_border} = [circle, draw, minimum size=45pt, inner sep=0pt, text centered, opacity=0.5]


      \draw
        (-10.909, 0.0) node[style, draw=black] (0){Source}
        (-2.909, 5.0) node[style, draw=red, dashed] (1){$u$$_1$}
        (-2.909, 3.0) node[style, draw=red, dashed] (2){$u$$_2$}
        (-2.909, 1.0) node[no_border, draw=red, dashed] (3){$u$$_3$}
        (-2.909, -1) node[style, draw=red, dashed] (4){$u$$_4$}
        (-2.909, -3.0) node[no_border, draw=red, dashed] (5){$u$$_5$}
        (-2.909, -5.0) node[no_border, draw=red, dashed] (6){$u$$_6$}
        (5.091, 4) node[style, draw=blue, dotted] (7){$v$$_1$}
        (5.091, -4.0) node[style, draw=blue, dotted] (8){$v$$_3$}
        (5.091, 0) node[style, draw=blue, dotted] (9){$v$$_2$}
        (13.091, 0.0) node[style, draw=black] (10){Sink};
    
      \begin{scope}[->, >=BigLatex]
        \draw[lightgray, text=black, font=\normalsize] (0) to node[] {1 ; 0} (1);
        \draw[lightgray, text=black, font=\normalsize] (0) to node[] {1 ; 0} (2);
        \draw[transparent] (0) to node[] {0 ; 0} (3);
        \draw[lightgray, text=black, font=\normalsize] (0) to node[] {1 ; 0} (4);
        \draw[transparent] (0) to node[] {0 ; 0} (5);
        \draw[transparent] (0) to node[] {0 ; 0} (6);
        \draw[lightgray, text=black, font=\normalsize] (1) to node[] {1 ; 1} (7);
        \draw[lightgray, text=black, font=\normalsize] (2) to node[] {1 ; 7} (9);
        \draw[transparent] (3) to node[] {0 ; 7} (9);
        \draw[lightgray, text=black, font=\normalsize] (4) to node[] {1 ; 1} (8);
        \draw[transparent] (5) to node[] {0 ; 10} (9);
        \draw[transparent] (5) to node[] {0 ; 10} (8);
        \draw[transparent] (6) to node[] {0 ; 10} (7);
        \draw[lightgray, text=black, font=\normalsize] (7) to node[] {1 ; 0} (10);
        \draw[lightgray, text=black, font=\normalsize] (8) to node[] {1 ; 0} (10);
        \draw[lightgray, text=black, font=\normalsize] (9) to node[] {1 ; 0} (10);
      \end{scope}
    \end{tikzpicture}
}%
                \caption{Solution of the mapping of bipartite matching to flow problem. The minimum cost of the matching is defined as the sum of the costs of the selected edges, which is 9 in this case.}
                \label{fig:solucao_mcmf}
            \end{figure}
            
            In this context, the vertices are represented by the sets $U$ and $V$, while the edges define the relationship between them, each associated with a specific cost. 
            Each edge connecting $U$ to $V$ carries a cost, which may represent, for instance in a job allocation problem, the hiring cost. 
            The algorithm seeks to optimize this matching by minimizing the total cost, which can reflect, in practical terms, the reduction of operational or resource allocation expenses.
            
            The MCMF algorithm is then applied to this network representation to determine the optimal allocation. 
            % We can observe the result of this process in Figure~\ref{fig:mcmf}, where the optimal flow (matching) is calculated and displayed in Figure~\ref{fig:solucao_mcmf}.
            For the problem instance presented in Figure~\ref{fig:mcmf}, the result of the application of the MCMF algorithm generates the optimal flow (matching) displayed in Figure~\ref{fig:solucao_mcmf}.
            In Figure~\ref{fig:solucao_mcmf}, the selected nodes are highlighted with full colors, forming the match: $\{(u_1, v_1), (u_2, v_2), (u_4, v_3)\}$.
            Note that multiple optimal flows may be possible.
            
            By exploring this approach, we can leverage theoretical advancements in MCMF algorithms, such as their use in parallelized processors \cite{akidau2013millwheel}, near-linear time algorithms \cite{9996881}, decentralized network computation \cite{alon2019decentralized}, and potentially quantum algorithms \cite{brandao2019quantum}.
             
        
        % \subsection{Vertex Cover}
        
        %     A vertex cover of a graph \( G = (V, E) \) is a subset of vertices \( C \subseteq V \) such that every edge in \( E \) has at least one endpoint in \( C \). Vertex covers are used in various optimization problems and are closely related to matching in graphs \cite{cormen2009introduction, papadimitriou1998combinatorial, vazirani2001approximation, garey1979computers}. 

        %     \begin{figure}[!ht]
        %         \centering
        %         \begin{subfigure}{0.45\textwidth}
        %             \centering
        %             \begin{tikzpicture}[main/.style = {draw, circle}, dotted2/.style = {draw, dotted, circle}] 
        %                 \node[main] (1) {$v_1$};
        %                 \node[main] (4) [below of=1] {$v_4$};
        %                 \node[main] (5) [below of=4] {$v_5$};
                        
    
        %                 \node[dotted2] (2) [right of=1, xshift=2cm]{$v_2$};
        %                 \node[dotted2] (3) [below of=2] {$v_3$};
        %                 \node[dotted2] (6) [below of=3] {$v_6$};
                        
            
        %                 \draw (2) -- (4);
        %                 \draw (1) -- (2);
        %                 \draw (4) -- (3);
        %                 \draw (1) -- (3);
        %                 \draw (2) -- (5);
        %                 \draw (4) -- (6);
        %             \end{tikzpicture}
        %             \caption{Vertex Cover $C'$} 
        %             \label{fig:vertex_cover_1}
        %         \end{subfigure}
        %         \hfill
        %         \begin{subfigure}{0.45\textwidth}
        %             \centering
        %             \begin{tikzpicture}[main/.style = {draw, circle}, dotted2/.style = {draw, dotted, circle}] 
        %                 \node[main] (1) {$v_1$};
        %                 \node[main] (4) [below of=1] {$v_4$};
        %                 \node[dotted2] (5) [below of=4] {$v_5$};
                        
    
        %                 \node[main] (2) [right of=1, xshift=2cm]{$v_2$};
        %                 \node[dotted2] (3) [below of=2] {$v_3$};
        %                 \node[dotted2] (6) [below of=3] {$v_6$};
                        
            
        %                 \draw (2) -- (4);
        %                 \draw (1) -- (2);
        %                 \draw (4) -- (3);
        %                 \draw (1) -- (3);
        %                 \draw (2) -- (5);
        %                 \draw (4) -- (6);
        %             \end{tikzpicture}
        %             \caption{Vertex Cover $C''$}
        %             \label{fig:vertex_cover_2}
        %         \end{subfigure}
        %         \caption{Two possible vertex subsets of Graph $G$ that are vertex covers.}
        %     \end{figure}


        % \subsection{Edge Cover}

        %     An edge cover of a graph $G = (V, E)$ is a subset of edges $C \subseteq E$ such that every vertex in $V$ is incident to at least one edge in $C$. Edge covers are used in various optimization problems and are closely related to matchings in graphs \cite{cormen2009introduction, papadimitriou1998combinatorial, vazirani2001approximation, garey1979computers}.

        %     \begin{figure}[!ht]
        %         \centering
        %         \begin{subfigure}{0.45\textwidth}
        %             \centering
        %             \begin{tikzpicture}[main/.style = {draw, circle}, dotted2/.style = {draw, dotted, circle}] 
        %                 \node[main] (1) {$v_1$};
        %                 \node[main] (4) [below of=1] {$v_4$};
        %                 \node[main] (5) [below of=4] {$v_5$};
                        
    
        %                 \node[main] (2) [right of=1, xshift=2cm]{$v_2$};
        %                 \node[main] (3) [below of=2] {$v_3$};
        %                 \node[main] (6) [below of=3] {$v_6$};
                        
            
        %                 \draw[dotted] (2) -- (4);
        %                 \draw[dotted] (1) -- (2);
        %                 \draw[dotted] (4) -- (3);
        %                 \draw (1) -- (3);
        %                 \draw (2) -- (5);
        %                 \draw (4) -- (6);
        %             \end{tikzpicture}
        %             \caption{An Edge Cover} 
        %             \label{fig:edge_cover_1}
        %         \end{subfigure}
        %         \hfill
        %         \begin{subfigure}{0.45\textwidth}
        %             \centering
        %             \begin{tikzpicture}[main/.style = {draw, circle}, dotted2/.style = {draw, dotted, circle}] 
        %                 \node[main] (1) {$v_1$};
        %                 \node[main] (4) [below of=1] {$v_4$};
        %                 \node[main] (5) [below of=4] {$v_5$};
                        
    
        %                 \node[main] (2) [right of=1, xshift=2cm]{$v_2$};
        %                 \node[main] (3) [below of=2] {$v_3$};
        %                 \node[main] (6) [below of=3] {$v_6$};
                        
            
        %                 \draw (2) -- (4);
        %                 \draw[dotted] (1) -- (2);
        %                 \draw (4) -- (3);
        %                 \draw (1) -- (3);
        %                 \draw (2) -- (5);
        %                 \draw (4) -- (6);
        %             \end{tikzpicture}
        %             \caption{A Non-Optimal Edge Cover}
        %             \label{fig:edge_cover_2}
        %         \end{subfigure}
        %         \caption{Two possible edge subsets of Graph $G$ that are edge covers.}
        %     \end{figure}
        
    \section{Set Cover} \label{WSC}
    
        Set cover is a classic problem in computer science and mathematics. Given a set $X$ of elements and a collection $S$ of subsets of $X$, the set cover problem is to find the smallest subcollection of $S$ whose union is equal to $X$. It has applications in scheduling, DNA sequencing, and data compression, and is one of Karp's 21 NP-complete problems  \cite{karp1972reducibility}. Figure \ref{fig:set_cover} shows two examples of Set Cover.

    \begin{figure}[!ht]
        \centering
        \begin{subfigure}{0.45\textwidth}
            \centering
            \resizebox{\textwidth}{!}{%
            \begin{circuitikz}
                \tikzstyle{every node}=[font=\LARGE]
                
                \draw [, line width=1pt ] (9,13.25) circle (0.75cm) node {$x_1$};
                \draw [, line width=1pt ] (20.25,15.25) circle (0.75cm) node {$x_2$};
                \draw [, line width=1pt ] (11.5,5.75) circle (0.75cm) node {$x_3$};
                \draw [, line width=1pt ] (15.25,12.25) circle (0.75cm) node {$x_4$};
                \draw [, line width=1pt ] (21.25,10.25) circle (0.75cm) node {$x_5$};
                \draw [, line width=1pt ] (16.5,7) circle (0.75cm) node {$x_6$};
                \draw [, line width=1pt ] (8,9.75) circle (0.75cm) node {$x_7$};
                
                 \draw [ color={rgb,255:red,0; green,85; blue,255}, line width=1pt, dashed, ultra thick] (7.75,14.25) --(7.5,12.25);
                \draw [ color={rgb,255:red,0; green,85; blue,255}, line width=1pt, dashed, ultra thick] (17,5) .. controls (20,7.75) and (20.25,7.75) .. (23.25,10.5);
                \draw [ color={rgb,255:red,0; green,85; blue,255}, line width=1pt, dashed, ultra thick] (23.25,10.5) .. controls (22,11.25) and (22.25,11.25) .. (21.25,12);
                \draw [ color={rgb,255:red,0; green,85; blue,255}, line width=1pt, dashed, ultra thick] (21.25,12) .. controls (18,10.25) and (18.25,10.25) .. (15,8.75);
                \draw [ color={rgb,255:red,0; green,85; blue,255}, line width=1pt, dashed, ultra thick] (15,8.75) .. controls (12,11.75) and (12.25,11.75) .. (9.25,15);
                \draw [ color={rgb,255:red,0; green,85; blue,255}, line width=1pt, dashed, ultra thick] (9.25,15) .. controls (8.25,14.5) and (8.5,14.5) .. (7.75,14.25);
                \draw [ color={rgb,255:red,0; green,85; blue,255}, line width=1pt, dashed, ultra thick] (7.5,12.25) -- (10.25,8.25);
                \draw [ color={rgb,255:red,0; green,85; blue,255}, line width=1pt, dashed, ultra thick] (10.25,8.25) -- (10.25,3);
                \draw [ color={rgb,255:red,0; green,85; blue,255}, line width=1pt, dashed, ultra thick] (10.25,3) -- (17,5);
                
                \draw [ color={rgb,255:red,255; green,0; blue,0}, line width=1pt, loosely dotted, ultra thick] (12.25,4.75) .. controls (14.5,8.25) and (14.75,8.25) .. (17,12);
                \draw [ color={rgb,255:red,255; green,0; blue,0}, line width=1pt, loosely dotted, ultra thick] (17,12) .. controls (19.75,13.25) and (20,13.25) .. (22.75,14.5);
                \draw [ color={rgb,255:red,255; green,0; blue,0}, line width=1pt, loosely dotted, ultra thick] (22.75,14.5) .. controls (21.5,15.75) and (21.75,15.75) .. (20.75,17);
                \draw [ color={rgb,255:red,255; green,0; blue,0}, line width=1pt, loosely dotted, ultra thick] (20.75,17) .. controls (16.75,14.5) and (17,14.5) .. (13.25,12.25);
                \draw [ color={rgb,255:red,255; green,0; blue,0}, line width=1pt, loosely dotted, ultra thick] (13.25,12.25) .. controls (11.25,8.25) and (11.5,8.25) .. (9.5,4.5);
                \draw [ color={rgb,255:red,255; green,0; blue,0}, line width=1pt, loosely dotted, ultra thick] (9.5,4.5) .. controls (10.5,4.25) and (10.75,4.25) .. (12,4);
                \draw [ color={rgb,255:red,255; green,0; blue,0}, line width=1pt, loosely dotted, ultra thick] (12,4) .. controls (12,4.25) and (12.25,4.25) .. (12.25,4.75);
                
                \draw [ color={rgb,255:red,0; green,143; blue,2}, line width=1pt, loosely dashdotted, ultra thick] (6.75,7.75) .. controls (6,9.25) and (6.25,9.25) .. (5.75,10.75);
                \draw [ color={rgb,255:red,0; green,143; blue,2}, line width=1pt, loosely dashdotted, ultra thick] (5.75,10.75) .. controls (13.75,14.5) and (14,14.5) .. (22.25,18.5);
                \draw [ color={rgb,255:red,0; green,143; blue,2}, line width=1pt, loosely dashdotted, ultra thick] (22.25,18.5) .. controls (23.25,13.5) and (23.5,13.5) .. (24.5,8.5);
                \draw [ color={rgb,255:red,0; green,143; blue,2}, line width=1pt, loosely dashdotted, ultra thick] (24.5,8.5) .. controls (23.25,7.75) and (23.5,7.75) .. (22.5,7);
                \draw [ color={rgb,255:red,0; green,143; blue,2}, line width=1pt, loosely dashdotted, ultra thick] (6.75,7.75) .. controls (12.75,9.75) and (13,9.75) .. (19,12);
                \draw [ color={rgb,255:red,0; green,143; blue,2}, line width=1pt, loosely dashdotted, ultra thick] (19,12) .. controls (20.5,9.5) and (20.75,9.5) .. (22.5,7);

            \end{circuitikz}
            }%
            \caption{A Non-Optimal Set Cover. Subsets selected are $s_1$, $s_2$, and $s_3$.} 
            \label{fig:set_cover_1}
        \end{subfigure}
        \hfill
        \begin{subfigure}{0.45\textwidth}
            \centering
            \resizebox{\textwidth}{!}{%
            \begin{circuitikz}
                \tikzstyle{every node}=[font=\LARGE]
                
                \draw [, line width=1pt ] (9,13.25) circle (0.75cm) node {$x_1$};
                \draw [, line width=1pt ] (20.25,15.25) circle (0.75cm) node {$x_2$};
                \draw [, line width=1pt ] (11.5,5.75) circle (0.75cm) node {$x_3$};
                \draw [, line width=1pt ] (15.25,12.25) circle (0.75cm) node {$x_4$};
                \draw [, line width=1pt ] (21.25,10.25) circle (0.75cm) node {$x_5$};
                \draw [, line width=1pt ] (16.5,7) circle (0.75cm) node {$x_6$};
                \draw [, line width=1pt ] (8,9.75) circle (0.75cm) node {$x_7$};

                 \draw [ color={rgb,255:red,0; green,85; blue,255}, line width=1pt, dashed, ultra thick] (7.75,14.25) --(7.5,12.25);
                \draw [ color={rgb,255:red,0; green,85; blue,255}, line width=1pt, dashed, ultra thick] (17,5) .. controls (20,7.75) and (20.25,7.75) .. (23.25,10.5);
                \draw [ color={rgb,255:red,0; green,85; blue,255}, line width=1pt, dashed, ultra thick] (23.25,10.5) .. controls (22,11.25) and (22.25,11.25) .. (21.25,12);
                \draw [ color={rgb,255:red,0; green,85; blue,255}, line width=1pt, dashed, ultra thick] (21.25,12) .. controls (18,10.25) and (18.25,10.25) .. (15,8.75);
                \draw [ color={rgb,255:red,0; green,85; blue,255}, line width=1pt, dashed, ultra thick] (15,8.75) .. controls (12,11.75) and (12.25,11.75) .. (9.25,15);
                \draw [ color={rgb,255:red,0; green,85; blue,255}, line width=1pt, dashed, ultra thick] (9.25,15) .. controls (8.25,14.5) and (8.5,14.5) .. (7.75,14.25);
                \draw [ color={rgb,255:red,0; green,85; blue,255}, line width=1pt, dashed, ultra thick] (7.5,12.25) -- (10.25,8.25);
                \draw [ color={rgb,255:red,0; green,85; blue,255}, line width=1pt, dashed, ultra thick] (10.25,8.25) -- (10.25,3);
                \draw [ color={rgb,255:red,0; green,85; blue,255}, line width=1pt, dashed, ultra thick] (10.25,3) -- (17,5);
                
                \draw [ color={rgb,255:red,0; green,143; blue,2}, line width=1pt, loosely dashdotted, ultra thick] (6.75,7.75) .. controls (6,9.25) and (6.25,9.25) .. (5.75,10.75);
                \draw [ color={rgb,255:red,0; green,143; blue,2}, line width=1pt, loosely dashdotted, ultra thick] (5.75,10.75) .. controls (13.75,14.5) and (14,14.5) .. (22.25,18.5);
                \draw [ color={rgb,255:red,0; green,143; blue,2}, line width=1pt, loosely dashdotted, ultra thick] (22.25,18.5) .. controls (23.25,13.5) and (23.5,13.5) .. (24.5,8.5);
                \draw [ color={rgb,255:red,0; green,143; blue,2}, line width=1pt, loosely dashdotted, ultra thick] (24.5,8.5) .. controls (23.25,7.75) and (23.5,7.75) .. (22.5,7);
                \draw [ color={rgb,255:red,0; green,143; blue,2}, line width=1pt, loosely dashdotted, ultra thick] (6.75,7.75) .. controls (12.75,9.75) and (13,9.75) .. (19,12);
                \draw [ color={rgb,255:red,0; green,143; blue,2}, line width=1pt, loosely dashdotted, ultra thick] (19,12) .. controls (20.5,9.5) and (20.75,9.5) .. (22.5,7);


            \end{circuitikz}
            }%
            \caption{A Optimal Set Cover. Subsets selected are $s_1$ and $s_2$.} 
            \label{fig:set_cover_2}
        \end{subfigure}
        \caption{Example of Set Cover. Where $S=\{s_1, s_2, s_3\}$, $s_1=\{x_1, x_3, x_6, x_5\}$, $s_2=\{x_7, x_4, x_2, x_5\}$ and $s_3=\{x_3, x_4, x_2\}$.} 
        \label{fig:set_cover}
    \end{figure}


    \subsection{Exact Set Cover}
    The exact set cover problem is a variant where each element in $X$ must be covered exactly once by the selected sets. This variant is particularly relevant in scenarios where overlapping cover is not allowed or desired, such as certain types of resource allocation and exact sequence alignment in computational biology. Finding an exact set cover is also an NP-complete problem, requiring sophisticated algorithms or approximation techniques for large instances.
    
    \subsection{Weighted Set Cover}
    In the weighted set cover problem, each set in the collection $S$ is assigned a positive weight (representing its cost), and the objective is to find a set cover that minimizes the total weight. The unweighted set cover problem can be seen as a special case where all sets in $S$ have a uniform weight of 1.
    
    \begin{itemize}
        \item \textbf{Definition:} formally, let $X$ be a set of $n$ elements and $S = \{S_1, S_2, \ldots, S_m\}$ be a collection of $m$ subsets of $X$. Each set $S_i$ is associated with a weight $w_i$. The goal is to find a sub-collection $C \subseteq S$ such that the union of the sets in $C$ equals $X$ and the sum of the weights of the sets in $C$ is minimized.
        \item \textbf{Applications:} the weighted set cover problem has practical applications in various domains. For example, in network design, where different network configurations have different costs, the goal is to cover all network nodes with minimum cost. In sensor placement, each sensor has a different deployment cost, and the objective is to cover a region with the least expense. Another example is the timing table problem, which can be represented as a weighted set cover problem. In this context, various tasks need to be scheduled within a timing table, where each task has a different cost associated with its execution time, and the goal is to cover all required tasks with the minimum total cost.
        \item \textbf{Complexity:} similar to the unweighted version, the weighted set cover problem is NP-hard. This means that there is no known polynomial-time algorithm to solve the problem exactly for large instances. However, there are approximation algorithms that can provide near-optimal solutions in reasonable time.
    \end{itemize}
    
    The set cover problem, whether weighted or unweighted, is a fundamental problem in theoretical computer science and has motivated the development of various algorithmic techniques, including greedy algorithms, linear programming relaxations, and randomized algorithms. Understanding and solving the set cover problem is crucial for tackling more complex optimization problems in various fields.
    


    \section{ORM (Object-Relational Mapping)} \label{sec:orm}

ORM (Object-Relational Mapping) is a programming technique that allows developers to map objects from object-oriented programming languages to relational database tables \cite{ambler2002object}. This technique facilitates the integration between the object-oriented and relational paradigms by automating the conversion of data between incompatible systems. In ORM, each class in an object-oriented system becomes a table/entity in a relational database, and the attributes of the class correspond to the columns of the table.

Relationships between classes are also represented in ORM \cite{fowler2003patterns}. For example, a one-to-many relationship between two classes would be represented by a foreign key in the database table of the "many" class, pointing to the primary key of the "one" class. This mapping ensures that the relational database structure reflects the object-oriented design of the application. Figure \ref{fig:uml_teacher_subject_implicit_example} demonstrates these relations, with examples of one-to-one, many-to-one and many-to-many relations. In order to implement a many-to-many relation, a relation class is created, as shown in Figures \ref{fig:uml_teacher_subject_implicit_example} and \ref{fig:uml_teacher_new_class}.


\begin{figure}[!ht]
    \centering
    \begin{subfigure}[b]{0.47\textwidth}
    \centering
    \resizebox{1\textwidth}{!}{%
    \begin{tikzpicture}[scale=0.5]
        \begin{class}{Person}{0, 8}
            \attribute {id : String}
            
        \end{class}
        \begin{class}{Teacher}{0, 0}
            \attribute {id : String}
            
        \end{class}
        \begin{class}{Subject}{15, 0}
            \attribute {id : String}
            
        \end{class}
        \begin{class}{Course}{15, 8}
            \attribute {id : String}
            
        \end{class}
        \association {Teacher}{}{0:N}{Subject}{0:N}{}
        \association {Teacher}{}{0:1}{Person}{0:1}{}
        \association {Course}{}{0:1}{Subject}{0:N}{}
    \end{tikzpicture}
    }
    \caption{Representation of simple relations between entities of a school.}
    \label{fig:uml_teacher_subject_implicit_example}
    \end{subfigure}
    \hfill
    \begin{subfigure}[b]{0.47\textwidth}
    \centering
    \resizebox{1\textwidth}{!}{%
    \begin{tikzpicture}[scale=0.5]
        \begin{class}{Person}{0, 8}
            \attribute {id : String}
            
        \end{class}
        \begin{class}{Teacher}{0, 0}
            \attribute {id : String}
            
        \end{class}
        \begin{class}{Subject}{15, 0}
            \attribute {id : String}
            
        \end{class}
        \begin{class}{Course}{15, 8}
            \attribute {id : String}
            
        \end{class}
        \begin{class}{Assignment}{7.5, -8}
            \attribute {id : String}
            \attribute {professorId : String}
            \attribute {classId : String}
        \end{class}
        \association {Teacher}{}{0:1}{Assignment}{0:1}{}
        \association {Assignment}{}{0:1}{Subject}{0:1}{}
        \association {Teacher}{}{0:1}{Person}{0:1}{}
        \association {Course}{}{0:1}{Subject}{0:N}{}
    \end{tikzpicture}
    }
    \caption{New Class \textit{Assignment} to represent the many-to-many relation of Figure \ref{fig:uml_teacher_subject_implicit_example}.}
    \label{fig:uml_teacher_new_class}
    \end{subfigure}
    \caption{Examples of implicit relations and their representation in UML.}
\end{figure}

ORM is commonly used in industry to simplify the development process and reduce the amount of boilerplate code needed for database interactions. By abstracting away the complexities of SQL queries and database schema management, ORM frameworks enable developers to focus more on the business logic of the application and less on the intricacies of database operations \cite{larman2004applying}.


There are various ORM frameworks and tools available for different programming languages and databases. Some popular ORM frameworks include Hibernate for Java, Entity Framework for .NET, and Django for Python \cite{bernstein2009object}. These frameworks provide robust mechanisms for data manipulation, transaction management, and query generation, thus enhancing productivity and ensuring consistency in database interactions.

\subsection{Relation Rules} \label{sec:relation_rules}

In the context of object-relational mapping (ORM), the relation rules (relation geometry) describes the structure and constraints that govern relationships between entities in a database schema. It encompasses the rules defining how entities interact, how many objects can relate to one another, and how these relationships are navigated. Understanding this geometry is essential for designing consistent and efficient database systems that accurately reflect the domain's requirements.

\subsubsection{One-to-One (1:1) Relationships}
 A \textbf{one-to-one relationship} represents a scenario where each instance of an entity is related to exactly one instance of another entity and vice versa. This type of relationship is typically used when an entity's attributes are logically split into separate tables for modularity or privacy purposes.

For example, in a system where each individual has one unique passport, the relationship between the entities \texttt{Person} and \texttt{Passport} can be modeled as one-to-one. In ORM terms, the \texttt{Person} entity includes a foreign key that links to the \texttt{Passport} entity, and the reverse link is also maintained.

The cardinality constraints for this relationship are:
\begin{itemize}
    \item \(0..1\): An optional relationship where an entity might not be linked (e.g., a person without a passport).
    \item \(1..1\): A mandatory one-to-one relationship where both entities must exist.
\end{itemize}

\subsubsection{One-to-Many (1:N) Relationship} A \textbf{one-to-many relationship} occurs when a single instance of an entity relates to multiple instances of another entity. This relationship is commonly used to represent hierarchical or ownership relationships.

For example, consider the relationship between \texttt{Department} and \texttt{Employee}. Each department can have multiple employees, but each employee belongs to a single department. In ORM, the \texttt{Department} entity has a collection of \texttt{Employee} entities, while each \texttt{Employee} contains a foreign key referencing the \texttt{Department}.

The cardinality constraints are the following.
\begin{itemize}
    \item \(1..N\): A department must have at least one employee.
    \item \(0..N\): A department can exist without any employees.
\end{itemize}

\subsubsection{Many-to-Many (N:N) Relationship} A \textbf{many-to-many relationship} arises when multiple instances of an entity are related to multiple instances of another entity. This relationship is often implemented using an intermediate or junction table that connects the two entities, storing the relationships explicitly.

For example, in an academic setting, a \texttt{Student} can enroll in multiple \texttt{Courses}, and each \texttt{Course} can have multiple \texttt{Students}. The \texttt{Enrollment} table acts as the junction table, containing foreign keys referencing both \texttt{Student} and \texttt{Course} entities.

The cardinality for such relationships is:
\begin{itemize}
    \item \(N:N\): Each student can be linked to multiple courses, and each course can be linked to multiple students.
\end{itemize}

\subsubsection{Relation Rules and Constraints}

To maintain logical consistency and integrity, certain rules and constraints govern the relationships between entities:
\begin{enumerate}
    \item \textbf{Cardinality Rules:} Define the minimum and maximum number of objects allowed on each side of the relationship (e.g., \(0..1\), \(1..N\), \(N..N\)).
    \item \textbf{Ownership Rules:} Specify which entity is the owner or controller of the relationship, determining how changes propagate between related entities.
    \item \textbf{Cascade Rules:} Define the behavior of related objects when an entity is modified or deleted. For example, cascading deletes ensure that when a \texttt{Department} is deleted, all associated \texttt{Employee} records are also removed.
\end{enumerate}

These constraints are critical in ensuring that relationships remain coherent and reflect the domain requirements. Using ORM frameworks, developers can abstract these rules into high-level representations, reducing the complexity of database interactions \cite{larman2004applying}.


\subsection{Example of ORM Usage}
Script \ref{script:orm} is an example of how ORM can be used in Python with Django.


\begin{lstlisting}[language=Python, caption={Example of ORM usage in Python with Django.}, label={script:orm}]
class Person(models.Model):
    id = models.CharField(max_length=255, primary_key=True)

class Teacher(models.Model):
    id = models.CharField(max_length=255, primary_key=True)
    person = models.OneToOneField(Person, related_name='teacher')

class Course(models.Model):
    id = models.CharField(max_length=255, primary_key=True)

class Subject(models.Model):
    id = models.CharField(max_length=255, primary_key=True)
    course = models.ForeignKey(Course, related_name='subjects')

class Assignment(models.Model):
    id = models.CharField(max_length=255, primary_key=True)
    professor = models.ForeignKey(Teacher)
    subject = models.ForeignKey(Subject)

# Adding the Many-to-Many relationship through Assignment
Subject.teachers = models.ManyToManyField(Teacher, through=Assignment, related_name='subjects')

\end{lstlisting}

 Both Figure \ref{fig:uml_teacher_new_class} and Script \ref{script:orm}
 show a new object called \textit{Assignment} to manage the relation between \textit{Teachers} and \textit{Subjects}.


Overall, ORM is a powerful tool in modern software development, offering significant benefits in terms of productivity and code maintainability. %However, developers must be aware of its limitations and carefully consider when and how to use it effectively.

% \subsection{Advantages of ORM}
%     ORM offers several advantages over traditional data access techniques:
%     \begin{itemize}
%         \item \textbf{Abstraction:} ORM provides a higher level of abstraction over database operations, allowing developers to work with objects instead of SQL queries.
%         \item \textbf{Productivity:} by automating repetitive tasks such as CRUD (Create, Read, Update, Delete) operations, ORM frameworks reduce development time and effort.
%         \item \textbf{Maintainability:} ORM promotes cleaner and more maintainable code by separating business logic from database access logic.
%         \item \textbf{Portability:} ORM frameworks often support multiple database systems, making it easier to switch databases without significant code changes.
%     \end{itemize}

% \subsection{Challenges of ORM}
%     Despite its advantages, ORM also presents certain challenges:
%     \begin{itemize}
%         \item \textbf{Performance:} ORM can introduce performance overhead due to the abstraction layer, especially for complex queries.
%         \item \textbf{Complexity:} while ORM simplifies many aspects of database interactions, it can add complexity in understanding the framework itself and troubleshooting issues.
%         \item \textbf{Flexibility:} ORM frameworks may impose restrictions on database design and operations, limiting the use of advanced database features.
%     \end{itemize}

%     Despite its advantages, ORM also presents certain challenges:
%     \begin{itemize}
%         \item \textbf{Performance:} ORM can introduce performance overhead due to the abstraction layer, especially for complex queries.
%         \item \textbf{Complexity:} While ORM simplifies many aspects of database interactions, it can add complexity in understanding the framework itself and troubleshooting issues.
%         \item \textbf{Flexibility:} ORM frameworks may impose restrictions on database design and operations, limiting the use of advanced database features.
%     \end{itemize}

    \section{Genetic Algorithms} \label{sec:ga}
    
    Genetic algorithms (GAs) are optimization methods inspired by the principles of natural selection and genetics. These algorithms aim to find solutions to problems by evolving a population of candidate solutions over time. In this section, we will explore the key components of genetic algorithms and the processes involved in evolving solutions.
    
    \subsection{Genes, Chromosomes, and Population}
    
    In genetic algorithms, the concept of a gene, chromosome (or genotype), and population is crucial to understanding how solutions are encoded and evolved. 
    
    A gene represents the smallest unit of information in a genetic algorithm, similar to how a gene in biology contains hereditary information. Several genes together form a chromosome, which is a complete candidate solution to the problem being addressed. Finally, a population is a set of chromosomes representing a diverse set of possible solutions.
    
    Figure \ref{fig:ga} visually demonstrates these relationships. Each black rectangle in the diagram corresponds to a gene, and groups of genes form chromosomes. The entire collection of chromosomes is known as the population.
    
    \begin{figure}[!ht]
    \centering
    \begin{minipage}[c]{0.4\textwidth} % [c] para centralizar verticalmente
        \centering
        \resizebox{1\textwidth}{!}{%
        \begin{circuitikz}
            \tikzstyle{every node}=[font=\large]
            
            % Retângulos com traço sólido (não coloridos) escalados
            \draw  (5.8,11.3) rectangle (6.8,10.3) node[pos=.5] {$G_{1}$};
            \draw  (6.8,10.3) rectangle (7.8,11.3) node[pos=.5] {$G_{2}$};
            \draw  (7.8,11.3) rectangle (8.8,10.3) node[pos=.5] {$G_{3}$};
            \draw  (8.8,10.3) rectangle (9.8,11.3) node[pos=.5] {$G_{4}$};
            \draw  (9.8,11.3) rectangle (10.8,10.3) node[pos=.5] {$G_{5}$};
            \draw  (10.8,10.3) rectangle (11.8,11.3) node[pos=.5] {$G_{6}$};
            \draw  (10.8,9.2) rectangle (11.8,8.2) node[pos=.5] {$G_{12}$};
            \draw  (10.8,8.2) rectangle (9.8,9.2) node[pos=.5] {$G_{11}$};
            \draw  (9.8,9.2) rectangle (8.8,8.2) node[pos=.5] {$G_{10}$};
            \draw  (8.8,8.2) rectangle (7.8,9.2) node[pos=.5] {$G_{9}$};
            \draw  (7.8,9.2) rectangle (6.8,8.2) node[pos=.5] {$G_{8}$};
            \draw  (6.8,8.2) rectangle (5.8,9.2) node[pos=.5] {$G_{7}$};
            \draw  (6.8,7.2) rectangle (5.8,6.2) node[pos=.5] {$G_{13}$};
            \draw  (6.8,6.2) rectangle (7.8,7.2) node[pos=.5] {$G_{14}$};
            \draw  (7.8,6.2) rectangle (8.8,7.2) node[pos=.5] {$G_{15}$};
            \draw  (8.8,7.2) rectangle (9.8,6.2) node[pos=.5] {$G_{16}$};
            \draw  (9.8,6.2) rectangle (10.8,7.2) node[pos=.5] {$G_{17}$};
            \draw  (10.8,5.2) rectangle (11.8,4.2) node[pos=.5] {$G_{24}$};
            \draw  (10.8,4.2) rectangle (9.8,5.2) node[pos=.5] {$G_{23}$};
            \draw  (9.8,5.2) rectangle (8.8,4.2) node[pos=.5] {$G_{22}$};
            \draw  (8.8,4.2) rectangle (7.8,5.2) node[pos=.5] {$G_{21}$};
            \draw  (7.8,5.2) rectangle (6.8,4.2) node[pos=.5] {$G_{20}$};
            \draw  (6.8,4.2) rectangle (5.8,5.2) node[pos=.5] {$G_{19}$};
            \draw  (10.8,7.2) rectangle (11.8,6.2) node[pos=.5] {$G_{18}$};
            
            % Retângulos coloridos com diferentes tipos de traços escalados
            \draw [color={rgb,255:red,0; green,0; blue,255}, ultra thick, line width=1pt, loosely dashed] (12.2,7.8) rectangle (5.4,9.6) node[pos=.5] {};
            \draw [color={green!50!black}, ultra thick, line width=1pt, loosely dotted] (10.5,11.6) rectangle (12.1,10.0) node[pos=.5] {};
            \draw [color={rgb,255:red,255; green,0; blue,0}, thick, line width=1pt, loosely dashdotted] (13,3.2) rectangle (4.6,12.4) node[pos=.5] {};
        \end{circuitikz}
        }%
    \end{minipage}
    \begin{center}
        \begin{tabular}{ccc}
        {\color{green!50!black} \textbf{. . .}}  & {\color{blue} \textbf{- - -}} & {\color{red} \textbf{- . -}} \\
        $Gene$  &  $Chromosome$ & $Population$
        \end{tabular}
    \end{center}
    \caption{Representation of all elements used in a genetic algorithm.}
    \label{fig:ga}
\end{figure}

    
    In summary, the success of a genetic algorithm heavily depends on how well the problem is encoded into genes, chromosomes, and populations, as these structures form the foundation for all subsequent evolutionary processes.
    
    \subsection{Mutations and Crossovers}
    
    Genetic algorithms rely on two primary genetic operators to evolve the population of chromosomes: mutation and crossover. These operators introduce variety into the population, allowing the algorithm to explore new solution spaces and avoid premature convergence to local optima.
    
    The mutation operator introduces random changes to one or more genes within a chromosome. This randomness simulates natural mutations in biological organisms, helping to maintain genetic diversity in the population and explore previously unexplored regions of the solution space.
    
    The crossover operator combines two parent chromosomes to produce offspring. During this process, the genetic material from each parent is exchanged, creating new chromosomes that contain traits from both parents. This operator simulates sexual reproduction in biology, where offspring inherit features from both parents, contributing to the evolution of better solutions over generations.
    
    In conclusion, the mutation and crossover operators are vital for balancing the trade-off between exploration (searching new areas of the solution space) and exploitation (refining existing good solutions) in genetic algorithms.

\chapter{CHARACTERIZATION OF GROUPING PROBLEMS} \label{chap:grouping_problems} 

    In this chapter, we explore the characterization of grouping problems, focusing on their definitions, common types, and recent developments. We begin by defining how grouping problems are approached in industry. This concept is then related to graph theory, where vertices represent objects and edges represent connections, providing a theoretical framework for understanding grouping problems \cite{newman2018networks}.
    
    We then delve into common types of grouping problems, including explicit matching, implicit matching, one-to-one matching, one-to-many matching, many-to-many matching, user-item matching, and others \cite{gusfield1989stable, demange1986multi}. Each type presents unique challenges and applications in various industries.
    
    Finally, we discuss recent developments in the field of grouping problems, including advanced algorithms for solving matching problems efficiently and applications in diverse fields such as quantum computing \cite{quantum_matching}, artificial inteligence \cite{gnn_graph_matching, cross_modal_matching}, and Humanitarian and Resource Allocation \cite{humanitarian_routing}. By examining these developments, we aim to provide a comprehensive overview of the current state of grouping problems and their significance in various domains \cite{ieee_survey}.
    
\section{HOW TO DEFINE AN INDUSTRY MATCHING PROBLEM}

    As the name indicates, they are problems where the objective is to match some elements, usually the matching optimizes a function.
    
    When faced with a set of objects, such as job oppenings and candidates, the objective is to group these objects, assing candidates to jobs in this context, where each group generate a set of statistics, and this statistics are used to optimize an objective function. This concept is well-established in the field of operations research and combinatorial optimization \cite{manlove2013algorithmics}.

    \subsection{Job Assignment} \label{subsec:job_assignment}
        One of the most well-known examples of a matching problem is job assignment. The objective is to match job positions with candidates who possess the necessary skills, experience, and qualifications. The process involves considering the specific requirements of each position and the attributes of the candidates, aiming to maximize overall efficiency or satisfaction. This type of problem has been widely studied, particularly in the context of labor markets and automated recruitment systems \cite{roth1990two}.

    \subsection{Resource Allocation} \label{subsec:resource_allocation}
        Resource allocation problems involve distributing limited resources—such as rooms, equipment, or funds—among competing tasks or projects. The challenge lies in ensuring that the allocation maximizes efficiency while meeting constraints, such as availability and demand. These problems are critical in industries like healthcare, manufacturing, and project management, where effective resource distribution directly impacts outcomes \cite{schrijver2003combinatorial}.
    
    \subsection{Network Routing} \label{subsec:network_routing}
        In network routing problems, the goal is to determine the optimal way to route data, goods, or resources through a network. This involves minimizing costs, delays, or energy consumption while ensuring that demands are met across the network. Such problems are particularly relevant in logistics, telecommunications, and supply chain management, where optimizing flow is essential for performance \cite{ahuja1993network}.
    
    \subsection{Stable Matching} \label{subsec:stable_matching}
        Stable matching problems focus on creating pairings where no two entities would prefer being matched with each other over their current assignments. The stable marriage problem is a classic example, where the objective is to create stable pairs based on mutual preferences. Applications of stable matching range from college admissions processes to organ donation programs, where stability and fairness are crucial \cite{gusfield1989stable, demange1986multi}.
    
    
    
    These problems are fundamental in various industries and have been extensively studied \cite{manlove2013algorithmics}.


    \subsection{Industrial Terminology compared to Graph Theory}
    
   In the context of graph theory, vertices may represent objects, and edges represent the connections or relationships between these objects \cite{west2001introduction, diestel2017graph}. 
    This basic terminology is crucial when translating industrial problems into graph models.

    Edges often contain statistics that can represent preferences, costs, capacities, and other characteristics.
    For example, in the stable marriage problem  (see Subsection \ref{subsec:stable_matching}), edges' statistics represent preferences between pairs \cite{gale1962college}.
    In the Hungarian algorithm for the assignment problem  (see Subsection \ref{subsec:resource_allocation}), edges represent costs associated with assignments \cite{kuhn1955hungarian}. 
    In network flow problems (see Subsection \ref{subsec:network_routing}), edges represent capacities that limit the amount of flow through the network \cite{ford1956maximal}.
    
    For industry applications, any graph cover problems are often collectively referred to as matching problems \cite{larman2004applying, fowler2003patterns}. However, to avoid confusion, in the context of this work, we will use the terms ``grouping'' and ``matching"" interchangeably.

    For instance, in a logistics network, vertices could represent warehouses and stores, while edges represent possible delivery routes. 
    Finding an optimal set of routes that minimizes cost can be modeled as a grouping problem.

    Using this terminology allows us to frame complex industrial problems in a structured way, making it easier to apply graph-theoretical algorithms and techniques to find optimal solutions.
    
    
    \section{Usual Matching Problems}
        
        In this section, grouping problems will be referred to as "matching" problems, as this is how they are often called in the industry. The categorization of problems into subcategories, based on the classification provided by the survey \cite{ieee_survey}, reveals several common grouping problem structures, such as explicit and implicit matching. Table \ref{tbl:summarisation_applications_matching} shows a detailed relation of problems, characteristics, and applications. Other characteristics that can be applied to all problems include online/offline (dynamism), scalability, heterogeneity, noise tolerance, and privacy requirements \cite{ieee_survey}.
        
        \begin{table}[ht]
    \centering
    \scalebox{0.64}{
        \begin{tabular}{|c|c|c|c|c|}
            \hline
            \textbf{Category}                   & \textbf{Sub-category}                     & \textbf{Algorithms}                     & \textbf{References}                                                                                                                           & \textbf{Applications}                                                                                                           \\ \hline
            \multirow{4}{*}{Explicit Matching}  & One-to-one Matching                       & -                                       & \cite{gale1962college,roth1992two,becker1973theory,bergstrom1993courtship}                                                                                                        & marriage market                                                                                                                 \\ \cline{2-5} 
                                                & \multirow{2}{*}{Many-to-one 
                                                Matching}     & \multirow{2}{*}{-}                      & \cite{bergstrom1993courtship,gale1962college,roth1992two,rosen1981economics,roth1984stability}                                                                                              & job matching                                                                                                                    \\ \cline{4-5} 
                                                &                                           &                                         & \cite{kremer1993ring,gabaix2008ceo,tervio2008ceo}                                                                                                                 & pay matching                                                                                                                    \\ \cline{2-5} 
                                                & Many-to-many Matching                     & Optimization algorithms                 & \cite{gabaix2008ceo,bergstrom1993courtship,roth1992two,becker1973theory}                                                                                                         & \begin{tabular}[c]{@{}c@{}}cognitive radio networks; \\ D2D communications\end{tabular}                                         \\ \hline
            \multirow{11}{*}{Implicit Matching} & \multirow{3}{*}{Retrieval Matching}       & Traditional matching algorithms         & \cite{ramos2003using,salton1988term,becker1973theory,dumais2004latent,hofmann1999probabilistic,bergstrom1993courtship}                                                                                    & \multirow{3}{*}{\begin{tabular}[c]{@{}c@{}}machine translation;\\ expertise matching; \\ question-answer matching\end{tabular}} \\ \cline{3-4}
                                                &                                           & Representation-based algorithms         & \cite{bengio2009learning,bengio2009learning,he2017neural,che2019stable}                                                                                                       &                                                                                                                                 \\ \cline{3-4}
                                                &                                           & Interaction-based algorithms            & \begin{tabular}[c]{@{}c@{}}\cite{dumais2004latent,bengio2009learning,che2019stable,che2019stable,becker1973theory}, \\\cite{gabaix2008ceo,che2019stable,dumais2004latent,roth1992two,becker1973theory}\end{tabular} &                                                                                                                                 \\ \cline{2-5} 
                                                & \multirow{3}{*}{User-item Matching}       & Basic algorithms                        & \cite{su2009survey,che2019stable}                                                                                                                            & \multirow{3}{*}{recommendation systems}                                                                                         \\ \cline{3-4}
                                                &                                           & Representation-based algorithms         & \cite{pang2016text,dumais2004latent,koren2015advances,dumais2004latent,roth1992two}                                                                                              &                                                                                                                                 \\ \cline{3-4}
                                                &                                           & Matching function-based algorithms      & \cite{blei2003latent,bengio2009learning,he2017neuralfm,gabaix2008ceo}                                                                                                       &                                                                                                                                 \\ \cline{2-5} 
                                                & \multirow{3}{*}{Entity-relation Matching} & Factorization-based algorithms          & \cite{dumais2004latent,bay2008surf,gabaix2008ceo,gabaix2008ceo}
                                                & \multirow{3}{*}{\begin{tabular}[c]{@{}c@{}}recommendation systems;\\ knowledge fusion; \\ information retrieval\end{tabular}}   \\ \cline{3-4}
                                                &                                           & Neural network-based algorithms         &  \cite{rosen1981economics,blei2003latent,becker1973theory,roth1992two}                                                                                                      &                                                                                                                                 \\ \cline{3-4}
                                                &                                           & Translational distance-based algorithms &  \cite{rosen1981economics,roth1992two,dumais2004latent,bergstrom1993courtship,roth1992two}                                                                                             &                                                                                                                                 \\ \cline{2-5} 
                                                & \multirow{2}{*}{Image Matching}           & Area-based algorithms                   & \cite{blei2003latent,gruen1985adaptive,dumais2004latent,dumais2004latent,yang2018image}                                                                                              & \multirow{2}{*}{\begin{tabular}[c]{@{}c@{}}robot vision; \\ object recognition;\\ medical image diagnosis\end{tabular}}         \\ \cline{3-4}
                                                &                                           & Feature-based algorithms                & \begin{tabular}[c]{@{}c@{}}\cite{blei2003latent,lowe2004distinctive,gabaix2008ceo,becker1973theory,rosen1981economics}, \\ \cite{rosen1981economics,roth1992two,blei2003latent,roth1992two,roth1992two}\end{tabular} &                                                                                                                                 \\ \hline
        \end{tabular}
    }
\caption{Summarisation of Matching Algorithms and Applications According to \cite{ieee_survey}.}
    \label{tbl:summarisation_applications_matching}
\end{table}

        
        Matching problems can be broadly divided into two main categories: explicit and implicit.
        
        \subsection{Explicit Matching}
            Explicit matching problems involve objects that have preferences about whom they prefer to group with \cite{ieee_survey, manlove2013algorithmics}. These can be further categorized into:
            
            \begin{itemize}
                \item \textbf{One-to-one}: also known as bipartite matching, common in job assignment where each job is assigned to a single candidate and vice versa \cite{kuhn1955hungarian}.
                \item \textbf{Many-to-one}: found in scenarios like college admissions, where multiple students can be assigned to a single college \cite{gale1962college}.
                \item \textbf{Many-to-many}: occurs in contexts such as organ donation, where multiple donors can provide organs to multiple recipients \cite{roth2004kidney}.
            \end{itemize}
        
        \subsection{Implicit Matching}
            Implicit matching focuses on calculating the 'score' of the grouping without explicit preferences, optimizing based on a grouping function \cite{ieee_survey}. Examples include:
            
            \begin{itemize}
                \item \textbf{Retrieval Matching}: in information retrieval, it involves users inputting queries that express their needs and obtaining the desired information from a search engine's database \cite{manning2008introduction}.
                \item \textbf{User-item Matching}: in recommender systems, it helps users obtain items of interest accurately \cite{ricci2011introduction}.
                \item \textbf{Entity-relation Matching}: it involves the use of knowledge graphs in applications such as semantic parsing, information extraction, link prediction, recommender systems, and question answering \cite{ji2021survey}.
                \item \textbf{Image Matching}: it compares different images to identify similarities or correspondences, used in fields like computer vision and pattern recognition \cite{szeliski2010computer}.
            \end{itemize}

        

    \section{Recent Developments}
    
    The field of matching problems is constantly evolving, with ongoing research introducing new problems, solutions, and algorithmic approaches to address the complexity and diversity of these challenges in industry. In this section, we outline a preliminary review aimed at identifying the latest developments in matching problems, focusing on advancements made since 2021.
    
    To capture the most recent and relevant studies, we conducted an initial search in Google Scholar using the following search string:
    
    \begin{quote}
    (TITLE-ABS-KEY("pairing") OR TITLE-ABS-KEY("matching") OR TITLE-ABS-KEY("grouping")) AND (TITLE-ABS-KEY("solver") OR TITLE-ABS-KEY("algorithm")) AND TITLE-ABS-KEY("graph") AND PUBYEAR AFTER 2021
    \end{quote}
    
    The review process will involve applying inclusion criteria to filter studies that contribute to the development of matching algorithms, whether through incremental improvements, novel problem types, algorithmic frameworks, or solvers that expand the applicability of matching techniques in industry.
    
    After collecting the top 100 studies by number of citations, they were filtered by language, retaining only those in English. Then, based on titles and abstracts, works unrelated to matching were removed, including six that focused on diverse clustering problems.  Figure \ref{fig:bib} will illustrate the selection process, showing the number of studies removed and retained at each stage. Furthermore, we analyze the most prominent topics that emerged from the remaining studies.
    
    \begin{figure}[!ht] \label{gig:bib}
\centering

\tikzstyle{process} = [rectangle, minimum width=1.5cm, minimum height=6.5cm, text centered, draw=black, fill=white, text width=1.75cm, font=\small]
\tikzstyle{mid_process} = [rectangle, minimum width=1cm, minimum height=1cm, text centered, draw=black, fill=white, text width=1.75cm, font=\small]
\tikzstyle{arrow} = [thick,->,>=stealth]

\scalebox{0.7}{%
\begin{tikzpicture}[node distance=3cm, auto]

    % Nodes
    \node (start) [process] {\textbf{Search string execution on Google Scholar}};
    \node (remaining) [mid_process, right of=start] {\textbf{Remaining} \\ 99};
    \node (repeat) [mid_process, above of=remaining] {\textbf{Not in english} \\ 1};
    \node (returned) [mid_process, below of=remaining] {\textbf{Returned} \\ 100};
    
    \node (step1) [process, right of=remaining] {\textbf{Step 1} \\ Selection by reading \\ title and abstract};
    \node (included1) [mid_process, right of=step1] {\textbf{Selected} \\ 93};
    \node (excluded1) [mid_process, above of=included1] {\textbf{Excluded} \\ 6};


    \node (step2) [process, right of=included1] {\textbf{Step 2} \\ Selection a work for each \\ emerging trend};
    \node (included2) [mid_process, right of=step2] {\textbf{Selected} \\ 9};
    \node (excluded2) [mid_process, above of=included2] {\textbf{Excluded} \\ 84};

    % Arrows
    \draw [arrow] (start) -- (remaining);
    \draw [arrow] (start) -- (repeat);
    \draw [arrow] (start) -- (returned);
    
    \draw [arrow] (remaining) -- (step1);
    \draw [arrow] (step1) -- (excluded1);
    \draw [arrow] (step1) -- (included1);

    
    \draw [arrow] (included1) -- (step2);
    \draw [arrow] (step2) -- (excluded2);
    \draw [arrow] (step2) -- (included2);
    

\end{tikzpicture}
}

\caption{Selection Process of the Articles.}
\label{fig:bib}
\end{figure}
    
    \subsection{Overview of Recent Existing Approaches}
    Table \ref{tbl:old_types} presents a summary of the selected studies in step 1, categorizing them based on their focus and contributions to the field of matching. Each study is classified according to the definitions of Implicit Matching by \cite{ieee_survey}. Additionally, we detail the common subjects treated by the new publications.

    

    \begin{table}[ht]
\centering
\begin{tabular}{|c|c|}
\hline
\textbf{Implicit Matching Category} & \textbf{Total} \\
\hline
Retrieval Matching & 16 \\
User-item Matching & 10 \\
Image Matching & 33 \\
Entity-relation Matching & 34 \\
\hline
\end{tabular}
\caption{Total of Papersa after step 1 by Implicit Matching Category.}
\label{tbl:old_types}
\end{table}

    
    This review aims to demonstrate the breadth of recent developments in matching, highlighting both the versatility of applications and the growing need for centralized tools that simplify the implementation and adaptation of matching solutions across various industrial contexts.

    \subsection{Emerging Trends in Graph Matching Algorithms}
        Here we overview the 9 selected studies about emerging trends.
        \begin{enumerate}
            \item \textbf{Quantum Computing}
            \begin{itemize}
                \item \textbf{Trend:} Quantum computing is starting to impact optimization and matching algorithms. Some works suggest that quantum algorithms are being explored for graph-based problems, particularly in sparse data and matching contexts \cite{quantum_matching}.
                \item \textbf{Context:} Quantum algorithms may offer exponential speedup over classical methods in certain areas, and combining quantum techniques with combinatorial optimization is a growing field.
            \end{itemize}
            
            \item \textbf{AI and Deep Learning in Graph Matching}
            \begin{itemize}
                \item \textbf{Trend:} Many studies focus on graph neural networks (GNNs) and deep learning in graph matching \cite{gnn_graph_matching, cross_modal_matching}.
                \item \textbf{Context:} AI, particularly deep learning, is being applied to improve the accuracy and efficiency of graph matching, especially in image recognition, NLP, and multi-modal data integration.
            \end{itemize}
        
            \item \textbf{Video and Image Recognition}
            \begin{itemize}
                \item \textbf{Trend:} Recognition algorithms using graph matching are becoming more common, especially in computer vision \cite{image_keypoint_matching}.
                \item \textbf{Context:} This is closely related to AI, where graph-based methods are used for tasks like multi-object tracking and 3D object detection.
            \end{itemize}
        
            \item \textbf{Combinatorial Algorithms and Optimization}
            \begin{itemize}
                \item \textbf{Trend:} Classical combinatorial algorithms remain an important area of research \cite{faster_bipartite_matching}.
                \item \textbf{Context:} These algorithms are central in problems like scheduling, resource allocation, and network flow.
            \end{itemize}
        
            \item \textbf{Networks and Multi-Agent Systems}
            \begin{itemize}
                \item \textbf{Trend:} Network theory and multi-agent systems use graph matching for optimization \cite{network_matching, privacy_graph_matching}.
                \item \textbf{Context:} These topics focus on applications in wireless communication, sensor networks, and privacy-preserving computation.
            \end{itemize}
        
            \item \textbf{Cross-modal and Multi-modal Matching}
            \begin{itemize}
                \item \textbf{Trend:} Matching across different types of data (e.g., text and images) is an emerging trend in AI and cross-modal retrieval \cite{cross_modal_matching}.
                \item \textbf{Context:} These algorithms are applicable in industries like e-commerce, digital media, and autonomous systems.
            \end{itemize}
        
            \item \textbf{Humanitarian and Resource Allocation Applications}
            \begin{itemize}
                \item \textbf{Trend:} Graph matching and optimization algorithms are used in resource allocation and humanitarian problems \cite{humanitarian_routing}.
                \item \textbf{Context:} This trend focuses on real-world applications like disaster management and healthcare.
            \end{itemize}
        
            \item \textbf{Graph Matching for Knowledge Graphs and Semantic Systems}
            \begin{itemize}
                \item \textbf{Trend:} Knowledge graphs and semantic matching are gaining attention in AI reasoning and data integration \cite{knowledge_graph_matching}.
                \item \textbf{Context:} These methods are key in NLP, semantic web technologies, and AI-driven data systems.
            \end{itemize}
        
            \item \textbf{Privacy and Security in Graph Matching}
            \begin{itemize}
                \item \textbf{Trend:} Privacy-preserving graph matching is a significant concern, particularly for sensitive data \cite{privacy_graph_matching}.
                \item \textbf{Context:} These methods are relevant in domains like healthcare, finance, and social networks.
            \end{itemize}
        \end{enumerate}
        


\chapter{Designing a pipeline for solving grouping problems} \label{chap:design}

    Grouping problems are a common challenge in various fields, requiring an effective and efficient approach for their resolution. To achieve this, it is essential to represent these problems in both a simple and generic manner. Simplicity ensures ease of understanding, while a generic approach guarantees that the representation can encompass the majority of grouping problems encountered.
    
    In this chapter, we propose using the SCF (Set Coverage Framework), which uses the Set Cover problem (see Section \ref{WSC}) as the foundational representation for all grouping problems. The optimization version of the Set Cover problem is NP-Hard, indicating that any decidable grouping problem can be reduced to it \cite{garey1979computers}. This approach provides a unified framework that simplifies the conceptualization and handling of diverse grouping problems.
    
    To manage the specific properties of each problem, we introduce statistics that can identify the nuances of the problem, such as the number of groups, whether the problem is weight-based or capacity-based, whether an object is unique or can be in multiple groups, and other relevant attributes. These statistics allow for the customization and fine-tuning of the problem representation, facilitating the reduction of complex problems to simpler forms, potentially even to P problems, enabling the use of specialized solvers \cite{ieee_survey}.
    
    Furthermore, we discuss the application of a metaheuristic solver for the SCF problem, capable of handling all cases, including those where problem reduction is not feasible \cite{ren2021matching}. This solver ensures a robust and versatile solution approach, even for the most challenging instances.
    
    This chapter aims to lay the groundwork for a systematic pipeline in solving grouping problems, offering a comprehensive representation and a flexible, powerful solution mechanism.
    
    \section{Grouping problems as the Set Cover Framework}

        The Set Cover Framework is a way to represent grouping problems as the the Set Cover problem (see Section \ref{WSC}) and its variants thorough objects and its relations, since they are straightforward to be represented in this manner as shown in Figures \ref{fig:set_coverage_framework_1} and \ref{fig:set_coverage_framework_2}.
        %
        The weighted version of the Set Cover can consider the weights as being defined and computed based on the relation of those objects.

        In addition, every decidable grouping problem can be seen as a set cover problem.
        While it sounds as a bold statement, this is in fact, a trivial one, due to the fact that the decision version of the Set-cover is a NP-Complete problem \cite{karp1972reducibility}.
        %
        It implicates that any decidable grouping problem can theoretically be reduced to it \cite{garey1979computers}.

        \begin{figure}[!ht]
    \centering
    \begin{subfigure}{0.53\textwidth}
        \centering
        \resizebox{\textwidth}{!}{%
        \begin{circuitikz}
        \tikzstyle{every node}=[font=\Huge]
        \draw  (14.25,14.5) circle (0cm);
        \draw [, dashed] (10.5,16) circle (3.75cm);
        \draw [, dashed] (25.5,15.5) circle (6.25cm);
        \draw [, dashed] (12.75,5) circle (4.25cm);
        \draw  (8.75,14) circle (0.5cm);
        \draw  (10,15.5) circle (0.5cm);
        \draw  (12,15) circle (0.5cm);
        \draw  (10.25,4) circle (0.5cm);
        \draw  (13,6.75) circle (0.5cm);
        \draw  (12.75,2) circle (0.5cm);
        \draw  (14.5,4) circle (0.5cm);
        \draw  (14.75,5.75) circle (0.5cm);
        \draw  (11.5,4.5) circle (0.5cm);
        \draw  (22.75,11.75) circle (0.5cm);
        \draw  (29.75,16) circle (0.5cm);
        \draw  (28.5,13.25) circle (0.5cm);
        \draw  (26.25,11.75) circle (0.5cm);
        \draw  (22.25,16) circle (0.5cm);
        \draw  (25.25,14.75) circle (0.5cm);
        \draw  (27,18.5) circle (0.5cm);
        \draw  (27.25,16) circle (0.5cm);
        \draw  (21,14) circle (0.5cm);
        \draw  (23.5,17.75) circle (0.5cm);
        \draw  (8,17) circle (0.5cm);
        \draw  (12.5,17) circle (0.5cm);
        \node [font=\Huge] at (10.5,18.5) {Class A};
        \node [font=\Huge] at (12.75,8.25) {Class B};
        \node [font=\Huge] at (25,20.25) {Class C};
        \draw [ ultra thick, color={rgb,255:red,255; green,0; blue,0}, line width=1pt, dashed] (8,14.25) -- (8.75,0.75);
        \draw [ ultra thick, color={rgb,255:red,255; green,0; blue,0}, line width=1pt, dashed] (8.75,0.75) -- (15.25,1.75);
        \draw [ ultra thick, color={rgb,255:red,255; green,0; blue,0}, line width=1pt, dashed] (15.25,1.75) -- (30.5,13.5);
        \draw [ ultra thick, color={rgb,255:red,255; green,0; blue,0}, line width=1pt, dashed] (30.5,13.5) -- (28,14.75);
        \draw [ ultra thick, color={rgb,255:red,255; green,0; blue,0}, line width=1pt, dashed] (28,14.75) -- (27.25,11.5);
        \draw [ ultra thick, color={rgb,255:red,255; green,0; blue,0}, line width=1pt, dashed] (27.25,11.5) -- (14.25,2.75);
        \draw [ ultra thick, color={rgb,255:red,255; green,0; blue,0}, line width=1pt, dashed] (14.25,2.75) -- (9.75,2.5);
        \draw [ ultra thick, color={rgb,255:red,255; green,0; blue,0}, line width=1pt, dashed] (9.75,2.5) -- (9,11.75);
        \draw [ ultra thick, color={rgb,255:red,255; green,0; blue,0}, line width=1pt, dashed] (9,11.75) -- (9.75,14.5);
        \draw [ ultra thick, color={rgb,255:red,255; green,0; blue,0}, line width=1pt, dashed] (9.75,14.5) -- (8.25,14.75);
        \draw [ ultra thick, color={rgb,255:red,255; green,0; blue,0}, line width=1pt, dashed] (8.25,14.75) -- (8,14.25);
        \draw [ ultra thick, color={rgb,255:red,0; green,85; blue,255}, line width=1pt, dashed] (12,17.75) -- (22,14.5);
        \draw [ ultra thick, color={rgb,255:red,0; green,85; blue,255}, line width=1pt, dashed] (22,14.5) -- (15.25,5);
        \draw [ ultra thick, color={rgb,255:red,0; green,85; blue,255}, line width=1pt, dashed] (15.25,5) -- (14,5.5);
        \draw [ ultra thick, color={rgb,255:red,0; green,85; blue,255}, line width=1pt, dashed] (14,5.5) -- (19.75,13.5);
        \draw [ ultra thick, color={rgb,255:red,0; green,85; blue,255}, line width=1pt, dashed] (19.75,13.5) -- (11.25,16.5);
        \draw [ ultra thick, color={rgb,255:red,0; green,85; blue,255}, line width=1pt, dashed] (11.25,16.5) -- (12,17.75);
        \draw [ ultra thick, color={rgb,255:red,42; green,81; blue,11}, line width=1pt, dashed] (11.25,15.25) .. controls (11.5,10.25) and (11.75,10.25) .. (12.25,5.5);
        \draw [ ultra thick, color={rgb,255:red,42; green,81; blue,11}, line width=1pt, dashed] (12.25,5.5) .. controls (13,6) and (13.25,6) .. (14.25,6.5);
        \draw [ ultra thick, color={rgb,255:red,42; green,81; blue,11}, line width=1pt, dashed] (14.25,6.5) .. controls (19,9) and (19.25,9) .. (24.25,11.75);
        \draw [ ultra thick, color={rgb,255:red,42; green,81; blue,11}, line width=1pt, dashed] (24.25,11.75) .. controls (23.75,12.25) and (24,12.25) .. (23.75,13);
        \draw [ ultra thick, color={rgb,255:red,42; green,81; blue,11}, line width=1pt, dashed] (23.75,13) .. controls (18.5,10.5) and (18.75,10.5) .. (13.5,8);
        \draw [ ultra thick, color={rgb,255:red,42; green,81; blue,11}, line width=1pt, dashed] (13.5,8) .. controls (13,12) and (13.25,12) .. (12.75,16);
        \draw [ ultra thick, color={rgb,255:red,42; green,81; blue,11}, line width=1pt, dashed] (12.75,16) .. controls (11.75,15.5) and (12,15.5) .. (11.25,15.25);
        \end{circuitikz}
        }%
        \caption{Set Coverage Over Elements from Multiple Classes.}
        \label{fig:set_coverage_framework_1}
    \end{subfigure}
    \hfill
    \begin{subfigure}{0.45\textwidth}
    \centering
    \resizebox{\textwidth}{!}{%
    \begin{circuitikz}
    \tikzstyle{every node}=[font=\Huge]
    \draw [, line width=1pt ] (9,13.25) circle (0.75cm);
    \draw [, line width=1pt ] (20.25,15.25) circle (0.75cm);
    \draw [, line width=1pt ] (11.5,5.75) circle (0.75cm);
    \draw [, line width=1pt ] (15.25,12.25) circle (0.75cm);
    \draw [, line width=1pt ] (21.25,10.25) circle (0.75cm);
    \draw [, line width=1pt ] (16.5,7) circle (0.75cm);
    \draw [, line width=1pt ] (8,9.75) circle (0.75cm);
    
                \draw [ color={rgb,255:red,0; green,85; blue,255}, line width=1pt, dashed, ultra thick] (7.75,14.25) -- (7.5,12.25);
                \draw [ color={rgb,255:red,0; green,85; blue,255}, line width=1pt, dashed, ultra thick] (17,5) .. controls (20,7.75) and (20.25,7.75) .. (23.25,10.5);
                \draw [ color={rgb,255:red,0; green,85; blue,255}, line width=1pt, dashed, ultra thick] (23.25,10.5) .. controls (22,11.25) and (22.25,11.25) .. (21.25,12);
                \draw [ color={rgb,255:red,0; green,85; blue,255}, line width=1pt, dashed, ultra thick] (21.25,12) .. controls (18,10.25) and (18.25,10.25) .. (15,8.75);
                \draw [ color={rgb,255:red,0; green,85; blue,255}, line width=1pt, dashed, ultra thick] (15,8.75) .. controls (12,11.75) and (12.25,11.75) .. (9.25,15);
                \draw [ color={rgb,255:red,0; green,85; blue,255}, line width=1pt, dashed, ultra thick] (9.25,15) .. controls (8.25,14.5) and (8.5,14.5) .. (7.75,14.25);
                \draw [ color={rgb,255:red,0; green,85; blue,255}, line width=1pt, dashed, ultra thick] (7.5,12.25) -- (10.25,8.25);
                \draw [ color={rgb,255:red,0; green,85; blue,255}, line width=1pt, dashed, ultra thick] (10.25,8.25) -- (10.25,3);
                \draw [ color={rgb,255:red,0; green,85; blue,255}, line width=1pt, dashed, ultra thick] (10.25,3) -- (17,5);
                
                \draw [ color={rgb,255:red,255; green,0; blue,0}, line width=1pt, loosely dotted, ultra thick] (12.25,4.75) .. controls (14.5,8.25) and (14.75,8.25) .. (17,12);
                \draw [ color={rgb,255:red,255; green,0; blue,0}, line width=1pt, loosely dotted, ultra thick] (17,12) .. controls (19.75,13.25) and (20,13.25) .. (22.75,14.5);
                \draw [ color={rgb,255:red,255; green,0; blue,0}, line width=1pt, loosely dotted, ultra thick] (22.75,14.5) .. controls (21.5,15.75) and (21.75,15.75) .. (20.75,17);
                \draw [ color={rgb,255:red,255; green,0; blue,0}, line width=1pt, loosely dotted, ultra thick] (20.75,17) .. controls (16.75,14.5) and (17,14.5) .. (13.25,12.25);
                \draw [ color={rgb,255:red,255; green,0; blue,0}, line width=1pt, loosely dotted, ultra thick] (13.25,12.25) .. controls (11.25,8.25) and (11.5,8.25) .. (9.5,4.5);
                \draw [ color={rgb,255:red,255; green,0; blue,0}, line width=1pt, loosely dotted, ultra thick] (9.5,4.5) .. controls (10.5,4.25) and (10.75,4.25) .. (12,4);
                \draw [ color={rgb,255:red,255; green,0; blue,0}, line width=1pt, loosely dotted, ultra thick] (12,4) .. controls (12,4.25) and (12.25,4.25) .. (12.25,4.75);
                
                \draw [ color={rgb,255:red,0; green,143; blue,2}, line width=1pt, loosely dashdotted, ultra thick] (6.75,7.75) .. controls (6,9.25) and (6.25,9.25) .. (5.75,10.75);
                \draw [ color={rgb,255:red,0; green,143; blue,2}, line width=1pt, loosely dashdotted, ultra thick] (5.75,10.75) .. controls (13.75,14.5) and (14,14.5) .. (22.25,18.5);
                \draw [ color={rgb,255:red,0; green,143; blue,2}, line width=1pt, loosely dashdotted, ultra thick] (22.25,18.5) .. controls (23.25,13.5) and (23.5,13.5) .. (24.5,8.5);
                \draw [ color={rgb,255:red,0; green,143; blue,2}, line width=1pt, loosely dashdotted, ultra thick] (24.5,8.5) .. controls (23.25,7.75) and (23.5,7.75) .. (22.5,7);
                \draw [ color={rgb,255:red,0; green,143; blue,2}, line width=1pt, loosely dashdotted, ultra thick] (6.75,7.75) .. controls (12.75,9.75) and (13,9.75) .. (19,12);
                \draw [ color={rgb,255:red,0; green,143; blue,2}, line width=1pt, loosely dashdotted, ultra thick] (19,12) .. controls (20.5,9.5) and (20.75,9.5) .. (22.5,7);

    
    \draw [, line width=2pt , loosely dashed] (15.5,11.25) circle (10.75cm);
    \node [font=\Huge] at (15.25,19.25) {Class A};
    \end{circuitikz}
    }%
    \caption{Set Coverage over Elements from a Single Class.}
    \label{fig:set_coverage_framework_2}
    \end{subfigure}
    
    \caption[Visualizing Set Cover Problems in the context of Classes and its relations.]{Visualizing Set Cover Problems in the context of Classes and its relations.
    Every small circle represent an instance of a class.
    The colored boxes represent both relations between classes and sets.}
    \label{fig:set_coverage_framework}
\end{figure}
        
    
        \subsection{Representation}
             As the main purpose of Grouping Problems is to group objects, they can intuitively be represent as a relational diagram. Figure \ref{fig:examples_grouping_as_relations} shows how to represent several grouping problems earlier discussed in Chapter \ref{chap:grouping_problems}.
             %
             Note that the grouping problems can always be represented as a many-to-many relation. Here we do not refer to the many-to-many matching from Chapter \ref{chap:grouping_problems}, but we refer to the many-to-many object relation, explained at Section \ref{sec:orm}.

             \begin{figure}[!ht]
    \centering
    \begin{subfigure}[b]{0.45\textwidth}
        \centering
        \scalebox{0.6}{%
        \begin{tikzpicture}[scale=0.4]
            \begin{class}{Worker}{0, 6}
                \attribute {id : String}
            \end{class}
            \begin{class}{Job}{15, 6}
                \attribute {id : String}
            \end{class}

            \begin{class}{Grouping}{7.5, -2}
                \attribute {id : String}
                \attribute {JobId : String}
                \attribute {WorkerId : String}
            \end{class}
            \association {Worker}{}{1..1}{Grouping}{0..1}{}
            \association {Grouping}{}{1..1}{Job}{1..1}{}
        \end{tikzpicture}
        }
        \caption{Job Assignment Problem.}
        \label{fig:uml_job_matching_problem}
    \end{subfigure}
    \hfill
    \begin{subfigure}[b]{0.45\textwidth}
        \centering
        \scalebox{0.6}{%
        \begin{tikzpicture}[scale=0.4]
            \begin{class}{Room}{0, 6}
                \attribute {id : String}
            \end{class}
            \begin{class}{Fund}{16, 6}
                \attribute {id : String}
            \end{class}
            \begin{class}{Project}{0, -3.5}
                \attribute {id : String}
            \end{class}
            \begin{class}{Grouping}{16, -2}
                \attribute {id : String}
                \attribute {roomId : String}
                \attribute {fundId : String}
                \attribute {projectId : String}
            \end{class}
            \association {Room}{}{1..N}{Grouping}{0..1}{}
            \association {Grouping}{}{0..N}{Fund}{0..N}{}
            \association {Project}{}{1..1}{Grouping}{1..1}{}
        \end{tikzpicture}
        }
        \caption{Resource Allocation Problem.}
        \label{fig:uml_resorce_allocation_problem}
    \end{subfigure}
    \hfill
    \begin{subfigure}[b]{0.45\textwidth}
        \centering
        \scalebox{0.6}{%
        \begin{tikzpicture}[scale=0.4]
            \begin{class}{User}{0, 6}
                \attribute {id : String}
            \end{class}
            \begin{class}{Item}{15, 6}
                \attribute {id : String}
            \end{class}

            \begin{class}{Grouping}{7.5, -2}
                \attribute {id : String}
                \attribute {UserId : String}
                \attribute {ItemrId : String}
            \end{class}
            \association {User}{}{1..1}{Grouping}{1..1}{}
            \association {Grouping}{}{1..1}{Item}{1..N}{}
        \end{tikzpicture}
        }
        \caption{User-Item recommendation Problem.}
        \label{fig:uml_recomendation_problem}
    \end{subfigure}
    \hfill
    \begin{subfigure}[b]{0.45\textwidth}
        \centering
        \scalebox{0.6}{%
        \begin{tikzpicture}[scale=0.4]
            \begin{class}{Image}{0, 6}
                \attribute {id : String}
            \end{class}

            \begin{class}{Grouping}{0, -2}
                \attribute {id : String}
                \attribute {image1Id : String}
                \attribute {image2Id : String}
            \end{class}
            \association {Image}{}{2..2}{Grouping}{0..1}{}
        \end{tikzpicture}
        }
        \caption{Image Matching Problem.}
        \label{fig:uml_image_matching_problem}
    \end{subfigure}
    
    \caption{Visualizing Grouping Problems as a relational diagram.}
    \label{fig:examples_grouping_as_relations}
\end{figure}     
             
            It is important to also properly define the geometry (dimension) of the relations, as they can represent different problems, as shown in Figure \ref{fig:relation_geometry}.

             \begin{figure}[!ht]
    \centering
    \begin{subfigure}[b]{0.45\textwidth}
        \centering
        \scalebox{0.6}{%
        \begin{tikzpicture}[scale=0.4]
            \begin{class}{User}{0, 6}
                \attribute {id : String}
            \end{class}
            \begin{class}{Item}{15, 6}
                \attribute {id : String}
            \end{class}

            \begin{class}{Grouping}{7.5, -2}
                \attribute {id : String}
                \attribute {UserId : String}
                \attribute {ItemId : String}
            \end{class}
            \association {User}{}{1:1}{Grouping}{0:1}{}
            \association {Grouping}{}{0:1}{Item}{1:1}{}
        \end{tikzpicture}
        }
        \caption{An user has to be assigned to a single item.}
        \label{fig:1_!}
    \end{subfigure}
    \hfill
    \begin{subfigure}[b]{0.45\textwidth}
        \centering
        \scalebox{0.6}{%
        \begin{tikzpicture}[scale=0.4]
            \begin{class}{User}{0, 6}
                \attribute {id : String}
            \end{class}
            \begin{class}{Item}{15, 6}
                \attribute {id : String}
            \end{class}

            \begin{class}{Grouping}{7.5, -2}
                \attribute {id : String}
                \attribute {UserId : String}
                \attribute {ItemId : String}
            \end{class}
            \association {User}{}{1:1}{Grouping}{0:1}{}
            \association {Grouping}{}{0:1}{Item}{1:N}{}
        \end{tikzpicture}
        }
        \caption{An user can be assigned to multiple items.}
        \label{fig:1_N}
    \end{subfigure}
    
    \caption[The User-Item Recommendation Problem.]{The User-Item Recommendation Problem showcases how different dimensions on the relations result in different restrictions.
            In both examples a grouping must have an user, but not all users have an assignment.}
    \label{fig:relation_geometry}
\end{figure}            

            
            As we discussed before, sets can be collections of instances, which do not necessarily belong to the same class, which represent relationships among these instances. Using this parallel, it is possible to see a Set Cover Problem and its variants as a relational diagram too, for instance, Figure \ref{fig:3_classes} is the relational diagram version of the set cover problem shown at Figure \ref{fig:set_coverage_framework_1}, while Figure \ref{fig:1_class} is the relational diagram version of the set cover problem shown at Figure \ref{fig:set_coverage_framework_2}.

            \begin{figure}[!ht]
    \centering
    \resizebox{0.9\textwidth}{!}{%
    \begin{tikzpicture}[scale=0.4]
        \begin{class}{Class A}{0, 6}
            \attribute {id : String}
            
        \end{class}
        \begin{class}{Class B}{15, 6}
            \attribute {id : String}
            
        \end{class}
        \begin{class}{Class C}{30, 6}
            \attribute {id : String}
            
        \end{class}
        \begin{class}{Grouping}{15, -2}
            \attribute {id : String}
            \attribute {classAId : String}
            \attribute {classBId : String}
            \attribute {classCId : String}
        \end{class}
        \association {Class A}{}{1..1}{Grouping}{0..1}{}
        \association {Grouping}{}{0..1}{Class B}{1..1}{}
        \association {Class C}{}{1..1}{Grouping}{0..1}{}
    \end{tikzpicture}
    }
    \caption{Representation of the relation between 3 differents Classes.}
    \label{fig:3_classes}
\end{figure}

\begin{figure}[!ht]
    \centering
    \resizebox{0.279\textwidth}{!}{%
    \begin{tikzpicture}[scale=0.4]
        \begin{class}{Class A}{0, 15}
            \attribute {id : String}
                        
        \end{class}
        \begin{class}{Grouping}{0, 6}
            \attribute {id : String}
            \attribute {ids : Set<String>}
        \end{class}
        \association {Class A}{}{1..N}{Grouping}{0..N}{}
    \end{tikzpicture}
    }
    \caption{Representation of the relation between elements of a same Class.}
    \label{fig:1_class}
\end{figure}

            To summarize, the SCF is the way to represent the grouping problems as relation between objects, which is also the same way one can represent a set cover problem. It permits the representation of every decidable grouping problem \cite{garey1979computers} and also makes it more practical to understand and use it, since the object relation paradigm is already well established in the industrial environment.

            The solution for a grouping problem in the SCF can be represented as a list of grouping objects.
                

        \subsection{Statistics}
            Depending on the problem, each grouping can encapsulate various statistics such as costs, capacities, number of elements, qualities, and validity, among other characteristics relevant to the grouping problem. At the SCF  the statistics are attributes of the \textit{Grouping} class.
            
            It is crucial for these statistics to be well-defined beforehand for a specific problem. Depending on the nature of the problem, these statistics can either be computed on-demand or predefined, depending upon whether preferences are implicit or explicit. Figure \ref{fig:statistics} shows how this statistics can be represented.

            \begin{figure}[!ht]
    \centering
    \begin{subfigure}[b]{0.45\textwidth}
        \centering
        \scalebox{0.6}{%
        \begin{tikzpicture}[scale=0.4]
            \begin{class}{Person}{0, 6}
                \attribute {id : String}
            \end{class}
            \begin{class}{Person}{15, 6}
                \attribute {id : String}
            \end{class}

            \begin{class}{Grouping}{7.5, -2}
                \attribute {id : String}
                \attribute {Person1Id : String}
                \attribute {Person2Id : String}
                \attribute {\textbf{Preference : int}}
            \end{class}
            \association {Worker}{}{0:1}{Grouping}{1:1}{}
            \association {Grouping}{}{0:1}{Job}{1:1}{}
        \end{tikzpicture}
        }
        \caption{Stable Marriage Problem.}
        \label{fig:stat_stable}
    \end{subfigure}
    \hfill
    \begin{subfigure}[b]{0.45\textwidth}
        \centering
        \scalebox{0.6}{%
        \begin{tikzpicture}[scale=0.4]
            \begin{class}{Worker}{0, 6}
                \attribute {id : String}
            \end{class}
            \begin{class}{Job}{15, 6}
                \attribute {id : String}
            \end{class}

            \begin{class}{Grouping}{7.5, -2}
                \attribute {id : String}
                \attribute {JobId : String}
                \attribute {WorkerId : String}
                \operation {\textbf{Cost(Job, Worker) : double}}
            \end{class}
            \association {Worker}{}{0:1}{Grouping}{1:1}{}
            \association {Grouping}{}{1:1}{Job}{1:1}{}
        \end{tikzpicture}
        }
        \caption{Job Assignment Problem.}
        \label{fig:stat_assignment}
    \end{subfigure}
    
    \caption[Implementing Statistics at the Grouping Class.]{Implementing Statistics at the Grouping Class.\\
            The statistics in each grouping class are bolded.}
    \label{fig:statistics}
\end{figure}
    
            At Figure \ref{fig:stat_stable}, showing the Stable Marriage Problem, a grouping represents potential relationships between pairs, where the statistics of these groupings reflect preferences. In the context of the Job Assignment Problem, at Figure \ref{fig:stat_assignment}, a grouping represents potential assignments between workers and jobs, with the statistics indicating the costs associated with each assignment.

        \subsection{Objective Function} \label{sec:obj_func}

            In optimization problems, the objective function is a mathematical expression that quantifies what needs to be optimized. The goal of an optimization problem is to find the values (variables) that either maximize (fitness function) or minimize (cost function) the result of this function, subject to certain constraints.
            
            Formally, if \( x \) represents a vector of decision variables, the objective function \( f(x) \) is a function mapping the decision variables \( x \) to a real number. The optimization process involves finding the values of \( x \) that either maximize or minimize \( f(x) \), depending on the problem.
            
            In the case of this work, the objective function is computed based on the statistics of every grouping used in a possible solution for a given problem.
            Lets assume a possible solution is a list of groupings $L$. An objective function could be like the Algorithm \ref{algo:example_obj_func}.

            \begin{algorithm}[!ht]
    \caption{Example of Objective Function - Maximize the Return} \label{algo:example_obj_func}
    \begin{algorithmic}[1]
        \STATE $Sum \Leftarrow 0$
        \FOR{each $Grouping$ in $L$}
            \STATE $Sum \Leftarrow Sum$ + value of $Grouping$
        \ENDFOR
        \RETURN $Sum$
    \end{algorithmic}
\end{algorithm}

            Algorithms \ref{algo:job_assign_obj_func} and \ref{algo:stbl_marriage_obj_func} show a possible objective function for the job assignment and stable marriage problems, respectively. They have very different behavior, while Algorithm \ref{algo:job_assign_obj_func} focus on finding low cost assignments and wants to minimizes the function, Algorithm \ref{algo:stbl_marriage_obj_func} focus to find the biggest amount of stable matchings through maximizing the function.

            \begin{algorithm}[!ht]
    \caption{Job Assignment Objective Function - Minimize the Return} \label{algo:job_assign_obj_func}
    \begin{algorithmic}[1]
        \IF{amount of $Assignments$ $\neq$ amount of $Jobs$}
            \RETURN $\infty$
        \ENDIF
        \STATE $Sum \Leftarrow 0$
        \FOR{each $Assignment$ in $Assignments$}
            \IF{amount of $Workers$ in $Assignment$ $\neq$ 1 \OR amount of $Jobs$ in $Assignment$ $\neq$ 1}
                \RETURN $\infty$
            \ENDIF
            \STATE{$Sum \Leftarrow Sum$ + cost of $Assignment$}
        \ENDFOR
        \RETURN $Sum$
    \end{algorithmic}
\end{algorithm}
            \begin{algorithm}
    \caption{Stable Mariage Objective Function - Maximize the Return} \label{algo:stbl_marriage_obj_func}
    \begin{algorithmic}[1]
        \FOR{each $Marriage$ in $Marriages$}
            \IF{amount of $People$ in $Marriage$ $\neq$ 2}
                \RETURN $0$
            \ENDIF
        \ENDFOR

        \FOR{each $Person$ in $Marriages$}
            \IF{$Person$ appears more than 2 times}
                \RETURN $0$
            \ENDIF
        \ENDFOR

        \FOR{each $Person_1$ in $Marriages$}
             \STATE{$PairPerson_1 \Leftarrow$ pair of $Person_1$}
             \FOR{each $Person_2$ in $Marriages$}
                  \STATE{$PairPerson_2 \Leftarrow$ pair of $Person_2$}
                % \IF{$Person_2$ $\neq$ $Person_1$ \AND $Person_2$ $\neq$ $PairPerson_1$}
                    \IF{$Person_1$ prefers $Person_2$ over $PairPerson_1$ \AND
                    $Person_2$ prefers $Person_1$ over $PairPerson_2$}
                        \RETURN $0$
                    \ENDIF
                % \ENDIF
            \ENDFOR
        \ENDFOR
        
        \RETURN amount of $Marriages$
    \end{algorithmic}
\end{algorithm}

            Note in Algorithm \ref{algo:job_assign_obj_func} the return of infinity in lines 2 and 7 enforce hard constraints, respectively, all jobs must be assigned (lines 1-3) and each assignment must have a single worker and single job (lines 6-8). Soft constraints could be implemented by increasing the value of sum with a penalty values instead of returning infinity.

            Similarly, Algorithm \ref{algo:stbl_marriage_obj_func} enforces hard constraints by returning zero (maximum function) in lines 3, 8 and 16.
    
        \subsection{Problem Specification and Configurations}
            As a lot of grouping problems have similar characteristics as shown in Chapter \ref{chap:grouping_problems}. Due to this observation, it is possible to create template statistics with a default behavior. Such as:
            \begin{itemize}
                \item \textbf{Cost:} it defines the cost to make a grouping. Usually the objective is to minimize the sum of the costs, while having the bigger amount of grouping (priority one).
                \item \textbf{Preference:} it is based on preferences of objects. Have the greater amount of groupings while respecting preferences of objects, where none objet wants to switch to a different grouping.
                \item \textbf{Usage Limit:} it defines how many different groupings an element can be part of.
                \item \textbf{Ranking:} it defines how similar objects are, maximize the average ranking.

            \end{itemize}
        
    \section{Pipeline}
    Translating grouping problems to the SCF (Set-Cover Framework) is not sufficient; it is also crucial to solve the underlying optimization problems. 
    However, each grouping problem presents its own characteristics, which in turn can be better tackled with a particular solver.
    Furthermore, for different solvers, different internal representations of the problem are required.
    To address these problems, a specialized pipeline is employed. 
    Figure \ref{fig:pipeline} illustrates a pipeline that demonstrates how a grouping problem, represented using the SCF framework, can be assigned to a specialist solver and then returned to the same SCF representation.
    First the SFC is sent to the assigner, that will choose a solver based on the statistics of the problem, each solver is an specialized algorithm with a converter to convert the SCF to the expected input format to the algorithm, after that a normalizer will get the output of the algorithm and translate it again to the SCF, outputting a list of groupings objects. 

    \begin{figure}[!ht]
    \centering

     \scalebox{0.75}{%
        \begin{tikzpicture}[main/.style = {draw, rectangle},  invisible/.style = {circle}, cloud2/.style = {draw, cloud}]
            \tikzset{edge/.style = {->,> = latex'}}
            \node[invisible, text width=3cm, align=center] (input) {Groupings\\ or \\ Instances of the Classes};
            
            \node[main] (assigner) [right of=input, xshift=2cm] {$Assigner$};
            
            \node[invisible] (ret) [right of=assigner,  xshift=4cm] {\vdots};
            
            \node[main] (solver_2) [above of=ret] {$Solver_2$};
            
            \node[main] (solver_1) [above of=solver_2] {$Solver_1$};
            
            \node[main] (solver_N) [below of=ret] {$Solver_N$};
            
            \node[main] (meta) [below of=solver_N] {$Solver_{Metaheuristic}$};
        
            \node[invisible] (output) [right of=ret, xshift=4cm] {List of Tuples};
    
            \draw [edge] (input) to [out=0,in=180,looseness=1.5] (assigner);
            \draw [edge] (assigner) to [out=0,in=180,looseness=1.5] (solver_1) ;
            \draw [edge] (assigner) to [out=0,in=180,looseness=1.5] (solver_2) ;
            \draw [edge] (assigner) to [out=0,in=180,looseness=1.5] (solver_N) ;
            \draw [edge] (assigner) to [out=0,in=180,looseness=1.5] (meta) ;

            \draw [edge] (solver_1) to [out=0,in=180,looseness=1.5] (output) ;
            \draw [edge] (solver_2) to [out=0,in=180,looseness=1.5] (output) ;
            \draw [edge] (solver_N) to [out=0,in=180,looseness=1.5] (output) ;
            \draw [edge] (meta) to [out=0,in=180,looseness=1.5] (output) ;
    
        \end{tikzpicture}
    }%
    \caption{A pipeline to solve Grouping Problems.}
    \label{fig:pipeline}
\end{figure}

    \subsection{Assigner}
        Since the problems are represented using the SCF, one might assume that only NP-hard solutions are available. However, due to the information stored in the statistics, it is sometimes possible to reduce the problem to a simpler one.

        The assigner gathers information from the statistics and metadata from the grouping to identify a specific and specialized solver for that particular grouping problem, if no one is available, a metaheuristic one is assigned.

        Figure \ref{fig:flow_chart} shows a simplified version of the assigner; a more advanced assigner can be implemented in future work.
        %% give breath explanation of Preference, Cost, Usage Limit and Ranking
        For this example, the assigner analyzes the size of the grouping, if the relations have preferences (i.e. each element has a preference for which element to be grouped with), if the relations have costs (i.e. each element has a cost to be grouped with another element), and if there is a ranking (i.e. each element has a ranking with respect to the other elements). Based on these characteristics, it selects the appropriate solver.
        \definecolor{adaee4ed-88c8-5b21-a9e7-31316ebef86f}{RGB}{255, 255, 255}
\definecolor{f3551e38-74df-57e2-b793-83d7fe876c85}{RGB}{0, 0, 0}
\definecolor{0b71a967-1f15-55a5-9bb9-70efa7b4fc58}{RGB}{51, 51, 51}
\definecolor{70af333d-4869-5f2a-97ca-9904a9fce6c3}{RGB}{255, 255, 255}
\definecolor{5856d031-3da1-575c-834e-c77e9e438c62}{RGB}{0, 0, 0}

\tikzstyle{c5610da3-5598-50aa-b141-78500324a30c} = [rectangle, rounded corners, text width=4cm, minimum height=1cm, text centered, font=\normalsize, color=0b71a967-1f15-55a5-9bb9-70efa7b4fc58, draw=f3551e38-74df-57e2-b793-83d7fe876c85, line width=1, fill=adaee4ed-88c8-5b21-a9e7-31316ebef86f]

\tikzstyle{62e98c06-79c2-5a04-aeb6-076755d5465e} = [diamond, text width=2cm, text centered, font=\normalsize, color=0b71a967-1f15-55a5-9bb9-70efa7b4fc58, draw=f3551e38-74df-57e2-b793-83d7fe876c85, line width=1, fill=70af333d-4869-5f2a-97ca-9904a9fce6c3]

\tikzstyle{7be24b85-97d0-5b76-ba9e-d94005dca8f2} = [thick, draw=5856d031-3da1-575c-834e-c77e9e438c62, line width=2, ->, >=stealth]


\begin{figure}[!ht]
    \centering
    \scalebox{0.6}{%
        \begin{tikzpicture}[node distance=2.5cm]
            \centering
            \node (e7a1f4bb-66b5-433e-b798-2d76642f9da8) [c5610da3-5598-50aa-b141-78500324a30c] {Start};
            \node (1c3eea52-adcd-41c9-a8ed-1bd194dab4e8) [62e98c06-79c2-5a04-aeb6-076755d5465e, below of=e7a1f4bb-66b5-433e-b798-2d76642f9da8, yshift=-0.5cm] {2 Elements at each Grouping?};
            
            \node (b9080e40-57f0-43b3-b78f-b40298fe0b83) [62e98c06-79c2-5a04-aeb6-076755d5465e, below right of=1c3eea52-adcd-41c9-a8ed-1bd194dab4e8, yshift=-0.5cm, xshift=2cm] {Are there Preferences?};
            
            \node (7f3b2db4-0741-4d32-aa43-84d42604d957) [62e98c06-79c2-5a04-aeb6-076755d5465e, below left of=b9080e40-57f0-43b3-b78f-b40298fe0b83, yshift=-0.5cm, xshift=-2cm] {Are there Ranks?};
            
            \node (8fc44ab5-5e5e-4354-a909-806d2f3e494a) [62e98c06-79c2-5a04-aeb6-076755d5465e, right of=7f3b2db4-0741-4d32-aa43-84d42604d957, xshift=8cm] {Are there Costs?};
            
            \node (d8a39a87-1b0e-4d35-9e1d-eda5964f6268) [c5610da3-5598-50aa-b141-78500324a30c, below left of=7f3b2db4-0741-4d32-aa43-84d42604d957, yshift=-0.5cm, xshift=-1cm] {Hungarian Algorithm};
            
            \node (bee77f50-dddf-46a5-9dca-ce03d4871c4d) [c5610da3-5598-50aa-b141-78500324a30c, right of=d8a39a87-1b0e-4d35-9e1d-eda5964f6268, xshift=3cm] {Binary Search + Hungarian Algorithm};
            
            \node (233a4f92-7e74-4f97-9a1b-b4d8842e0cce) [c5610da3-5598-50aa-b141-78500324a30c, right of=bee77f50-dddf-46a5-9dca-ce03d4871c4d, xshift=2cm] {Stable Marriage Algorithm};
            
            \node (2b5fc8e9-0a88-4abb-9e5b-aaa676180d01) [c5610da3-5598-50aa-b141-78500324a30c, below left of=d8a39a87-1b0e-4d35-9e1d-eda5964f6268, xshift=-1cm] {Metaheuristic Solver};
        
        
        
        
            
            \draw [7be24b85-97d0-5b76-ba9e-d94005dca8f2] (e7a1f4bb-66b5-433e-b798-2d76642f9da8) --  (1c3eea52-adcd-41c9-a8ed-1bd194dab4e8);
            \draw [7be24b85-97d0-5b76-ba9e-d94005dca8f2] (1c3eea52-adcd-41c9-a8ed-1bd194dab4e8) -| node[anchor=west] {Yes} (b9080e40-57f0-43b3-b78f-b40298fe0b83);
            \draw [7be24b85-97d0-5b76-ba9e-d94005dca8f2] (b9080e40-57f0-43b3-b78f-b40298fe0b83) -| node[anchor=west] {Yes} (8fc44ab5-5e5e-4354-a909-806d2f3e494a);
            \draw [7be24b85-97d0-5b76-ba9e-d94005dca8f2] (b9080e40-57f0-43b3-b78f-b40298fe0b83) -| node[anchor=east] {No} (7f3b2db4-0741-4d32-aa43-84d42604d957);
            \draw [7be24b85-97d0-5b76-ba9e-d94005dca8f2] (1c3eea52-adcd-41c9-a8ed-1bd194dab4e8) -| node[anchor=east] {No} (2b5fc8e9-0a88-4abb-9e5b-aaa676180d01);
            \draw [7be24b85-97d0-5b76-ba9e-d94005dca8f2] (7f3b2db4-0741-4d32-aa43-84d42604d957) -| node[anchor=east] {No} (d8a39a87-1b0e-4d35-9e1d-eda5964f6268);
            \draw [7be24b85-97d0-5b76-ba9e-d94005dca8f2] (7f3b2db4-0741-4d32-aa43-84d42604d957) -| node[anchor=west] {Yes} (bee77f50-dddf-46a5-9dca-ce03d4871c4d);
            \draw [7be24b85-97d0-5b76-ba9e-d94005dca8f2] (8fc44ab5-5e5e-4354-a909-806d2f3e494a.east) |- node[anchor=west] {Yes} (2b5fc8e9-0a88-4abb-9e5b-aaa676180d01);
            \draw [7be24b85-97d0-5b76-ba9e-d94005dca8f2] (8fc44ab5-5e5e-4354-a909-806d2f3e494a) -| node[anchor=east] {No} (233a4f92-7e74-4f97-9a1b-b4d8842e0cce);
        \end{tikzpicture}
    }
    \caption[Example of a possible flowchart for the Assigner.]{Example of a possible flowchart for the Assigner. The solvers shown here are explained at Section \ref{sec:solvers}.}
    \label{fig:flow_chart}
\end{figure}

    \subsection{Solvers} \label{sec:solvers}
    
        According to the problem identified by the assigner, it is important to have algorithm-specific solvers. In cases where there is no predefined specification for the problem, a metaheuristic algorithm based on genetic algorithms will be used to solve the problem.

        Algorithm-specific solvers are tailored to solve particular types of problems more efficiently than general-purpose solvers. One such solver is the Hungarian algorithm, which is used for finding the optimal assignment in a bipartite graph \cite{kuhn1955hungarian}. This algorithm ensures that the total cost of the assignment is minimized, making it particularly effective for solving assignment problems where each element in one set must be matched with an element in another set at minimal cost.

        For cases where the objective is to minimize the maximum difference among the matchings, a combination of binary search and the Hungarian algorithm is employed \cite{land1974automatic}. The binary search is used to find the optimal threshold, and for each threshold value, the Hungarian algorithm determines if a valid matching exists within that threshold. This combined approach allows for efficient minimization of the maximum difference among the matchings.

        Another important algorithm-specific solver is the Stable Matching Algorithm, also known as the Stable Marriage algorithm \cite{gale1962college}. This algorithm is used for finding stable matchings, where there are no two elements that would both prefer each other over their current matches. The Stable Marriage algorithm is widely applied in scenarios such as matching students to schools or residents to hospitals, ensuring that the matchings are stable and no participants have an incentive to deviate from their assigned matches.
            
       \subsection{Metaheuristic Solver: Genetic Algorithm}
            In situations where no specialized solver is available, a metaheuristic approach, such as a genetic algorithm, is used. This method attempts to find a near-optimal solution by mimicking the process of natural selection, iteratively improving the solution through processes such as selection, crossover, and mutation \cite{holland1992adaptation}. Section \ref{sec:ga} gives a better understanding on how the genetic algorithm works.

        \subsubsection{Gene and Chromosome Representation} \label{sec:gene}
            The design allows the user to enter two types of input, the allowed Grouping Objects, or all instances of classes that can be part of the groupings.
            %%
            For the first case, the chromosome is a bit mask $b$ of size $n$, where $n$ is the number of allowed Grouping Objects. The $i$-th bit in $b$ indicates whether or not the $i$-th grouping object is selected at that solution option. Figure \ref{fig:genome_1} shows an example of it.
            %%
            For the second case, the chromosome is a list of $n$ bit masks (a matrix). 
            Therefore, we have a predefined maximum of $n$ grouping.
            % Each group has its own bit mask, and each mask will be of size $m$, where $m$ is the number of instances available. 
            Each bit mask, of size $m$, represents a grouping, where
            $m$ is the number of instances available. 
            %In this case, we assume we have a fixed number of $n$ groupings, and each 
            The $i$-th bit of the $j$-th bit mask indicates whether the $i$-th instance is part of the $j$-th grouping. 
            Therefore, the chromosome is a matrix of size $n\times m$. 
            This can generate invalid groupings, so they need to be checked after mutations and cross-overs. Figure \ref{fig:genome_2} shows an example of it.
            \begin{figure}[!ht]
        \centering
        \begin{subfigure}{0.45\textwidth}
            \centering
            \resizebox{\textwidth}{!}{%
            \begin{circuitikz}
                \tikzstyle{every node}=[font=\LARGE]
                
                \draw [, line width=1pt ] (9,13.25) circle (0.75cm) node {$x_1$};
                \draw [, line width=1pt ] (20.25,15.25) circle (0.75cm) node {$x_2$};
                \draw [, line width=1pt ] (11.5,5.75) circle (0.75cm) node {$x_3$};
                \draw [, line width=1pt ] (15.25,12.25) circle (0.75cm) node {$x_4$};
                \draw [, line width=1pt ] (21.25,10.25) circle (0.75cm) node {$x_5$};
                \draw [, line width=1pt ] (16.5,7) circle (0.75cm) node {$x_6$};
                \draw [, line width=1pt ] (8,9.75) circle (0.75cm) node {$x_7$};
                
                \draw [ color={rgb,255:red,0; green,85; blue,255}, line width=1pt, dashed, ultra thick] (7.75,14.25) -- (7.5,12.25);
                \draw [ color={rgb,255:red,0; green,85; blue,255}, line width=1pt, dashed, ultra thick] (17,5) .. controls (20,7.75) and (20.25,7.75) .. (23.25,10.5);
                \draw [ color={rgb,255:red,0; green,85; blue,255}, line width=1pt, dashed, ultra thick] (23.25,10.5) .. controls (22,11.25) and (22.25,11.25) .. (21.25,12);
                \draw [ color={rgb,255:red,0; green,85; blue,255}, line width=1pt, dashed, ultra thick] (21.25,12) .. controls (18,10.25) and (18.25,10.25) .. (15,8.75);
                \draw [ color={rgb,255:red,0; green,85; blue,255}, line width=1pt, dashed, ultra thick] (15,8.75) .. controls (12,11.75) and (12.25,11.75) .. (9.25,15);
                \draw [ color={rgb,255:red,0; green,85; blue,255}, line width=1pt, dashed, ultra thick] (9.25,15) .. controls (8.25,14.5) and (8.5,14.5) .. (7.75,14.25);
                \draw [ color={rgb,255:red,0; green,85; blue,255}, line width=1pt, dashed, ultra thick] (7.5,12.25) -- (10.25,8.25);
                \draw [ color={rgb,255:red,0; green,85; blue,255}, line width=1pt, dashed, ultra thick] (10.25,8.25) -- (10.25,3);
                \draw [ color={rgb,255:red,0; green,85; blue,255}, line width=1pt, dashed, ultra thick] (10.25,3) -- (17,5);
                
                \draw [ color={rgb,255:red,255; green,0; blue,0}, line width=1pt, loosely dotted, ultra thick] (12.25,4.75) .. controls (14.5,8.25) and (14.75,8.25) .. (17,12);
                \draw [ color={rgb,255:red,255; green,0; blue,0}, line width=1pt, loosely dotted, ultra thick] (17,12) .. controls (19.75,13.25) and (20,13.25) .. (22.75,14.5);
                \draw [ color={rgb,255:red,255; green,0; blue,0}, line width=1pt, loosely dotted, ultra thick] (22.75,14.5) .. controls (21.5,15.75) and (21.75,15.75) .. (20.75,17);
                \draw [ color={rgb,255:red,255; green,0; blue,0}, line width=1pt, loosely dotted, ultra thick] (20.75,17) .. controls (16.75,14.5) and (17,14.5) .. (13.25,12.25);
                \draw [ color={rgb,255:red,255; green,0; blue,0}, line width=1pt, loosely dotted, ultra thick] (13.25,12.25) .. controls (11.25,8.25) and (11.5,8.25) .. (9.5,4.5);
                \draw [ color={rgb,255:red,255; green,0; blue,0}, line width=1pt, loosely dotted, ultra thick] (9.5,4.5) .. controls (10.5,4.25) and (10.75,4.25) .. (12,4);
                \draw [ color={rgb,255:red,255; green,0; blue,0}, line width=1pt, loosely dotted, ultra thick] (12,4) .. controls (12,4.25) and (12.25,4.25) .. (12.25,4.75);
                
                \draw [ color={rgb,255:red,0; green,143; blue,2}, line width=1pt, loosely dashdotted, ultra thick] (6.75,7.75) .. controls (6,9.25) and (6.25,9.25) .. (5.75,10.75);
                \draw [ color={rgb,255:red,0; green,143; blue,2}, line width=1pt, loosely dashdotted, ultra thick] (5.75,10.75) .. controls (13.75,14.5) and (14,14.5) .. (22.25,18.5);
                \draw [ color={rgb,255:red,0; green,143; blue,2}, line width=1pt, loosely dashdotted, ultra thick] (22.25,18.5) .. controls (23.25,13.5) and (23.5,13.5) .. (24.5,8.5);
                \draw [ color={rgb,255:red,0; green,143; blue,2}, line width=1pt, loosely dashdotted, ultra thick] (24.5,8.5) .. controls (23.25,7.75) and (23.5,7.75) .. (22.5,7);
                \draw [ color={rgb,255:red,0; green,143; blue,2}, line width=1pt, loosely dashdotted, ultra thick] (6.75,7.75) .. controls (12.75,9.75) and (13,9.75) .. (19,12);
                \draw [ color={rgb,255:red,0; green,143; blue,2}, line width=2pt, loosely dashdotted, ultra thick] (19,12) .. controls (20.5,9.5) and (20.75,9.5) .. (22.5,7);

            \end{circuitikz}
            }%
            \caption{} 
            \label{fig:set_cover_genome_1}
        \end{subfigure}
        \hfill
        \begin{subfigure}{0.45\textwidth}
            \centering
            \resizebox{0.38\textwidth}{!}{%
                \begin{circuitikz}
                    \tikzstyle{every node}=[font=\LARGE]
                    
                    % Linha 1 (000)
                    \draw  (5.8,11.3) rectangle (6.8,10.3) node[pos=.5] {$0$};
                    \draw  (6.8,11.3) rectangle (7.8,10.3) node[pos=.5] {$0$};
                    \draw  (7.8,11.3) rectangle (8.8,10.3) node[pos=.5] {$0$};
                    \node[right] at (8.8,10.8) { = \{$\emptyset$\}};
        
                    % Linha 2 (001)
                    \draw  (5.8,9.2) rectangle (6.8,8.2) node[pos=.5] {$0$};
                    \draw  (6.8,9.2) rectangle (7.8,8.2) node[pos=.5] {$0$};
                    \draw  (7.8,9.2) rectangle (8.8,8.2) node[pos=.5] {$1$};
                    \node[right] at (8.8,8.7) { = \{$s_3$\}};
        
                    % Linha 3 (010)
                    \draw  (5.8,7.2) rectangle (6.8,6.2) node[pos=.5] {$0$};
                    \draw  (6.8,7.2) rectangle (7.8,6.2) node[pos=.5] {$1$};
                    \draw  (7.8,7.2) rectangle (8.8,6.2) node[pos=.5] {$0$};
                    \node[right] at (8.8,6.7) { = \{$s_2$\}};
        
                    % Linha 4 (011)
                    \draw  (5.8,5.2) rectangle (6.8,4.2) node[pos=.5] {$0$};
                    \draw  (6.8,5.2) rectangle (7.8,4.2) node[pos=.5] {$1$};
                    \draw  (7.8,5.2) rectangle (8.8,4.2) node[pos=.5] {$1$};
                    \node[right] at (8.8,4.7) { = \{$s_2, s_3$\}};
        
                    % Linha 5 (100)
                    \draw  (5.8,3.2) rectangle (6.8,2.2) node[pos=.5] {$1$};
                    \draw  (6.8,3.2) rectangle (7.8,2.2) node[pos=.5] {$0$};
                    \draw  (7.8,3.2) rectangle (8.8,2.2) node[pos=.5] {$0$};
                    \node[right] at (8.8,2.7) { = \{$s_1$\}};
        
                    % Linha 6 (101)
                    \draw  (5.8,1.2) rectangle (6.8,0.2) node[pos=.5] {$1$};
                    \draw  (6.8,1.2) rectangle (7.8,0.2) node[pos=.5] {$0$};
                    \draw  (7.8,1.2) rectangle (8.8,0.2) node[pos=.5] {$1$};
                    \node[right] at (8.8,0.7) { = \{$s_1, s_3$\}};
        
                    % Linha 7 (110)
                    \draw  (5.8,-1.2) rectangle (6.8,-2.2) node[pos=.5] {$1$};
                    \draw  (6.8,-1.2) rectangle (7.8,-2.2) node[pos=.5] {$1$};
                    \draw  (7.8,-1.2) rectangle (8.8,-2.2) node[pos=.5] {$0$};
                    \node[right] at (8.8,-1.7) { = \{$s_1, s_2$\}};
        
                    % Linha 8 (111)
                    \draw  (5.8,-3.2) rectangle (6.8,-4.2) node[pos=.5] {$1$};
                    \draw  (6.8,-3.2) rectangle (7.8,-4.2) node[pos=.5] {$1$};
                    \draw  (7.8,-3.2) rectangle (8.8,-4.2) node[pos=.5] {$1$};
                    \node[right] at (8.8,-3.7) { = \{$s_1, s_2, s_3$\}};
        
                \end{circuitikz}
            }%
            \caption{} 
            \label{fig:genome_1_example}
        \end{subfigure}
        \caption[Example of a chromosome 1]{Example of a chromosome from a problem where the allowed Grouping Objects are predefined. (a) Allowed Grouping Objects are $s_1$, $s_2$, and $s_3$. (b) Chromosome binary representation. Each line is a different chromosome that represents the selected groups.} 
        \label{fig:genome_1}
    \end{figure}
            \begin{figure}[!ht]
        \centering
        \begin{subfigure}{0.35\textwidth}
            \centering
            \resizebox{0.55\textwidth}{!}{%
            \begin{circuitikz}
                \tikzstyle{every node}=[font=\LARGE]
                
                \draw [, line width=1pt ] (10.5,12.25) circle (0.75cm) node {$x_1$};
                \draw [, line width=1pt ] (12.75,11.25) circle (0.75cm) node {$x_2$};
                \draw [, line width=1pt ] (9,10.25) circle (0.75cm) node {$x_3$};
                \draw [, line width=1pt ] (13.25,9.25) circle (0.75cm) node {$x_4$};
                \draw [, line width=1pt ] (10.5,7.75) circle (0.75cm) node {$x_5$};
                
            \end{circuitikz}
            }%
            \caption{} 
            \label{fig:instances_genome_2}
        \end{subfigure}
        \hfill
        \begin{subfigure}{0.6\textwidth}
            \centering
            \resizebox{0.8\textwidth}{!}{%
                \begin{circuitikz}
                    \tikzstyle{every node}=[font=\LARGE]
                    
                    % Linha 1 (000)
                    
                    \node[left] at (5.8,10.8) {$g_1 \rightarrow$};
                    
                    \node[above] at (6.3,12.2) {$x_1$};
                    \node[above] at (6.3,11.4) {$\downarrow$};
                    \node[above] at (7.3,12.2) {$x_2$};
                    \node[above] at (7.3,11.4) {$\downarrow$};
                    \node[above] at (8.3,12.2) {$x_3$};
                    \node[above] at (8.3,11.4) {$\downarrow$};
                    \node[above] at (9.3,12.2) {$x_4$};
                    \node[above] at (9.3,11.4) {$\downarrow$};
                    \node[above] at (10.3,12.2) {$x_5$};
                    \node[above] at (10.3,11.4) {$\downarrow$};
                    
                    \draw  (5.8,11.3) rectangle (6.8,10.3) node[pos=.5] {$1$};
                    \draw  (6.8,11.3) rectangle (7.8,10.3) node[pos=.5] {$1$};
                    \draw  (7.8,11.3) rectangle (8.8,10.3) node[pos=.5] {$0$};
                    \draw  (8.8,11.3) rectangle (9.8,10.3) node[pos=.5] {$0$};
                    \draw  (9.8,11.3) rectangle (10.8,10.3) node[pos=.5] {$1$};
                    \node[right] at (10.8,10.8) {\vspace{100 pt} grouping $g_1$ is \{$x_1, x_2, x_5$\}};
        
                    % Linha 2 (001)
                    \node[left] at (5.8,9.8) {$g_2 \rightarrow$};
                    \draw  (5.8,10.3) rectangle (6.8,9.3) node[pos=.5] {$1$};
                    \draw  (6.8,10.3) rectangle (7.8,9.3) node[pos=.5] {$0$};
                    \draw  (7.8,10.3) rectangle (8.8,9.3) node[pos=.5] {$0$};
                    \draw  (8.8,10.3) rectangle (9.8,9.3) node[pos=.5] {$0$};
                    \draw  (9.8,10.3) rectangle (10.8,9.3) node[pos=.5] {$1$};
                    \node[right] at (10.8,9.8) {\vspace{100 pt} grouping $g_2$ is \{$x_1, x_5$\}};
        
                    % Linha 3 (010)
                    \node[left] at (5.8,8.8) {$g_3 \rightarrow$};
                    \draw  (5.8,9.3) rectangle (6.8,8.3) node[pos=.5] {$0$};
                    \draw  (6.8,9.3) rectangle (7.8,8.3) node[pos=.5] {$0$};
                    \draw  (7.8,9.3) rectangle (8.8,8.3) node[pos=.5] {$1$};
                    \draw  (8.8,9.3) rectangle (9.8,8.3) node[pos=.5] {$0$};
                    \draw  (9.8,9.3) rectangle (10.8,8.3) node[pos=.5] {$1$};
                    \node[right] at (10.8,8.8) {\vspace{100 pt} grouping $g_3$ is \{$x_3, x_5$\}};

                    \node[below] at (8.5,8.3) {$\vdots$};
        
                    % Linha 4 (011)
                    \node[left] at (5.8,6.8) {$g_{n-1} \rightarrow$};
                    \draw  (5.8,6.3) rectangle (6.8,7.3) node[pos=.5] {$0$};
                    \draw  (6.8,6.3) rectangle (7.8,7.3) node[pos=.5] {$0$};
                    \draw  (7.8,6.3) rectangle (8.8,7.3) node[pos=.5] {$1$};
                    \draw  (8.8,6.3) rectangle (9.8,7.3) node[pos=.5] {$0$};
                    \draw  (9.8,6.3) rectangle (10.8,7.3) node[pos=.5] {$0$};
                    \node[right] at (10.8,6.8) {\vspace{100 pt} grouping $g_{n-1}$ is \{$x_2, x_5$\}};
        
                    % Linha 5 (100)
                    \node[left] at (5.8,5.8) {$g_n \rightarrow$};
                    \draw  (5.8,5.3) rectangle (6.8,6.3) node[pos=.5] {$1$};
                    \draw  (6.8,5.3) rectangle (7.8,6.3) node[pos=.5] {$1$};
                    \draw  (7.8,5.3) rectangle (8.8,6.3) node[pos=.5] {$0$};
                    \draw  (8.8,5.3) rectangle (9.8,6.3) node[pos=.5] {$0$};
                    \draw  (10.8,5.3) rectangle (9.8,6.3) node[pos=.5] {$0$};
                    \node[right] at (10.8,5.8) {\vspace{100 pt} grouping $g_n$ is \{$x_1, x_2$\}};

                    % \node[below] at (8.8,5.3) {Note that $x_4$ is grouped alone.};
        
                \end{circuitikz}
            }%
            \caption{} 
            \label{fig:genome_2_example}
        \end{subfigure}
        \caption[Example of a chromosome 2]{Example of a chromosome when there are no allowed grouping objects predefined. (a) Instances of arbitrary classes. They are $x_1$, $x_2$, $x_3$, $x_4$, and $x_5$. There are no predefined allowed grouping objects. (b) Chromosome represented as a binary matrix for $m=5$. It represents which instance is part of each grouping.} 
        \label{fig:genome_2}
    \end{figure}

            These representations are not optimized, but can cover all the cases.
            %%
            As these representations are binary, there is no need to elaborate specific mutations and crossover operations, since the literature already implements many operations (e.g. single-point crossover, multi-point crossover, flip mutation) \cite{deap}.

            
        
     \subsection{Mapping Inputs and Outputs of Solvers}
        Each $Solver$ needs to translate the input from the SCF to something that its optimization algorithm can comprehend and optimize. It adapts the SCF to the particular requirements of the algorithm. This step is essential for making the data usable by the algorithm in question, as different algorithms may have unique input formats or requirements.

        On the other hand, there is also need to translate the output to a SCF, ensuring consistency and compatibility with the SCF representation. It outputs a list of groupings, which optimizes the grouping problem. This final step ensures that the results are presented in a organized manner, facilitating further applications.
\chapter{Fair Matching} \label{chap:fair_matching}
    
    Matching in bipartite graphs refers to determining optimal combinations between elements of two sets, $U$ and $V$, often considering predefined constraints and specific preferences. 
    %
    This paradigm is frequently applied in contexts such as resource allocation, network design, and matching in distributed systems, which demonstrates its versatility by offering solutions to a wide range of practical problems \cite{Karp1972}.
    %
    
    %
    One of the conventional approaches to solving bipartite matching relies on reducing the problem to a Minimum Cost Max Flow (MCMF) problem, where several algorithms may be employed.
    %
    However, it does not often consider attributes among specific sub-sets of $U$ and $V$, which may possess distinct traits that are essential in various contexts.
    %
    Considering such sub-sets' characteristics makes the bipartite matching problem relevant to a large set of real-world applications, we call this the \emph{Fair Bipartite Matching Problem}.
    %

    %
    We focus on the \emph{minimum quota} characteristic among sub-sets.
    The \emph{minimum quota} defines the minimal amount of elements of each sub-set that must be matched.  % within the match // in the solution ....
    %
    By incorporating \emph{minimum quotas} for sub-sets, the proposed method aims for both: operational efficiency and equity in the distribution of resources among sub-sets. 
    %
    In this chapter, we present a reduction from the Fair Bipartite Matching Problem with \emph{minimum quota} to the MCMF.
    %

    %
    The practicality of this approach is evident in various scenarios where both qualitative and quantitative criteria are crucial in the pairing process. 
    A prominent example is in recruitment, where evaluating a potential employee's soft skills is as important as assessing their hard skills through objective methods like exams. 
    Another clear application is in network optimization, particularly in selecting the best data centers for different applications. 
    Factors such as backup type and location are just as important as latency in making these decisions.
    %
    
    %    
    Several works already address the concept of fairness, which has become increasingly important. However, most approaches rely on heuristic \cite{hatfield2005matching, rostami2018fairness, pappalardo2020algorithmic} or metaheuristic methods \cite{manlove2017algorithmics, yan2021metaheuristic, chapman2017matching}. Here, however, we develop and present a deterministic model in polynomial time, introducing an innovative approach for solving the Fair Bipartite Matching problem. The central proposal involves applying the concept of Minimum-Cost Maximum Flow (MCMF) through an effective mapping of the problem, allowing for the incorporation of fairness criteria, such as minimum quotas for specific sets.
    %

    \section{Related Works}
        Several methods have been proposed to address problems of matching and fair allocation. Table~\ref{tab:related_work} summarizes some of the most relevant approaches in terms of accuracy, fairness, and the use of minimum quotas.
        
        \begin{table}[ht]
          \centering
          \begin{tabular}{lccc}
            \toprule
            Methods & Exact & Fair & Minimum Quotas \\
            \midrule
            \textbf{Proposed} & \textbf{Yes} & \textbf{Yes} & \textbf{Yes} \\
            \cite{ahuja1993network}; \cite{edmonds1972theoretical}; \cite{tarjan1997dynamic} & Yes & No & No \\
            \cite{hatfield2005matching}; \cite{rostami2018matching}; \cite{pappalardo2020combining}; \cite{manlove2017algorithmics}; \cite{yan2021evolutionary}; \cite{chapman2017multi} & No & Yes & No \\
            \cite{sankar2021set} & No & Yes & Yes \\
            \cite{garcia2020fair} & Yes & Yes & No \\
            \bottomrule
          \end{tabular}
          \caption{Comparison of Allocation Methods, proposed and from literature, in Terms of Accuracy, Fairness, and Minimum Quotas.}
          \label{tab:related_work}
        \end{table}
        
        The methods proposed by Ahuja et al. (1993) \cite{ahuja1993network}, Edmonds and Karp (1972) \cite{Karp1972}, and Tarjan (1997) \cite{tarjan1997dynamic} focus primarily on matching accuracy without considering fairness or specific quotas.
        
        On the other hand, Hatfield and Milgrom (2005) \cite{hatfield2005matching}, Rostami et al. (2018) \cite{rostami2018matching}, Pappalardo and Conitzer (2020) \cite{pappalardo2020combining}, Manlove (2017) \cite{manlove2017algorithmics}, Yan et al. (2021) \cite{yan2021evolutionary}, and Chapman et al. (2017) \cite{chapman2017multi} address fairness in matching but do not include specific quotas in their formulations.
        
        Sankar et al. (2021) \cite{sankar2021set} present a method that considers both fairness and minimum quotas, but without exact matching. García-Soriano and Bonchi (2020) \cite{garcia2020fair} propose an exact and fair method that does not incorporate quotas.
        
        Our method proposes an approach that is exact, fair, and respects minimum quotas, promoting efficient and equitable allocation, inspired by the fairness set concept as proposed by Sankar et al. (2021) \cite{sankar2021set}.

        
        \section{Solution Modeling}
        
            As discussed in Section {\ref{subsubsec:resolucao-fluxo-matching}}, the MCMF algorithm is traditionally employed to solve matching problems without considering the concept of fairness. We present how to extend the use of MCMF to the proposed concept of fairness. This extension will be achieved through a mapping that constructs a graph to be processed by an MCMF algorithm to solve the original problem.
            
            \subsection{Objective}
            
                Based on the intrinsic characteristics of the MCMF, this method aims to maximize the flow, and among all maximum flows, select the one with the minimum cost.
                %
                In our case, we search for maximum flow to ensure fair selection in the matching, with the guarantee of respecting the predefined minimum quotas. In situations where there are multiple ways to achieve this allocation, our aim is to minimize the associated matching costs.
                
                The solution aims to meet the following points:
                
                \begin{enumerate} \item \textbf{Comply for Minimum Quotas}: Ensure that the predefined minimum quotas for all subsets are fully fulfill (if possible) during the matching process, promoting equity and inclusion. \item \textbf{Cost Minimization}: In situations where there are various matching alternatives, the objective is to minimize the costs associated with these matchings, providing operational and economic efficiency. \end{enumerate}
        
        
        \section{Modifications to Include Fairness in Bipartite Matching}
        
            The fundamental changes made to the mapping aim to make it impossible for the flow to avoid paths with minimum quota elements (in other words, to force the flow through paths with minimum quota elements), as well as to prevent the repeated selection of the same element.
            Subsequently, the reduction of the Fair Matching problem to MCMF is addressed, along with the proofs of the modeling.
            
            \subsection{Minimum Quotas}
            
                The sets of minimum quota members are defined as subsets of vertices of the same class (e.g., workers) that share one or more characteristics (e.g., workers with disability) and have a minimum number of \textit{matches} if the maximum match is achieved.
                
                Each set of minimum quota members is represented by a vertex, where the incoming edge has a capacity equivalent to the number of \textit{matches} the set needs to obtain, with an associated cost of zero. This configuration encourages the passage of the maximum flow through this vertex, as part of the total flow is required to transit through the quota vertices.
                Considering a graph with a total possible of $N$ \textit{matches} and having $|Q|$ sets of quotas, where the \( i \)-th quota, $q_i$ has $n_{q_i}$ slots.
                In Equation \ref{eq:quota_sum} we define the total matches assigned to quota slots, $N_Q$ as:

                \begin{equation}
                    N_Q = \sum_{i=1}^{|Q|} n_{q_i} \leq N
                    \label{eq:quota_sum}
                \end{equation}
                
                Thus, a set with $n_{q_i}$ quota, with the described mapping, ensures that the other vertices can only contribute a maximum of $N - N_Q$ flow units. This guarantees that if there is a maximum flow passing through the quota set, it will be selected.
                %
                In Figures~\ref{fig:mapeamento}~and~\ref{fig:solucao}, these vertices are shown in purple / loosely dash-dotted.
                
                Here, \( N \) represents the total number of possible matches, typically defined by the smaller of the two sets (demand or supply). In our case, demand (\( V \)) is always less than or equal to supply (\( U \)), so \( N \) is at most the size of the demand set \( V \). However, depending on the graph's structure and quota requirements, it is possible that not all demand is matched, meaning \( N \) can be equal to or less than \( V \).
        
        
            \subsection{Wide Competition (WC)} \label{sec:wc}
            
                The Wide Competition (WC) indicates the number of \textit{matches} that have not been previously assigned to any quota set. Let $N$ be the total possible matches overall, and $N_q$ the total matching assigned to quota slots. The wide competition slots, $n_{wc}$, is defined by Equation \ref{eq:n_wc}.
                \begin{equation}
                    n_{wc} =
                    \begin{cases}
                        N - N_Q & \text{if } N > N_Q \\
                        0 & \text{otherwise}
                    \end{cases}
                    \label{eq:n_wc}
                \end{equation}
                
                Without loss of generality, the WC set can be understood as a quota. All elements are part of the wide competition. This configuration ensures that there will be no shortage of matches if the maximum matching is possible. In Figures~\ref{fig:mapeamento} and~\ref{fig:solucao}, this vertex is represented in gray.
            
            \subsection{Proxies}
            
                A proxy is an entity represented by a vertex, whose outgoing edge has unit capacity.
                %
                There is always a proxy at the entrance of each element of the set where quotas are applied.
                %
                This proxy ensures that each element is selected only once when it participates in multiple quota groups (e.g., a worker that belongs to both, disable and female subsets, or a worker that belongs to both, WC and quota group). In Figures~\ref{fig:mapeamento} and~\ref{fig:solucao}, this vertex is identified in dark green / dash-dotted.
            
            \subsection{Mapping}
            
            For a more formal description of the mapping, consider two sets $U$ and $V$. The goal is to perform a matching $M$ between $U$ and $V$, minimizing the sum of the costs of the edges in $M$. Additionally, there is a set of quotas $Q$ of arbitrary size $|Q|$, each quota is identified by $q_i$, where $0 < i \leq |Q|$. Each quota $q_i$ represents the elements of a specific subset of $U$ (the elements that belong to the $q_i$ quota group, namely $U_{q_i}$). Each quota has an assigned number $n_{q_i}$ that represents how many elements of $q_i$ should be in the final matching. Each element of $U$, denoted as $u_i$, has its own proxy, called $p_{i}$. The set of all proxies is denoted by $P$, and a subset $P_{U_{q_i}}$ represents all proxies of the elements of $U$ that belong to quota $q_i$.
            
            The mapping always results in a multilayer graph containing six layers:
            
            \begin{itemize}
                \item \textbf{Relations between Layer 1 (\textit{Source}) and Layer 2}: The flow always originates from the \textit{Source}, located in Layer 1. It has $|Q|+1$ outgoing edges, connecting to the vertices of Layer 2, which represent each quota in $Q$ and also the \textit{wide competition} vertex, called $wc$. The weights of these edges are always zero, while the capacity is defined as $n_{q_i}$ when connected to a $q_i$ or $n_{wc}$ (number of remaining matchings) when connected to $wc$.
                
                \item \textbf{Relations between Layer 2 and Layer 3}: In Layer 3 are the proxies. Each vertex in Layer 2, $q_i$ or $wc$, has an edge with capacity 1 and cost 0 connecting it to the corresponding proxy of the element of $U$ contained in its subset. That is, every edge $q_i$ will have an edge to the elements of $P_{U_{q_i}}$. Additionally, the vertex $wc$ is connected to all vertices in Layer 3, always with capacity 1 and cost 0.
                
                \item \textbf{Relations between Layer 3 and Layer 4}: This pair of layers aims to prevent an element of $U$ from being matched more than once. Every vertex $P_{u_i}$ connects to its corresponding $u_i$, with zero cost and unit capacity.
                
                \item \textbf{Relations between Layer 4 and Layer 5}: At this stage, matching costs are taken into account. The cost of the edges connecting the two sets is defined by the relationships between elements of $U$ (Layer 4) and $V$ (Layer 5), and the cost between element $u_i$ and $v_j$ is called $C_{u_i v_j}$. Additionally, the capacity remains 1.
                
                \item \textbf{Relations between Layer 5 and Layer 6 (\textit{Sync})}: These are the final edges and serve to ensure that the elements of $V$ are matched only once, as well as to close the circuit cycle. Each element of $V$ has an edge to the \textit{Sink}, with capacity 1 and cost 0.
            \end{itemize}
            
            In Figure \ref{fig:mapeamento}, a visual representation of this mapping can be seen. Additionally, Figure \ref{fig:solucao} shows the solution to the same problem as Figure \ref{fig:mapeamento} if $U_1$, $U_2$ and $U_3$ belonged to Quota$_1$ and $U_5$ belonged to Quota$_2$.
            
            \begin{figure}[!ht]
                \centering
                \begin{subfigure}[!ht]{.95\textwidth}
                    \centering
                        \resizebox{0.7\textwidth}{!}{%
      \begin{tikzpicture} [xscale=0.45, yscale=1.05]
        \tikzstyle{style} = [circle, draw, ultra thick, minimum size=45pt, inner sep=0pt, text centered]
        \tikzstyle{no_border} = [circle, minimum size=45pt, inner sep=0pt, text centered, opacity=0.5]
        \tikzstyle{layerlabel}=[text width=5cm, align=center, font=\Large]

        
       \draw
        (-20.0, 7.5) node[layerlabel] (-1){Layer 1}
        (-12.0, 7.5) node[layerlabel] (-1){Layer 2}
        (-4.0, 7.5) node[layerlabel] (-1){Layer 3}
        (4.0, 7.5) node[layerlabel] (-1){Layer 4}
        (12.0, 7.5) node[layerlabel] (-1){Layer 5}
        (20.0, 7.5) node[layerlabel] (-1){Layer 6}
        
        (-20.0, 0.0) node[draw=black,style] (0){Source}
        (-12.0, 2.0) node[draw=purple, loosely dashdotted,style] (1){$q_1$}
        (-12.0, 0.0) node[draw=purple, loosely dashdotted,style] (2){$q_2$}
        (-12.0, -2.0) node[draw=lightgray,loosely dotted,style] (3){$wc$}
        (4.0, 5.0) node[draw=red, dashed,style] (4){$u_1$}
        (4.0, 3.0) node[draw=red, dashed,style] (5){$u_2$}
        (4.0, 1.0) node[draw=red, dashed,style] (6){$u_3$}
        (4.0, -1.0) node[draw=red, dashed,style] (7){$u_4$}
        (4.0, -3.0) node[draw=red, dashed,style] (8){$u_5$}
        (4.0, -5.0) node[draw=red, dashed,style] (9){$u_6$}
        (-4.0, 5.0) node[draw=green!50!black, dashdotted,style] (10){$p_1$}
        (-4.0, 3.0) node[draw=green!50!black, dashdotted,style] (11){$p_2$}
        (-4.0, 1.0) node[draw=green!50!black, dashdotted,style] (12){$p_3$}
        (-4.0, -1.0) node[draw=green!50!black, dashdotted,style] (13){$p_4$}
        (-4.0, -3.0) node[draw=green!50!black, dashdotted,style] (14){$p_5$}
        (-4.0, -5.0) node[draw=green!50!black, dashdotted,style] (15){$p_6$}
        (12.0, 3.0) node[draw=blue, dotted,style] (16){$v_1$}
        (12.0, -3.0) node[draw=blue, dotted,style] (17){$v_3$}
        (12.0, 0.0) node[draw=blue, dotted,style] (18){$v_2$}
        (20.0, 0.0) node[draw=black,style] (19){Sink};
      \begin{scope}[->, >=BigLatex]
        \draw[lightgray, text=black, font=\footnotesize] (0) to node[] {$n_{q_1}$ ; 0} (1);
        \draw[lightgray, text=black, font=\footnotesize] (0) to node[] {$n_{q_2}$ ; 0} (2);
        \draw[lightgray, text=black, font=\footnotesize] (0) to node[] {$n_{WC}$ ; 0} (3);
        \draw[lightgray, text=black, font=\footnotesize] (1) to node[] {1 ; 0} (10);
        \draw[lightgray, text=black, font=\footnotesize] (1) to node[] {1 ; 0} (11);
        \draw[lightgray, text=black, font=\footnotesize] (1) to node[] {1 ; 0} (12);
        % \draw[lightgray, text=black, font=\footnotesize] (2) to node[] {1 ; 0} (13);
        \draw[lightgray, text=black, font=\footnotesize] (2) to node[] {1 ; 0} (14);
        \draw[lightgray, text=black, font=\footnotesize] (3) to node[] {1 ; 0} (10);
        \draw[lightgray, text=black, font=\footnotesize] (3) to node[] {1 ; 0} (11);
        \draw[lightgray, text=black, font=\footnotesize] (3) to node[] {1 ; 0} (12);
        \draw[lightgray, text=black, font=\footnotesize] (3) to node[] {1 ; 0} (13);
        \draw[lightgray, text=black, font=\footnotesize] (3) to node[] {1 ; 0} (15);
        \draw[lightgray, text=black, font=\footnotesize] (3) to node[] {1 ; 0} (14);
        \draw[lightgray, text=black, font=\footnotesize] (10) to node[] {1 ; 0} (4);
        \draw[lightgray, text=black, font=\footnotesize] (4) to node[] {1 ; $C_{u_1 v_1}$} (16);
        \draw[lightgray, text=black, font=\footnotesize] (11) to node[] {1 ; 0} (5);
        \draw[lightgray, text=black, font=\footnotesize] (5) to node[] {1 ; $C_{u_2 v_2}$} (18);
        \draw[lightgray, text=black, font=\footnotesize] (12) to node[] {1 ; 0} (6);
        \draw[lightgray, text=black, font=\footnotesize] (6) to node[] {1 ; $C_{u_3 v_2}$} (18);
        \draw[lightgray, text=black, font=\footnotesize] (13) to node[] {1 ; 0} (7);
        \draw[lightgray, text=black, font=\footnotesize] (7) to node[] {1 ; $C_{u_4 v_3}$} (17);
        \draw[lightgray, text=black, font=\footnotesize] (15) to node[] {1 ; 0} (9);
        \draw[lightgray, text=black, font=\footnotesize] (8) to node[] {1 ; $C_{u_5 v_2}$} (18);
        \draw[lightgray, text=black, font=\footnotesize] (8) to node[] {1 ; $C_{u_5 v_3}$} (17);
        \draw[lightgray, text=black, font=\footnotesize] (14) to node[] {1 ; 0} (8);
        \draw[lightgray, text=black, font=\footnotesize] (9) to node[] {1 ; $C_{u_6 v_1}$} (16);
        \draw[lightgray, text=black, font=\footnotesize] (16) to node[] {1 ; 0} (19);
        \draw[lightgray, text=black, font=\footnotesize] (17) to node[] {1 ; 0} (19);
        \draw[lightgray, text=black, font=\footnotesize] (18) to node[] {1 ; 0} (19);
      \end{scope}
    \end{tikzpicture}
    }%

                    \caption{Example of a possible mapping of Fair Bipartite Matching.}
                    \label{fig:mapeamento}
                \end{subfigure}
                \hfill
                \begin{subfigure}[!ht]{0.8\textwidth}
                    \centering
                    \resizebox{0.7\textwidth}{!}{%
      \begin{tikzpicture} [xscale=0.45, yscale=1.05]
        \tikzstyle{style} = [circle,ultra thick, draw, minimum size=45pt, inner sep=0pt, text centered]
        \tikzstyle{no_border} = [circle,ultra thick, minimum size=45pt, inner sep=0pt, text centered, opacity=0.5]
      \tikzstyle{WC_style} = [circle,ultra thick, draw, minimum size=45pt, inner sep=0pt, text centered, text width=1.5cm, align=center, font=\scriptsize]


       \draw
        (-20.0, 0.0) node[draw=black,style] (0){Source}
        (-12.0, 2.0) node[draw=purple, loosely dashdotted,style] (1){$q$$_1$}
        (-12.0, 0.0) node[draw=purple, loosely dashdotted,style] (2){$q$$_2$}
        (-12.0, -3.0) node[draw=lightgray,loosely dotted,style] (3){$wc$}
        (4.0, 5.0) node[draw=red, dashed,,style] (4){$u$$_1$}
        (4.0, 3.0) node[draw=red, dashed,,no_border] (5){$u$$_2$}
        (4.0, 1.0) node[draw=red, dashed,,no_border] (6){$u$$_3$}
        (4.0, -1.0) node[draw=red, dashed,,style] (7){$u$$_4$}
        (4.0, -5.0) node[draw=red, dashed,,no_border] (8){$u$$_6$}
        (4.0, -3.0) node[draw=red, dashed,,style] (9){$u$$_5$}
        (-4.0, 5.0) node[draw=green!50!black, dashdotted,style] (10){$p$$_1$}
        (-4.0, 3.0) node[draw=green!50!black, dashdotted,no_border] (11){$p$$_2$}
        (-4.0, 1.0) node[draw=green!50!black, dashdotted,no_border] (12){$p$$_3$}
        (-4.0, -1.0) node[draw=green!50!black, dashdotted,style] (13){$p$$_4$}
        (-4.0, -3.0) node[draw=green!50!black, dashdotted,style] (14){$p$$_5$}
        (-4.0, -5.0) node[draw=green!50!black, dashdotted,no_border] (15){$p$$_6$}
        (12.0, 2.0) node[draw=blue, dotted,style] (16){$v$$_1$}
        (12.0, -3.0) node[draw=blue, dotted,style] (17){$v$$_3$}
        (12.0, 0.0) node[draw=blue, dotted,style] (18){$v$$_2$}
        (20.0, 0.0) node[draw=black,style] (19){Sink};
      \begin{scope}[->, >=BigLatex]
        \draw[lightgray, text=black, font=\footnotesize] (0) to node[] {1 ; 0} (1);
        \draw[lightgray, text=black, font=\footnotesize] (0) to node[] {1 ; 0} (2);
        \draw[lightgray, text=black, font=\footnotesize] (0) to node[] {1 ; 0} (3);
        \draw[lightgray, text=black, font=\footnotesize] (1) to node[] {1 ; 0} (10);
        \draw[transparent] (1) to node[] {1 ; 0} (11);
        \draw[transparent] (1) to node[] {0 ; 0} (12);
        \draw[transparent] (2) to node[] {0 ; 0} (13);
        \draw[lightgray, text=black, font=\footnotesize] (2) to node[] {1 ; 0} (14);
        \draw[transparent] (3) to node[] {1 ; 0} (10);
        \draw[transparent] (3) to node[] {0 ; 0} (11);
        \draw[transparent] (3) to node[] {0 ; 0} (12);
        \draw[lightgray, text=black, font=\footnotesize] (3) to node[] {1 ; 0} (13);
        \draw[transparent] (3) to node[] {0 ; 0} (15);
        \draw[transparent] (3) to node[] {0 ; 0} (14);
        \draw[lightgray, text=black, font=\footnotesize] (10) to node[] {1 ; 0} (4);
        \draw[lightgray, text=black, font=\footnotesize] (4) to node[] {1 ; 1} (16);
        \draw[transparent] (5) to node[] {1 ; 0} (11);
        \draw[transparent] (5) to node[] {1 ; 7} (18);
        \draw[transparent] (6) to node[] {0 ; 0} (12);
        \draw[transparent] (6) to node[] {0 ; 7} (18);
        \draw[lightgray, text=black, font=\footnotesize] (13) to node[] {1 ; 0} (7);
        \draw[lightgray, text=black, font=\footnotesize] (7) to node[] {1 ; 1} (17);
        \draw[transparent] (8) to node[] {0 ; 0} (15);
        \draw[transparent] (8) to node[] {0 ; 10} (18);
        \draw[transparent] (8) to node[] {0 ; 10} (17);
        \draw[lightgray, text=black, font=\footnotesize] (14) to node[] {1 ; 0} (9);
        \draw[lightgray, text=black, font=\footnotesize] (9) to node[] {1 ; 10} (18);
        \draw[lightgray, text=black, font=\footnotesize] (16) to node[] {1 ; 0} (19);
        \draw[lightgray, text=black, font=\footnotesize] (17) to node[] {1 ; 0} (19);
        \draw[lightgray, text=black, font=\footnotesize] (18) to node[] {1 ; 0} (19);
      \end{scope}
    \end{tikzpicture}
    }%
                    \caption[Solution of the problem from Figure \ref{fig:mapeamento}]{Solution of the problem from Figure \ref{fig:mapeamento} if $u_1$, $u_2$, and $u_3$ belonged to quota $q_1$ and $u_5$ belonged to quota $q_2$. The selected elements are those whose edge flows are not 0. The unselected elements are faded out. Total cost of 12.}
                    \label{fig:solucao}
                \end{subfigure}
                \caption{Examples of Fair Bipartite Matching mapping and its solution.}
                \label{fig:matching_examples}
            \end{figure}
        
        \section{Proofs}
        
        The objective of this proof is to demonstrate that the optimization respects the minimum quotas and minimizes costs, as well as to show that elements of a quota can participate in wide competition if it optimizes the matching.
        
        \subsection{Respecting Minimum Quotas}
        
        \begin{lemma}
           The total amount of flow passing through the vertices representing a quota $q_i$ will not exceed $n_{q_i}$.
        \end{lemma}
        
        To ensure that the predefined minimum quotas for the sets are fully respected during the matching process, it is crucial that the capacity configuration is well-defined. Each vertex in Layer 2 (quotas $q_i$ and wide competition $wc$) is connected to the \textit{Source} vertex with capacities defined as $n_{q_i}$ for $q_i$ and $n_{wc}$ for $wc$. This guarantees that the maximum number of flows for each $q_i$ is $n_{q_i}$, respecting the minimum quotas.
        
        Furthermore, the MCMF problem finds the maximum flow that passes through the graph while respecting the edge capacities. The total amount of flow passing through the vertices representing the quotas $q_i$ will not exceed $n_{q_i}$, ensuring that the minimum quotas are respected.
        
        \subsection{Cost Minimization}
        
        \begin{lemma}
           MCMF maximizes the flow in a graph, and among all maximum flows, it finds the one whose sum of edge costs is the smallest \cite{edmonds1972theoretical}.
        \end{lemma}
        
        \begin{lemma}
           The edges between the elements that optimize the MCMF represent the matchings between the elements of groups $U$ and $V$ \cite{edmonds1972theoretical}.
        \end{lemma}
        
        To minimize the costs associated with the matchings, the edges between the elements of $U$ and $V$ in Layers 4 and 5 have associated costs that represent the cost of matching. The MCMF algorithm minimizes the total cost of the flow by choosing the lowest-cost matchings between $U$ and $V$.
        
        Each edge between $U$ and $V$ has a capacity of 1, ensuring that each element is matched at most once. The minimization of the total cost will be based on selecting matchings that result in the lowest aggregate cost, respecting the capacity structure.
        
        \subsection{Respecting Matching Uniqueness}
        
        \begin{lemma}
           Every element in $U$ and $V$ has only one unit of outgoing flow, ensuring that no member of $U$ and $V$ is matched more than once.
        \end{lemma}
        
        Every element in $U$ has only one unit of outgoing flow, ensuring that no member of $U$ is matched more than once. The same analysis applies to the vertices in $V$.
        
        \subsection{Theorem Formulation}
        
        \begin{theorem}
        If the filling of the quotas is possible, the mapping of a fair bipartite matching problem to an MCMF problem will fulfill the quotas, respect the use of each resource, and minimize costs.
        \end{theorem}
        
        As discussed, the capacities and costs of the edges ensure that the quotas $q_i$ are respected. The MCMF algorithm minimizes the total cost of the matching by selecting the lowest-cost edges between $U$ and $V$. Additionally, no element of $U$ and $V$ will be matched more than once.
        
        \section{Examples}
        %{\color{green}
        The objective of this section is to demonstrate various contexts in which the mapping of fairness in matching to the Minimum Cost Maximum Flow (MCMF) problem can be effectively applied.
        
        For clarity and readability in the illustrations, the costs of the unsolved mappings will be suppressed.
        
        \subsection{Fairness in workers to positions allocation}
        \label{sec:workers_jobs_example}

        In the first scenario, we address the challenge of forming a diverse team where some workers have specific limitations, such as low vision or mobility. The objective is to assemble the most diverse and optimal team possible. Each worker expresses their preferences for available job roles, and an objective evaluation method—such as an exam—assigns costs based on prior performance, with lower costs corresponding to higher performance to maintain a minimum-cost solution. Given that the jobs differ in nature, the evaluation process assigns distinct costs to each job-choice vertex, reflecting each worker's suitability for those roles.
        
        \begin{figure}[]
            \centering
            \begin{subfigure}[t]{0.8\textwidth}
                \centering
                \resizebox{0.7\textwidth}{!}{%
  \begin{tikzpicture} [xscale=0.45, yscale=1.05]
    \tikzstyle{style} = [circle, draw, ultra thick, minimum size=45pt, inner sep=0pt, text centered]
    \tikzstyle{no_border} = [circle, ultra thick, minimum size=45pt, inner sep=0pt, text centered, opacity=0.5]

    \draw
      (-20.0, 0.0) node[draw=black,style] (0){Source}
      (-12.0, 2.5) node[draw=purple, loosely dashdotted, style, text width=1.5cm, align=center] (1){$people$ $with$ $low$ \\ $vision$}
      (-12.0, 0.0) node[draw=purple, loosely dashdotted, style, text width=1.5cm, align=center] (2){$people$ $with$ $low$ \\ $mobility$}
      (-12.0, -2.5) node[draw=lightgray, loosely dotted,style] (3){$wc$}
      (4.0, 5.0) node[draw=red, dashed,style] (4){$worker_1$}
      (4.0, 3.0) node[draw=red, dashed,style] (5){$worker_2$}
      (4.0, 1.0) node[draw=red, dashed,style] (6){$worker_3$}
      (4.0, -1.0) node[draw=red, dashed,style] (7){$worker_4$}
      (4.0, -3.0) node[draw=red, dashed,style] (8){$worker_5$}
      (4.0, -5.0) node[draw=red, dashed,style] (9){$worker_6$}
      (-4.0, 5.0) node[draw=green!50!black, dashdotted,style] (10){$p_1$}
      (-4.0, 3.0) node[draw=green!50!black, dashdotted,style] (11){$p_2$}
      (-4.0, 1.0) node[draw=green!50!black, dashdotted,style] (12){$p_3$}
      (-4.0, -1.0) node[draw=green!50!black, dashdotted,style] (13){$p_4$}
      (-4.0, -3.0) node[draw=green!50!black, dashdotted,style] (14){$p_5$}
      (-4.0, -5.0) node[draw=green!50!black, dashdotted,style] (15){$p_6$}
      (12.0, 3.0) node[draw=blue, dotted,style] (16){$position_1$}
      (12.0, -3.0) node[draw=blue, dotted,style] (17){$position_3$}
      (12.0, 0.0) node[draw=blue, dotted,style] (18){$position_2$}
      (20.0, 0.0) node[draw=black,style] (19){Sink};

    \begin{scope}[->, >=BigLatex]
      \draw[lightgray, text=black, font=\footnotesize] (0) to node[] {$n_{PWLV}$ ; 0} (1);
      \draw[lightgray, text=black, font=\footnotesize] (0) to node[] {$n_{PWLM}$ ; 0} (2);
      \draw[lightgray, text=black, font=\footnotesize] (0) to node[] {$n_{wc}$ ; 0} (3);
      \draw[lightgray, text=black, font=\footnotesize] (1) to node[] {1 ; 0} (10);
      \draw[lightgray, text=black, font=\footnotesize] (1) to node[] {1 ; 0} (11);
      \draw[lightgray, text=black, font=\footnotesize] (1) to node[] {1 ; 0} (12);
      % Adjusted vertical spacing here
      \draw[lightgray, text=black, font=\footnotesize] (2) to node[] {1 ; 0} (13);
      \draw[lightgray, text=black, font=\footnotesize] (2) to node[] {1 ; 0} (14);
      \draw[lightgray, text=black, font=\footnotesize] (3) to node[] {1 ; 0} (10);
      \draw[lightgray, text=black, font=\footnotesize] (3) to node[] {1 ; 0} (11);
      \draw[lightgray, text=black, font=\footnotesize] (3) to node[] {1 ; 0} (12);
      \draw[lightgray, text=black, font=\footnotesize] (3) to node[] {1 ; 0} (13);
      \draw[lightgray, text=black, font=\footnotesize] (3) to node[] {1 ; 0} (15);
      \draw[lightgray, text=black, font=\footnotesize] (3) to node[] {1 ; 0} (14);
      \draw[lightgray, text=black, font=\footnotesize] (10) to node[] {1 ; 0} (4);
      \draw[lightgray, text=black, font=\footnotesize] (4) to node[] {1 ; $C_{W_1 P_1}$} (16);
      \draw[lightgray, text=black, font=\footnotesize] (11) to node[] {1 ; 0} (5);
      \draw[lightgray, text=black, font=\footnotesize] (5) to node[] {1 ; $C_{W_2 P_2}$} (18);
      \draw[lightgray, text=black, font=\footnotesize] (12) to node[] {1 ; 0} (6);
      \draw[lightgray, text=black, font=\footnotesize] (6) to node[] {1 ; $C_{W_3 P_2}$} (18);
      \draw[lightgray, text=black, font=\footnotesize] (13) to node[] {1 ; 0} (7);
      \draw[lightgray, text=black, font=\footnotesize] (7) to node[] {1 ; $C_{W_4 P_3}$} (17);
      \draw[lightgray, text=black, font=\footnotesize] (15) to node[] {1 ; 0} (9);
      \draw[lightgray, text=black, font=\footnotesize] (8) to node[] {1 ; $C_{W_5 P_2}$} (18);
      \draw[lightgray, text=black, font=\footnotesize] (8) to node[] {1 ; $C_{W_5 P_3}$} (17);
      \draw[lightgray, text=black, font=\footnotesize] (14) to node[] {1 ; 0} (8);
      \draw[lightgray, text=black, font=\footnotesize] (9) to node[] {1 ; $C_{W_6 P_1}$} (16);
      \draw[lightgray, text=black, font=\footnotesize] (16) to node[] {1 ; 0} (19);
      \draw[lightgray, text=black, font=\footnotesize] (17) to node[] {1 ; 0} (19);
      \draw[lightgray, text=black, font=\footnotesize] (18) to node[] {1 ; 0} (19);
    \end{scope}
  \end{tikzpicture}
}%

                \caption{Example of worker to positions mapping.}
                \label{fig:workers_jobs_unsolved}
            \end{subfigure}
            \vspace{1em}
            \begin{subfigure}[t]{0.8\textwidth}
                \centering
                 % \begin{tikzpicture}[scale=0.8]
 %      \draw
 %        (-9.0, 0.0) node[fill=cyan, rounded corners] (0){Source}
 %        (-5.745, 1.085) node[fill=pink, rounded corners] (1){Cota}
 %        (-5.745, 0.0) node[fill=pink, rounded corners] (2){Cota}
 %        (-5.745, -1.085) node[fill=gray, rounded corners] (3){Resto}
 %        (0.766, 0.0) node[fill=brown, rounded corners] (4){$u$}
 %        (0.766, -1.085) node[fill=brown, rounded corners] (5){$u$}
 %        (0.766, 1.085) node[fill=brown, rounded corners] (6){$u$}
 %        (-2.489, 1.085) node[fill=yellow, rounded corners] (7){$p$}
 %        (-2.489, 0.0) node[fill=yellow, rounded corners] (8){$p$}
 %        (-2.489, -1.085) node[fill=yellow, rounded corners] (9){$p$}
 %        (4.021, 2.713) node[fill=orange, rounded corners] (10){$u$}
 %        (4.021, 1.628) node[fill=orange, rounded corners] (11){$u$}
 %        (4.021, 0.543) node[fill=orange, rounded corners] (12){$u$}
 %        (4.021, -0.543) node[fill=orange, rounded corners] (13){$u$}
 %        (4.021, -1.628) node[fill=orange, rounded corners] (14){$u$}
 %        (4.021, -2.713) node[fill=orange, rounded corners] (15){$u$}
 %        (7.277, 0.0) node[fill=cyan, rounded corners] (16){Target};
 %      \begin{scope}[-]
 %        \draw[lightgray, text=black, font=\footnotesize] (0) to node[] {0} (1);
 %        \draw[lightgray, text=black, font=\footnotesize] (0) to node[] {0} (2);
 %        \draw[lightgray, text=black, font=\footnotesize] (0) to node[] {0} (3);
 %        \draw[lightgray, text=black, font=\footnotesize] (1) to node[] {0} (7);
 %        \draw[lightgray, text=black, font=\footnotesize] (2) to node[] {0} (8);
 %        \draw[transparent] (3) to node[] {0 ; 0} (8);
 %        \draw[lightgray, text=black, font=\footnotesize] (3) to node[] {0} (9);
 %        \draw[transparent] (3) to node[] {0 ; 0} (7);
 %        \draw[lightgray, text=black, font=\footnotesize] (4) to node[] {0} (8);
 %        \draw[transparent] (4) to node[] {0 ; 4} (14);
 %        \draw[lightgray, text=black, font=\footnotesize] (4) to node[] {2} (13);
 %        \draw[transparent] (4) to node[] {0 ; 5} (12);
 %        \draw[transparent] (4) to node[] {0 ; 4} (11);
 %        \draw[lightgray, text=black, font=\footnotesize] (5) to node[] {0} (9);
 %        \draw[transparent] (5) to node[] {0 ; 4} (14);
 %        \draw[transparent] (5) to node[] {0 ; 2} (13);
 %        \draw[lightgray, text=black, font=\footnotesize] (5) to node[] {2} (15);
 %        \draw[transparent] (5) to node[] {0 ; 3} (11);
 %        \draw[lightgray, text=black, font=\footnotesize] (6) to node[] {0} (7);
 %        \draw[lightgray, text=black, font=\footnotesize] (6) to node[] {1} (10);
 %        \draw[transparent] (6) to node[] {0 ; 1} (14);
 %        \draw[lightgray, text=black, font=\footnotesize] (10) to node[] {0} (16);
 %        \draw[transparent] (11) to node[] {0 ; 0} (16);
 %        \draw[transparent] (12) to node[] {0 ; 0} (16);
 %        \draw[lightgray, text=black, font=\footnotesize] (13) to node[] {0} (16);
 %        \draw[transparent] (14) to node[] {0 ; 0} (16);
 %        \draw[lightgray, text=black, font=\footnotesize] (15) to node[] {0} (16);
 %      \end{scope}
 %    \end{tikzpicture}

\resizebox{0.7\textwidth}{!}{%
      \begin{tikzpicture} [xscale=0.45, yscale=1.05]

      \tikzstyle{WC_style} = [circle, draw, fill, minimum size=45pt, inner sep=0pt, text centered, text width=1.5cm, align=center, font=\scriptsize]
        \tikzstyle{style} = [circle, draw, ultra thick, minimum size=45pt, inner sep=0pt, text centered]
            \tikzstyle{no_border} = [circle, ultra thick, minimum size=45pt, inner sep=0pt, text centered, opacity=0.5]
        
            \draw
              (-20.0, 0.0) node[draw=black,style] (0){Source}
              (-12.0, 2.5) node[draw=purple, loosely dashdotted, style, text width=1.5cm, align=center] (1){$people$ $with$ $low$ \\ $vision$}
              (-12.0, 0.0) node[draw=purple, loosely dashdotted, style, text width=1.5cm, align=center] (2){$people$ $with$ $low$ \\ $mobility$}
              (-12.0, -2.5) node[draw=lightgray, loosely dotted,style] (3){$wc$}
              (4.0, 5.0) node[draw=red, dashed,style] (4){$worker_1$}
              (4.0, 3.0) node[draw=red, dashed,no_border] (5){$worker_2$}
              (4.0, 1.0) node[draw=red, dashed,no_border] (6){$worker_3$}
              (4.0, -1.0) node[draw=red, dashed,style] (7){$worker_4$}
              (4.0, -3.0) node[draw=red, dashed,no_border] (8){$worker_5$}
              (4.0, -5.0) node[draw=red, dashed,style] (9){$worker_6$}
              (-4.0, 5.0) node[draw=green!50!black, dashdotted,style] (10){$p_1$}
              (-4.0, 3.0) node[draw=green!50!black, dashdotted,no_border] (11){$p_2$}
              (-4.0, 1.0) node[draw=green!50!black, dashdotted,no_border] (12){$p_3$}
              (-4.0, -1.0) node[draw=green!50!black, dashdotted,style] (13){$p_4$}
              (-4.0, -3.0) node[draw=green!50!black, dashdotted,style] (14){$p_5$}
              (-4.0, -5.0) node[draw=green!50!black, dashdotted,no_border] (15){$p_6$}
              (12.0, 3.0) node[draw=blue, dotted,style] (16){$position_1$}
              (12.0, -3.0) node[draw=blue, dotted,style] (17){$position_3$}
              (12.0, 0.0) node[draw=blue, dotted,style] (18){$position_2$}
              (20.0, 0.0) node[draw=black,style] (19){Sink};
        
      \begin{scope}[-]
        \draw[lightgray, text=black, font=\footnotesize] (0) to node[] {1 ; 0} (1);
        \draw[lightgray, text=black, font=\footnotesize] (0) to node[] {1 ; 0} (2);
        \draw[lightgray, text=black, font=\footnotesize] (0) to node[] {1 ; 0} (3);
        \draw[lightgray, text=black, font=\footnotesize] (1) to node[] {1 ; 0} (10);
        \draw[transparent] (1) to node[] {1 ; 0} (11);
        \draw[transparent] (1) to node[] {0 ; 0} (12);
        \draw[transparent] (2) to node[] {0 ; 0} (13);
        \draw[lightgray, text=black, font=\footnotesize] (2) to node[] {1 ; 0} (14);
        \draw[transparent] (3) to node[] {1 ; 0} (10);
        \draw[transparent] (3) to node[] {0 ; 0} (11);
        \draw[transparent] (3) to node[] {0 ; 0} (12);
        \draw[lightgray, text=black, font=\footnotesize] (3) to node[] {1 ; 0} (13);
        \draw[transparent] (3) to node[] {0 ; 0} (15);
        \draw[transparent] (3) to node[] {0 ; 0} (14);
        \draw[lightgray, text=black, font=\footnotesize] (4) to node[] {1 ; 0} (10);
        \draw[lightgray, text=black, font=\footnotesize] (4) to node[] {1 ; 1} (16);
        \draw[transparent] (5) to node[] {1 ; 0} (11);
        \draw[transparent] (5) to node[] {1 ; 7} (18);
        \draw[transparent] (6) to node[] {0 ; 0} (12);
        \draw[transparent] (6) to node[] {0 ; 7} (18);
        \draw[lightgray, text=black, font=\footnotesize] (7) to node[] {1 ; 0} (13);
        \draw[lightgray, text=black, font=\footnotesize] (7) to node[] {1 ; 1} (17);
        \draw[transparent] (8) to node[] {0 ; 0} (15);
        \draw[transparent] (8) to node[] {0 ; 10} (18);
        \draw[transparent] (8) to node[] {0 ; 10} (17);
        \draw[lightgray, text=black, font=\footnotesize] (9) to node[] {1 ; 0} (14);
        \draw[lightgray, text=black, font=\footnotesize] (9) to node[] {1 ; 10} (18);
        \draw[lightgray, text=black, font=\footnotesize] (16) to node[] {1 ; 0} (19);
        \draw[lightgray, text=black, font=\footnotesize] (17) to node[] {1 ; 0} (19);
        \draw[lightgray, text=black, font=\footnotesize] (18) to node[] {1 ; 0} (19);
      \end{scope}
    \end{tikzpicture}
    }%
                \caption[Example of a solution for mapping workers to jobs in Figure \ref{fig:workers_jobs_unsolved}.]{Example of a solution for mapping workers to jobs in Figure \ref{fig:workers_jobs_unsolved}. In this example each quota has at least one assemble. Unselected workers and their proxies are faded out.}
                \label{fig:workers_jobs_solved}
            \end{subfigure}
            \vspace{1em}
            \begin{subfigure}[t]{0.8\textwidth}
                \centering
                
\resizebox{0.7\textwidth}{!}{%
      \begin{tikzpicture} [xscale=0.45, yscale=1.05]

      \tikzstyle{WC_style} = [circle, draw, fill, minimum size=45pt, inner sep=0pt, text centered, text width=1.5cm, align=center, font=\scriptsize]
        \tikzstyle{style} = [circle, draw, ultra thick, minimum size=45pt, inner sep=0pt, text centered]
            \tikzstyle{no_border} = [circle, ultra thick, minimum size=45pt, inner sep=0pt, text centered, opacity=0.5]
        
            \draw
              (-20.0, 0.0) node[draw=black,style] (0){Source}
              (-12.0, 2.5) node[draw=purple, loosely dashdotted, style, text width=1.5cm, align=center] (1){$people$ $with$ $low$ \\ $vision$}
              (-12.0, 0.0) node[draw=purple, loosely dashdotted, style, text width=1.5cm, align=center] (2){$people$ $with$ $low$ \\ $mobility$}
              (-12.0, -2.5) node[draw=lightgray, loosely dotted,no_border] (3){$wc$}
              (4.0, 5.0) node[draw=red, dashed,style] (4){$worker_1$}
              (4.0, 3.0) node[draw=red, dashed,no_border] (5){$worker_2$}
              (4.0, 1.0) node[draw=red, dashed,no_border] (6){$worker_3$}
              (4.0, -1.0) node[draw=red, dashed,style] (7){$worker_4$}
              (4.0, -3.0) node[draw=red, dashed,no_border] (8){$worker_5$}
              (4.0, -5.0) node[draw=red, dashed,style] (9){$worker_6$}
              (-4.0, 5.0) node[draw=green!50!black, dashdotted,style] (10){$p_1$}
              (-4.0, 3.0) node[draw=green!50!black, dashdotted,no_border] (11){$p_2$}
              (-4.0, 1.0) node[draw=green!50!black, dashdotted,no_border] (12){$p_3$}
              (-4.0, -1.0) node[draw=green!50!black, dashdotted,style] (13){$p_4$}
              (-4.0, -3.0) node[draw=green!50!black, dashdotted,style] (14){$p_5$}
              (-4.0, -5.0) node[draw=green!50!black, dashdotted,no_border] (15){$p_6$}
              (12.0, 3.0) node[draw=blue, dotted,style] (16){$position_1$}
              (12.0, -3.0) node[draw=blue, dotted,style] (17){$position_3$}
              (12.0, 0.0) node[draw=blue, dotted,style] (18){$position_2$}
              (20.0, 0.0) node[draw=black,style] (19){Sink};
        
      \begin{scope}[->, >=BigLatex]
        \draw[lightgray, text=black, font=\footnotesize] (0) to node[] {1 ; 0} (1);
        \draw[lightgray, text=black, font=\footnotesize] (0) to node[] {2 ; 0} (2);
        \draw[transparent] (0) to node[] {2 ; 0} (3);
        \draw[lightgray, text=black, font=\footnotesize] (1) to node[] {1 ; 0} (10);
        \draw[transparent] (1) to node[] {1 ; 0} (11);
        \draw[transparent] (1) to node[] {0 ; 0} (12);
        \draw[transparent] (2) to node[] {0 ; 0} (13);
        \draw[transparent] (2) to node[] {1 ; 0} (14);
        \draw[transparent] (3) to node[] {1 ; 0} (10);
        \draw[transparent] (3) to node[] {0 ; 0} (11);
        \draw[transparent] (2) to node[] {1 ; 0} (12);
        \draw[lightgray, text=black, font=\footnotesize] (2) to node[] {1 ; 0} (13);
        \draw[transparent] (3) to node[] {0 ; 0} (15);
        \draw[lightgray, text=black, font=\footnotesize] (2) to node[] {1 ; 0} (14);
        \draw[lightgray, text=black, font=\footnotesize] (10) to node[] {1 ; 0} (4);
        \draw[lightgray, text=black, font=\footnotesize] (4) to node[] {1 ; 1} (16);
        \draw[transparent] (5) to node[] {1 ; 0} (11);
        \draw[transparent] (5) to node[] {1 ; 7} (18);
        \draw[transparent] (6) to node[] {0 ; 0} (12);
        \draw[transparent] (6) to node[] {0 ; 7} (18);
        \draw[lightgray, text=black, font=\footnotesize] (13) to node[] {1 ; 0} (7);
        \draw[lightgray, text=black, font=\footnotesize] (7) to node[] {1 ; 1} (17);
        \draw[transparent] (8) to node[] {0 ; 0} (15);
        \draw[transparent] (8) to node[] {0 ; 10} (18);
        \draw[transparent] (8) to node[] {0 ; 10} (17);
        \draw[lightgray, text=black, font=\footnotesize] (14) to node[] {1 ; 0} (9);
        \draw[lightgray, text=black, font=\footnotesize] (9) to node[] {1 ; 10} (18);
        \draw[lightgray, text=black, font=\footnotesize] (16) to node[] {1 ; 0} (19);
        \draw[lightgray, text=black, font=\footnotesize] (17) to node[] {1 ; 0} (19);
        \draw[lightgray, text=black, font=\footnotesize] (18) to node[] {1 ; 0} (19);
      \end{scope}
    \end{tikzpicture}
    }%
                \caption[Example of a solution for mapping workers to jobs in Figure \ref{fig:workers_jobs_unsolved}.]{Example of a solution for mapping workers to jobs in Figure \ref{fig:workers_jobs_unsolved}. In this example the `people with low mobility' quota has a minimum of 2 assembles, thus no wide competition is assembled. Unselected workers and their proxies are faded out.}
                \label{fig:workers_jobs_solved_no_wc}
            \end{subfigure}
            \caption{Examples of worker to positions mapping and their solutions. (a) shows the initial mapping, (b) presents a solution with at least one assemble per quota, and (c) shows a solution where no wide competition is assembled.}
            \label{fig:workers_jobs_example}
        \end{figure}
        
         
        The result that can be seen in Figure {\ref{fig:workers_jobs_solved}} is a minimum-cost solution that maximizes the diversity of attributes among the selected workers, ensuring both fairness and efficiency in the team formation process. 
        
        \subsection{Fairness in server to services allocation}
        
        In the second example, we address a resource allocation problem within a network of services belonging to a single application. The goal is to assign these services to servers, each of which possesses distinct attributes, such as geolocation and backup methods. The objective is to ensure the most diverse and optimal distribution of services across the available servers. An evaluation process assigns costs based on factors like latency, representing the suitability of each service-server pair. This cost structure ensures that resources are allocated in a way that balances fairness and efficiency while maximizing server characteristics that minimize overall costs.

        \begin{figure}[]
            \centering
            \begin{subfigure}[t]{0.8\textwidth}
            \centering
            \resizebox{0.7\textwidth}{!}{%
  \begin{tikzpicture} [xscale=0.45, yscale=1.05]
    \tikzstyle{style} = [circle, draw, ultra thick, minimum size=45pt, inner sep=0pt, text centered]
    \tikzstyle{no_border} = [circle, ultra thick, minimum size=45pt, inner sep=0pt, text centered, opacity=0.5]

    \draw
      (-20.0, 0.0) node[draw=black,style] (0){Source}
      (-12.0, 2.5) node[draw=purple, loosely dashdotted, style, text width=1.5cm, align=center] (1){$diverse$ $geolocation$}
      (-12.0, 0.0) node[draw=purple, loosely dashdotted, style, text width=1.5cm, align=center] (2){$robust$ $backup$}
      (-12.0, -2.5) node[draw=lightgray,loosely dotted,style] (3){$wc$}
      (4.0, 5.0) node[draw=red, dashed,style] (4){$server_1$}
      (4.0, 3.0) node[draw=red, dashed,style] (5){$server_2$}
      (4.0, 1.0) node[draw=red, dashed,style] (6){$server_3$}
      (4.0, -1.0) node[draw=red, dashed,style] (7){$server_4$}
      (4.0, -3.0) node[draw=red, dashed,style] (8){$server_5$}
      (4.0, -5.0) node[draw=red, dashed,style] (9){$server_6$}
      (-4.0, 5.0) node[draw=green!50!black, dashdotted,style] (10){$p_1$}
      (-4.0, 3.0) node[draw=green!50!black, dashdotted,style] (11){$p_2$}
      (-4.0, 1.0) node[draw=green!50!black, dashdotted,style] (12){$p_3$}
      (-4.0, -1.0) node[draw=green!50!black, dashdotted,style] (13){$p_4$}
      (-4.0, -3.0) node[draw=green!50!black, dashdotted,style] (14){$p_5$}
      (-4.0, -5.0) node[draw=green!50!black, dashdotted,style] (15){$p_6$}
      (12.0, 3.0) node[draw=blue, dotted,style] (16){$service_1$}
      (12.0, -3.0) node[draw=blue, dotted,style] (18){$service_2$}
      (20.0, 0.0) node[draw=black,style] (19){Sink};

    \begin{scope}[-]
      \draw[lightgray, text=black, font=\footnotesize] (0) to node[] {$n_{SDG}$ ; 0} (1);
      \draw[lightgray, text=black, font=\footnotesize] (0) to node[] {$n_{RB}$ ; 0} (2);
      \draw[lightgray, text=black, font=\footnotesize] (0) to node[] {$n_{wc}$ ; 0} (3);
      \draw[lightgray, text=black, font=\footnotesize] (1) to node[] {1 ; 0} (10);
      \draw[lightgray, text=black, font=\footnotesize] (1) to node[] {1 ; 0} (11);
      \draw[lightgray, text=black, font=\footnotesize] (1) to node[] {1 ; 0} (12);
      % Adjusted vertical spacing here
      \draw[lightgray, text=black, font=\footnotesize] (2) to node[] {1 ; 0} (13);
      \draw[lightgray, text=black, font=\footnotesize] (2) to node[] {1 ; 0} (14);
      \draw[lightgray, text=black, font=\footnotesize] (3) to node[] {1 ; 0} (10);
      \draw[lightgray, text=black, font=\footnotesize] (3) to node[] {1 ; 0} (11);
      \draw[lightgray, text=black, font=\footnotesize] (3) to node[] {1 ; 0} (12);
      \draw[lightgray, text=black, font=\footnotesize] (3) to node[] {1 ; 0} (13);
      \draw[lightgray, text=black, font=\footnotesize] (3) to node[] {1 ; 0} (15);
      \draw[lightgray, text=black, font=\footnotesize] (3) to node[] {1 ; 0} (14);
      \draw[lightgray, text=black, font=\footnotesize] (4) to node[] {1 ; 0} (10);
      \draw[lightgray, text=black, font=\footnotesize] (4) to node[] {1 ; $C_{SVE_1 SVI_1}$} (16);
      \draw[lightgray, text=black, font=\footnotesize] (5) to node[] {1 ; 0} (11);
      \draw[lightgray, text=black, font=\footnotesize] (5) to node[] {1 ; $C_{SVE_2 SVI_2}$} (18);
      \draw[lightgray, text=black, font=\footnotesize] (6) to node[] {1 ; 0} (12);
      \draw[lightgray, text=black, font=\footnotesize] (6) to node[] {1 ; $C_{SVE_3 SVI_2}$} (18);
      \draw[lightgray, text=black, font=\footnotesize] (7) to node[] {1 ; 0} (13);
      \draw[lightgray, text=black, font=\footnotesize] (7) to node[] {1 ; $C_{SVE_4 SVI_2}$} (18);
      \draw[lightgray, text=black, font=\footnotesize] (9) to node[] {1 ; 0} (15);
      \draw[lightgray, text=black, font=\footnotesize] (8) to node[] {1 ; $C_{SVE_5 SVI_2}$} (18);
      \draw[lightgray, text=black, font=\footnotesize] (8) to node[] {1 ; $C_{SVE_5 SVI_2}$} (18);
      \draw[lightgray, text=black, font=\footnotesize] (8) to node[] {1 ; 0} (14);
      \draw[lightgray, text=black, font=\footnotesize] (9) to node[] {1 ; $C_{SVE_6 SVI_2}$} (18);
      \draw[lightgray, text=black, font=\footnotesize] (16) to node[] {1 ; 0} (19);
      \draw[lightgray, text=black, font=\footnotesize] (17) to node[] {1 ; 0} (19);
      \draw[lightgray, text=black, font=\footnotesize] (18) to node[] {1 ; 0} (19);
    \end{scope}
  \end{tikzpicture}
}%

            \caption{Server to services mapping.}
            \label{fig:server_services_unsolved}
            \end{subfigure}
            \vspace{1em}
            \begin{subfigure}[t]{0.8\textwidth}
            \centering
             % \begin{tikzpicture}[scale=0.8]
 %      \draw
 %        (-9.0, 0.0) node[fill=cyan, rounded corners] (0){Source}
 %        (-5.745, 1.085) node[fill=pink, rounded corners] (1){Cota}
 %        (-5.745, 0.0) node[fill=pink, rounded corners] (2){Cota}
 %        (-5.745, -1.085) node[fill=gray, rounded corners] (3){Resto}
 %        (0.766, 0.0) node[fill=brown, rounded corners] (4){$u$}
 %        (0.766, -1.085) node[fill=brown, rounded corners] (5){$u$}
 %        (0.766, 1.085) node[fill=brown, rounded corners] (6){$u$}
 %        (-2.489, 1.085) node[fill=yellow, rounded corners] (7){$p$}
 %        (-2.489, 0.0) node[fill=yellow, rounded corners] (8){$p$}
 %        (-2.489, -1.085) node[fill=yellow, rounded corners] (9){$p$}
 %        (4.021, 2.713) node[fill=orange, rounded corners] (10){$u$}
 %        (4.021, 1.628) node[fill=orange, rounded corners] (11){$u$}
 %        (4.021, 0.543) node[fill=orange, rounded corners] (12){$u$}
 %        (4.021, -0.543) node[fill=orange, rounded corners] (13){$u$}
 %        (4.021, -1.628) node[fill=orange, rounded corners] (14){$u$}
 %        (4.021, -2.713) node[fill=orange, rounded corners] (15){$u$}
 %        (7.277, 0.0) node[fill=cyan, rounded corners] (16){Target};
 %      \begin{scope}[-]
 %        \draw[lightgray, text=black, font=\footnotesize] (0) to node[] {0} (1);
 %        \draw[lightgray, text=black, font=\footnotesize] (0) to node[] {0} (2);
 %        \draw[lightgray, text=black, font=\footnotesize] (0) to node[] {0} (3);
 %        \draw[lightgray, text=black, font=\footnotesize] (1) to node[] {0} (7);
 %        \draw[lightgray, text=black, font=\footnotesize] (2) to node[] {0} (8);
 %        \draw[transparent] (3) to node[] {0 ; 0} (8);
 %        \draw[lightgray, text=black, font=\footnotesize] (3) to node[] {0} (9);
 %        \draw[transparent] (3) to node[] {0 ; 0} (7);
 %        \draw[lightgray, text=black, font=\footnotesize] (4) to node[] {0} (8);
 %        \draw[transparent] (4) to node[] {0 ; 4} (14);
 %        \draw[lightgray, text=black, font=\footnotesize] (4) to node[] {2} (13);
 %        \draw[transparent] (4) to node[] {0 ; 5} (12);
 %        \draw[transparent] (4) to node[] {0 ; 4} (11);
 %        \draw[lightgray, text=black, font=\footnotesize] (5) to node[] {0} (9);
 %        \draw[transparent] (5) to node[] {0 ; 4} (14);
 %        \draw[transparent] (5) to node[] {0 ; 2} (13);
 %        \draw[lightgray, text=black, font=\footnotesize] (5) to node[] {2} (15);
 %        \draw[transparent] (5) to node[] {0 ; 3} (11);
 %        \draw[lightgray, text=black, font=\footnotesize] (6) to node[] {0} (7);
 %        \draw[lightgray, text=black, font=\footnotesize] (6) to node[] {1} (10);
 %        \draw[transparent] (6) to node[] {0 ; 1} (14);
 %        \draw[lightgray, text=black, font=\footnotesize] (10) to node[] {0} (16);
 %        \draw[transparent] (11) to node[] {0 ; 0} (16);
 %        \draw[transparent] (12) to node[] {0 ; 0} (16);
 %        \draw[lightgray, text=black, font=\footnotesize] (13) to node[] {0} (16);
 %        \draw[transparent] (14) to node[] {0 ; 0} (16);
 %        \draw[lightgray, text=black, font=\footnotesize] (15) to node[] {0} (16);
 %      \end{scope}
 %    \end{tikzpicture}

\resizebox{0.7\textwidth}{!}{%
      \begin{tikzpicture} [xscale=0.45, yscale=1.05]
        \tikzstyle{style} = [circle, draw, ultra thick, minimum size=45pt, inner sep=0pt, text centered]
    \tikzstyle{no_border} = [circle, ultra thick, minimum size=45pt, inner sep=0pt, text centered, opacity=0.5]

    \draw
      (-20.0, 0.0) node[draw=black,style] (0){Source}
      (-12.0, 2.5) node[draw=purple, loosely dashdotted, style, text width=1.5cm, align=center] (1){$diverse$ $geolocation$}
      (-12.0, 0.0) node[draw=purple, loosely dashdotted, style, text width=1.5cm, align=center] (2){$robust$ $backup$}
      (-12.0, -2.5) node[draw=lightgray,loosely dotted,no_border] (3){$wc$}
      (4.0, 5.0) node[draw=red, dashed,style] (4){$server_1$}
      (4.0, 3.0) node[draw=red, dashed,no_border] (5){$server_2$}
      (4.0, 1.0) node[draw=red, dashed,no_border] (6){$server_3$}
      (4.0, -1.0) node[draw=red, dashed,no_border] (7){$server_4$}
      (4.0, -3.0) node[draw=red, dashed,no_border] (8){$server_5$}
      (4.0, -5.0) node[draw=red, dashed,style] (9){$server_6$}
      (-4.0, 5.0) node[draw=green!50!black, dashdotted,style] (10){$p_1$}
      (-4.0, 3.0) node[draw=green!50!black, dashdotted,no_border] (11){$p_2$}
      (-4.0, 1.0) node[draw=green!50!black, dashdotted,no_border] (12){$p_3$}
      (-4.0, -1.0) node[draw=green!50!black, dashdotted,no_border] (13){$p_4$}
      (-4.0, -3.0) node[draw=green!50!black, dashdotted,style] (14){$p_5$}
      (-4.0, -5.0) node[draw=green!50!black, dashdotted,no_border] (15){$p_6$}
      (12.0, 3.0) node[draw=blue, dotted,style] (16){$service_1$}
      (12.0, -3.0) node[draw=blue, dotted,style] (18){$service_2$}
      (20.0, 0.0) node[draw=black,style] (19){Sink};
      
      \begin{scope}[-]
        \draw[lightgray, text=black, font=\footnotesize] (0) to node[] {1 ; 0} (1);
        \draw[lightgray, text=black, font=\footnotesize] (0) to node[] {1 ; 0} (2);
        \draw[transparent] (0) to node[] {1 ; 0} (3);
        \draw[lightgray, text=black, font=\footnotesize] (1) to node[] {1 ; 0} (10);
        \draw[transparent] (1) to node[] {1 ; 0} (11);
        \draw[transparent] (1) to node[] {0 ; 0} (12);
        \draw[transparent] (2) to node[] {0 ; 0} (13);
        \draw[lightgray, text=black, font=\footnotesize] (2) to node[] {1 ; 0} (14);
        \draw[transparent] (3) to node[] {1 ; 0} (10);
        \draw[transparent] (3) to node[] {0 ; 0} (11);
        \draw[transparent] (3) to node[] {0 ; 0} (12);
        \draw[transparent] (3) to node[] {1 ; 0} (13);
        \draw[transparent] (3) to node[] {0 ; 0} (15);
        \draw[transparent] (3) to node[] {0 ; 0} (14);
        \draw[lightgray, text=black, font=\footnotesize] (4) to node[] {1 ; 0} (10);
        \draw[lightgray, text=black, font=\footnotesize] (4) to node[] {1 ; 1} (16);
        \draw[transparent] (5) to node[] {1 ; 0} (11);
        \draw[transparent] (5) to node[] {1 ; 7} (18);
        \draw[transparent] (6) to node[] {0 ; 0} (12);
        \draw[transparent] (6) to node[] {0 ; 7} (18);
        \draw[transparent] (7) to node[] {1 ; 0} (13);
        \draw[transparent] (7) to node[] {1 ; 1} (17);
        \draw[transparent] (8) to node[] {0 ; 0} (15);
        \draw[transparent] (8) to node[] {0 ; 10} (18);
        \draw[transparent] (8) to node[] {0 ; 10} (17);
        \draw[lightgray, text=black, font=\footnotesize] (9) to node[] {1 ; 0} (14);
        \draw[lightgray, text=black, font=\footnotesize] (9) to node[] {1 ; 10} (18);
        \draw[lightgray, text=black, font=\footnotesize] (16) to node[] {1 ; 0} (19);
        \draw[lightgray, text=black, font=\footnotesize] (17) to node[] {1 ; 0} (19);
        \draw[lightgray, text=black, font=\footnotesize] (18) to node[] {1 ; 0} (19);
      \end{scope}
    \end{tikzpicture}
    }%
            \caption{Solution to server to services mapping.}
            \label{fig:server_services_solved}
            \end{subfigure}
            \caption{Example of server to services allocation. (a) shows the initial mapping, and (b) presents the solution.}
            \label{fig:server_services_example}
        \end{figure}
        
        Ultimately, this approach leads to optimization of resource allocation by prioritizing diverse server attributes and minimizing latency costs, achieving an efficient and fair solution.


\chapter{Framework} \label{chap:framework}

    Building upon the ideas presented in Chapter \ref{chap:design}, we outline how an prototype Python framework was implemented. In this chapter, we demonstrate the construction process, focusing on a problem-agnostic approach. To maintain generality, some classes have been simplified, allowing us to emphasize the framework's core design principles rather than problem-specific details.
    The implementation of the framework is available on GitHub\footnote{\url{https://github.com/RafaelGranza/Matching-Optimization-Framework}} \cite{myframework}.

    \section{Representing Relations}
        % To represent the relations, we provide the class         \textit{GroupingRule}, which stores each classes that can be part of a Grouping, the specifications, and also represents the geometry of the Grouping, i.e. how many relations each object can be part of, can ahve and so on. 
        To represent the rules of a grouping, as defined in Section \ref{sec:relation_rules}, we provide the class \textit{GroupRule} (see Script \ref{script:groupingrule}).
        The \textit{GroupRule} provides methods to define the classes that can be part of a \textit{Group}, the specifications, and also represents the rules of a specific \textit{Group}, as defined in Script \ref{script:groupingrule}.
        %
        In Section \ref{sec:examples}, we will provide examples of how to create these rules using the \textit{GroupRule} class.

        \begin{lstlisting}[language=Python, caption={\textit{GroupRule} class that defines valid \textit{Groups} and its specifications. Such as min/max cardinality of each relation, group validation functions, statistic functions, and object function.}, label={script:groupingrule}]
class GroupRule:
    # Sets the objective function (string or callable).
    def set_objective_function(self, function):
       # omitted code

    # Add classes to the group's relation and defines its min/max cardinality relation.
    def set_cardinality(self, cls, min_count, max_count):
       # omitted code

    # Adds a validator function for group validation.
    def add_validator(self, validator_fn):
       # omitted code

    # Adds a statistic function for the grouping.
    def add_statistic(self, statistic):
       # omitted code

    # Validates group, raises exception if invalid.
    def validate_or_raise(self, group):
       # omitted code

    # Validates group(s), returns True if valid.
    def validate(self, groups)
       # omitted code

    # Enables or disables stable matching.
    def set_stable_match(self, stable_match):
       # omitted code
\end{lstlisting}

    Note that the \textit{GroupRule} class supports defining cardinality constraints, validation functions, and statistics functions that can be used to evaluate the quality of the grouping.
    The validator functions are used to ensure that the grouping adheres to the specified rules, while the statistics functions can be used to gather information about the grouping, such as the number of members in each class or the average cost of the grouping.

    %One can see some examples of how to create a \textit{GroupingRule} at Section \ref{sec:examples}.
    The \textit{Group} class is a flexible container for objects, as demonstrated in Script \ref{script:grouping}.
    It represents the elements (instances) that are grouped together, i.e., members of a group.
    Importantly, these objects can be instances of any class, and they are organized by their class type.

    \begin{lstlisting}[language=Python, caption={\textit{Group} Class. It represents the elements (instances) that are grouped together.}, label={script:grouping}]
class Group:
    """
    A container of members, which can be any type of object.
    Members are organized by their class type.
    """

    # Adds a member to the group, allowing multiple instances.
    def add_member(self, *instances):
        # omitted code

    # Removes a member from the group, allowing multiple instances.
    def remove_member(self, *instances):
        # omitted code

    # Retrieves a dictionary of members divided by class type, optionally filtered by only one class type.
    def get_members(self, cls=None):
        # omitted code

    # Retrieves members of the group as a list, optionally filtered by class type.
    def get_members_as_list(self, cls=None):
       # omitted code

    # Returns a string representation of the group, showing all members and their types.
    def __repr__(self):
       # omitted code
    \end{lstlisting}

    The complete implementations of Script \ref{script:groupingrule} and Script \ref{script:grouping} can be found at Appendix \ref{app:group_source_code}.

    \section{Solvers}
        As described in Section \ref{sec:solvers}, the framework is designed to accommodate various solvers, each tailored to specific grouping problems. These solvers are implemented as classes that inherit from the abstract class \textit{Solver} (see Script \ref{script:solver}).
        Different \textit{Solvers} work with very different problems, and each problem may be solved by more than one available solver.
        Therefore, we provide an assigner to select the suitable solver for each problem.
        %and for the purpose of this work, they are enough to demonstrate the efficiency of the design.

        \subsection{Assigning Solvers}
            The Script \ref{script:assinger} implements a simple solver assignment strategy equivalent to the flow-chart of Figure \ref{fig:flow_chart}.
            Returns the first solver capable of handling the problem from a priority list.
            The high-priority algorithms are the deterministic ones.
            %The assigner gives the responsability to the solvers. 
            %They ask whether each solver can solve and specific problem as shown in Script \ref{script:assinger}.
            
    \begin{lstlisting}[language=Python, caption={Assigner Class.}, label={script:assinger}]
class Assigner:

    solvers: List[Solver] = [
        HungarianAlgorithm,
        BinSearchHungarianAlgorithm,
        StableMarriage,
        FairBipartiteMatching,
        GeneticAlgorithm, # Metaheuristic solver
        # Add more solvers here as needed (e.g., SimulatedAnnealing)
    ]

    def choose_solver(self, group_rule: GroupRule) -> Solver:
        """
        Choose the best solver based on the group members and their cardinality.
        """
        for solver in self.solvers:
            if solver.can_solve(group_rule):
                return solver
        raise ValueError("No suitable solver found for the given group.")

    def add_solver(self, solver: Type[Solver]):
        """
        Add a new solver to the list of available solvers.
        """
        self.solvers.append(solver)
    
    def remove_solver(self, solver: Type[Solver]):
        """
        Remove a solver from the list of available solvers.
        """
        self.solvers.remove(solver)
\end{lstlisting}

            The proposed design ensures a seamless integration of new solvers, as the responsibility of verifying whether a solver can handle a given problem lies within the solver itself. The order in which solvers are added, and consequently the sequence in which they are evaluated, dictates the preference when multiple solvers are capable of addressing the same problem.

            The assigner iterates through a predefined list of solvers, selecting the first one that successfully resolves the given problem. The Genetic Algorithm is positioned as the last resort, ensuring that it is only employed when no specialized solver is available.

        \subsection{Adding Solvers}
            A new solver can be implemented by creating a class that inherits from \textit{Solver} (see Script \ref{script:solver}) and by overriding the necessary virtual functions.
            %
            Once implemented, the solver must be added to the assigner's list, as demonstrated in Script \ref{script:assinger}.
            %
            In Section \ref{sec:implemented_solvers}, four default solvers are introduced.

\begin{lstlisting}[language=Python, caption={Solver Class.}, label={script:solver}]
class Solver:
    """
    Abstract class for solvers. All solvers should inherit from this class.
    """

    @staticmethod
    def can_solve(group_rule: GroupRule):
        """
        Check if the solver can solve the given group_rule.
        """
        raise NotImplementedError("Subclasses should implement this method.")

    @staticmethod
    def solve(group_rule: GroupRule, instances=List[object]):
        """
        Solve the given group.
        """
        raise NotImplementedError("Subclasses should implement this method.")
    
    @staticmethod
    def solve(group_rule: GroupRule, groups: List[Group]):
        """
        Solve the given group.
        """
        raise NotImplementedError("Subclasses should implement this method.")
    \end{lstlisting}

    \subsection{Implemented Solvers}
    \label{sec:implemented_solvers}
    In this section, we describe the different solvers that were implemented to address the grouping problems. Each solver is chosen based on its suitability for the problem at hand, ranging from optimization-based techniques like the Hungarian Algorithm to more heuristic-based approaches such as Genetic Algorithms. We also delve into a solver designed for specific cases like the Stable Marriage problem, which ensures a stable matching between two groups.

    It is important to notice that for this work, it is enough to demonstrate possibilities, indicating an arbitrary amount of solvers can be implemented and added to this framework in the future.
    The solvers implemented in this framework are available at Appendices \ref{app:solver_hungarian}, \ref{app:solver_binary_search}, \ref{app:solver_stable_marriage}, \ref{app:solver_ga}, and \ref{app:solver_fair}.

    \subsubsection{Metaheuristic}
        To implement the genetic algorithm, we utilized the DEAP library \cite{deap}, which provides a flexible framework for evolutionary algorithms in Python.
        In this implementation, individuals are represented as lists of booleans, and for matrix-based problems, we use a compressed matrix representation as a flat list.

        DEAP offers customizable genetic operators and supports user-defined fitness functions, allowing constraints to be incorporated directly. The default mutation and crossover operators from DEAP are used, enabling efficient exploration of large search spaces. This solver is particularly suitable for complex matching problems where traditional optimization algorithms are impractical and approximate solutions are sufficient.
    \subsubsection{Hungarian Algorithm}
    The Hungarian Algorithm, also known as the Kuhn-Munkres algorithm, was implemented to solve assignment problems where the goal is to find an optimal, minimum-cost matching between two sets of objects. The algorithm operates in polynomial time, making it an efficient choice for solving balanced bipartite matching problems \cite{kuhn1955hungarian}. 
    
    In this implementation, we followed the standart approach of constructing a cost matrix, where each element represents the "cost" or "penalty" of matching a pair of objects. The algorithm then attempts to minimize the total cost by identifying the optimal set of pairings that leads to the lowest overall value. This is particularly suited for scenarios where the goal is not only to match but to solve a single objective, such as minimizing total distance or maximizing total compensation in the distribution of resources. The Hungarian Algorithm is deterministic, guaranteeing an optimal solution for these types of problems.

    \subsubsection{Binary Search and Hungarian Algorithm}

        The combination of binary search and the Hungarian Algorithm provides an efficient approach to solving matching problems where constraints on edge rankings play a crucial role. This method is particularly useful when the objective is to find a maximum matching while ensuring that the lowest-ranked edge in the matching is as high as possible.
        
        The approach consists of performing a binary search on the minimum allowable edge ranking. For a given threshold value \( T \), all edges with rankings lower than \( T \) are temporarily removed from consideration, leaving only edges with rankings \(\geq T\). The Hungarian Algorithm is then applied to find the optimal matching under these constraints. The binary search then adjusts the threshold until the maximum possible minimum ranking is achieved while maintaining a valid matching.
        
        More formally, the algorithm follows these steps:
        
        \begin{enumerate}
            \item Define the search space by setting an initial range \([L, R]\), where \( L \) is the lowest ranking in the graph and \( R \) is the highest ranking.
            \item Perform binary search on \( T \):
            \begin{itemize}
                \item Set \( T = \frac{L + R}{2} \);
                \item Remove all edges with rankings lower than \( T \);
                \item Apply the Hungarian Algorithm to determine the maximum matching under the new graph constraints;
                \item If a valid matching exists, adjust \( L \) to try increasing \( T \); otherwise, adjust \( R \).
            \end{itemize}
            \item Repeat until convergence, returning the best feasible matching where the lowest-ranked edge is maximized.
        \end{enumerate}
        
        This technique ensures that the worst-case edge ranking in the matching is as high as possible, making it particularly useful in applications where fairness or quality guarantees are required. By leveraging binary search, the solution efficiently narrows down the feasible rankings, ensuring that the Hungarian Algorithm is executed only on relevant subsets of the graph, thereby improving computational efficiency in large-scale instances.
        
    
    \subsubsection{Stable Marriage Algorithm}
    The Stable Marriage Algorithm, based on the Gale-Shapley Deferred Acceptance algorithm, was implemented to solve matching problems where stability between two groups is the primary concern. Stability in this context refers to a matching where no pair of individuals would prefer each other over their current matches, thus preventing any potential disruptions to the arrangement.
    
    For the implementation, we focused on creating an efficient solution for the stable matching problem, where each group provides a preference list, and the algorithm iteratively matches individuals based on their rankings. We applied the "men-proposing" variant of the Stable Marriage Algorithm, ensuring that once a match is formed, it is either retained or improved upon without destabilizing previous matches.
    
    This algorithm is particularly effective in applications where stability are more important than optimization, such as job candidate assignment or college admissions. The result is a matching that respects individual preferences while maintaining stability across the entire system.

    \subsubsection{Fair Bipartite Matching}

    The theoretical foundations for this solver are presented in Chapter~\ref{chap:fair_matching}.  
    For the graph-related implementation, we used both the networkx and igraph libraries~\cite{hagberg2008exploring, igraph}, which offer tools for graph manipulation and for solving minimum-cost maximum-flow problems.


    \section{Solve}
    For the purpose of this framework, we provide a simple \textit{solve} function that selects the appropriate solver based on the problem's characteristics and solves the problem.
    The \textit{solve} function provides a direct interface for solving grouping problems.
    It accepts either a \textit{GroupRule} and a list of valid \textit{Groups}, or a \textit{GroupRule} and a list of instances.
    The function internally uses the \textit{Assigner} to select the appropriate solver and applies it to the problem, abstracting away the manual selection and invocation of solvers.
    See Script \ref{script:solve} for details.

    \begin{lstlisting}[language=Python, caption={Solve.}, label={script:solve}]
def solve(gr: GroupRule, instances: List[object]):
    assigner = Assigner()
    solver = assigner.choose_solver(gr)

    if all(isinstance(inst, Group) for inst in instances):
        return solver.solve_from_valid_groups(gr, instances)

    return solver.solve_from_instances(gr, instances)
\end{lstlisting}

    \section{Problem Examples} \label{sec:examples}
To illustrate the flexibility of the framework in solving various grouping problems, this section presents four distinct problems, each one with its unique requirements.
Note that whilst the classes and instances are manually created in the examples, in a real application this could come from a database or other data source, and the framework would still be able to handle it seamlessly.


\subsection{Job Assignment}
In this example, the goal is to match workers to jobs while minimizing a cost metric (e.g., skill) as described at Section \ref{subsec:job_assignment}.
In Script \ref{script:job_assignement}, we configure the framework with a cost function that evaluates the expense of assigning specific workers to jobs, with a one-to-one assignment.

\begin{lstlisting}[language=Python, caption={Creating and solving the Job Assignment Problem using this framework.}, label={script:job_assignement}]
from src.group import Group, GroupRule
import random
from typing import List, Type
import itertools
from src.solve import solve

# Define the Classes for the Example
class Worker:
    def __init__(self, name):
        self.name = name
        self.skills = random.randint(1, int(1e6)) # Simulating worker skill level
    def __repr__(self): return f"Worker({self.name})"
class Job:
    def __init__(self, title):
        self.title = title
        self.skills = random.randint(1, int(1e6)) # Simulating skills required
    def __repr__(self): return f"Job({self.title})"

# Define the Statistic Function
def skill_allignment(members: dict[Type, List]):
    workers = members.get(Worker, [None])[0]
    jobs = members.get(Job, [None])[0]
    return abs(workers.skills - jobs.skills)

# Create instances
workers = [
    Worker("Alice"), Worker("Bob"), Worker("Charlie"), Worker("Diana"),
    Worker("Eve"), Worker("Frank"), Worker("Grace"), Worker("Hank"),
    Worker("Ivy"), Worker("Jack"), Worker("Karen"), Worker("Leo"),
]
jobs = [
    Job("Painting"), Job("Driving"), Job("Plumbing"), Job("Cleaning"), Job("Gardening")
]

# Create Groups
groups = [Group().add_member(*comb) for comb in itertools.product(workers, jobs)]

gr = GroupRule()                   # Define the Group Rule
gr.set_cardinality(Worker, 1, 1)   # exactly 1 worker per job
gr.set_cardinality(Job, 1, 1)      # exactly 1 job per worker
gr.add_statistic(skill_allignment) # adding the skill alignment statistic
gr.set_objective_function("minimize_sum_of_single_statistic")

print("Answer: ", solve(gr, groups)) # Optimizing from a list of Groups
\end{lstlisting}


\subsection{Stable Marriage}
This example in Script \ref{script:stable_marriage} demonstrates stable pairing, where two groups (e.g., males and females) are matched based on mutual preferences, further explanation can be found at Section \ref{subsec:stable_matching}. The framework allows defining preference-based rules to ensure a stable solution where no pairs would prefer a different match.
The fitness function of the following function is shown in Algorithm \ref{algo:stbl_marriage_obj_func} in Section \ref{sec:obj_func}, and its implementation is builtin, there is only need to configure the preferences and set stable matching.

\begin{lstlisting}[language=Python, caption={Creating and solving the Stable Marriage Problem using this framework.}, label={script:stable_marriage}]
from src.group import Group, GroupRule
import random
from typing import List, Type
from src.solve import solve

# Define the Classes for the Example
class Man:
    def __init__(self, name, preferences=None):
        self.name = name
        self.preferences = preferences if preferences is not None else []
    def add_preference(self, preference):
        if preference not in self.preferences:
            self.preferences.append(preference)
    def __repr__(self): return f"Man({self.name})"

class Woman:
    def __init__(self, name, preferences=None):
        self.name = name
        self.preferences = preferences if preferences is not None else []
    def add_preference(self, preference):
        if preference not in self.preferences:
            self.preferences.append(preference)
    def __repr__(self): return f"Woman({self.name})"


# Define a function to validate if the preferences are repected:
# The Stabel Matching does not allow statistics, so this is a validator function.
def stable_matching_validator(members):
    males = members.get(Man, [])
    females = members.get(Woman, [])
    male_partner = {m: f for m, f in zip(males, females)}
    female_partner = {f: m for m, f in zip(males, females)}

    for m in males:
        for w_name in preferences[m.name]:
            w = next((f for f in females if f.name == w_name), None)
            if w is None:
                continue
            if female_partner[w] != m:
                current_w = male_partner[m]
                current_m = female_partner[w]

                m_prefers_w = preferences[m.name].index(w.name) < preferences[m.name].index(current_w.name)
                w_prefers_m = preferences[w.name].index(m.name) < preferences[w.name].index(current_m.name)

                if m_prefers_w and w_prefers_m:
                    return False
    return True


# Create instances
men = {
    "John":   Man("John"),   "Paul":  Man("Paul"),  "Mike":    Man("Mike"),
    "George": Man("George"), "Ringo": Man("Ringo"), "Pete":    Man("Pete"),
    "Brian":  Man("Brian"),  "Roger": Man("Roger"), "Freddie": Man("Freddie")
}

women = {
    "Mary":  Woman("Mary"),  "Patricia": Woman("Patricia"), "Susan":   Woman("Susan"),
    "Linda": Woman("Linda"), "Karen":    Woman("Karen"),    "Jessica": Woman("Jessica"),
    "Sarah": Woman("Sarah"), "Jennifer": Woman("Jennifer"), "Nancy":   Woman("Nancy")
}

# Define preferences
men_preferences = {
    "John":    [women["Mary"],     women["Linda"],    women["Susan"]   ],
    "Paul":    [women["Linda"],    women["Mary"],     women["Susan"]   ],
    "Mike":    [women["Susan"],    women["Mary"],     women["Linda"]   ],
    "George":  [women["Patricia"], women["Jennifer"], women["Jessica"] ],
    "Ringo":   [women["Jennifer"], women["Patricia"], women["Jessica"] ],
    "Pete":    [women["Jessica"],  women["Patricia"], women["Jennifer"]],
    "Brian":   [women["Sarah"],    women["Karen"],    women["Nancy"]   ],
    "Roger":   [women["Karen"],    women["Sarah"],    women["Nancy"]   ],
    "Freddie": [women["Nancy"],    women["Sarah"],    women["Karen"]   ],
}

woman_preferences = {
    "Mary":     [men["John"],    men["Paul"]   ],
    "Linda":    [men["Paul"],    men["Mike"]   ],
    "Susan":    [men["Mike"],    men["George"] ],
    "Patricia": [men["George"],  men["Ringo"]  ],
    "Jennifer": [men["Ringo"],   men["Pete"]   ],
    "Jessica":  [men["Pete"],    men["Brian"]  ],
    "Sarah":    [men["Brian"],   men["Roger"]  ],
    "Karen":    [men["Roger"],   men["Freddie"]],
    "Nancy":    [men["Freddie"], men["John"]   ],
}

for name, man in woman_preferences.items():
    women[name].add_preference(man)

for name, woman in men_preferences.items():
    men[name].add_preference(woman)

instances = list(men.values()) + list(women.values())


# Create the Group Rule
gr = GroupRule()
gr.set_cardinality(Woman, 1, 1)
gr.set_cardinality(Man, 1, 1)
gr.set_stable_match(True)                   # Enables stable matching
gr.add_validator(stable_matching_validator) # Adds the validator function

# Optimizing from a list of Instances, the groups are built automatically.
print("Answer: ", solve(gr, instances))
\end{lstlisting}

\subsection{Timetable Scheduling}
For complex scheduling scenarios, such as assigning professors, rooms, lectures, and times, we use a genetic algorithm. This example defines a fitness function to solve resource allocation while meeting scheduling constraints. 

\begin{lstlisting}[language=Python, caption={Creating and solving the Timetable Scheduling Problem using this framework.}, label={script:time_table}]
from src.group import Group, GroupRule
import itertools
from src.solve import solve


# Define the Classes for the Example
class Professor:
    def __init__(self, name): self.name = name
    def __repr__(self): return f"Professor({self.name})"

class Room:
    def __init__(self, name): self.name = name
    def __repr__(self): return f"Room({self.name})"

class Cohort:
    def __init__(self, name): self.name = name
    def __repr__(self): return f"Cohort({self.name})"

class TimeWindow:
    def __init__(self, slot): self.slot = slot
    def __repr__(self): return f"TimeWindow({self.slot})"

class Subject:
    def __init__(self, name): self.name = name
    def __repr__(self): return f"Subject({self.name})"

# Define an objective function to evaluate the groups
def objective_function(groups):
    professors_by_time = dict()
    rooms_by_time = dict()
    score = 0
    for group in groups:
        elements = group.get_members()
        time = elements[TimeWindow][0]
        professor = elements[Professor][0]
        room = elements[Room][0]

        if time not in professors_by_time:
            professors_by_time[time] = set()
        if time not in rooms_by_time:
            rooms_by_time[time] = set()

        # Check for conflicts in professors and rooms at the same time
        if professor in professors_by_time[time]:
            score += -1e6 # Conflict penalty for professor
        else:
            score += 1e6 # Reward for only one professor at a given time
        if room in rooms_by_time[time]:
            score += -1e6 # Conflict penalty for room
        else:
            score += 1e6 # Reward for only one room at a given time

        professors_by_time[time].add(professor)
        rooms_by_time[time].add(room)

    return score


# Create instances and groups for the problem
professors =   [Professor("ProfA"), Professor("ProfB"), Professor("ProfC")]
rooms =        [Room("Room1"), Room("Room2"), Room("Room3")]
cohorts =      [Cohort("Cohort1"), Cohort("Cohort2"), Cohort("Cohort3")]
time_windows = [TimeWindow("8h"), TimeWindow("10h"), TimeWindow("12h")]
subjects =     [Subject("Math"), Subject("History"), Subject("Science")]

instances =  [professors, rooms, cohorts, time_windows, subjects]
all_groups = [Group().add_member(*comb) for comb in itertools.product(*instances)]

# Define the Group Rule
gr = GroupRule()
gr.set_cardinality(Professor, 1, 1)
gr.set_cardinality(Room, 1, 1)
gr.set_cardinality(Cohort, 1, 1)
gr.set_cardinality(TimeWindow, 1, 1)
gr.set_cardinality(Subject, 1, 1)
gr.set_objective_function(objective_function)


# solve from a list of Groups
print(solve(gr, all_groups))
\end{lstlisting}

To see more information about resource allocation problems, see Section \ref{subsec:resource_allocation}.


\subsection{Fair Bipartite Matching}

An algorithm for the Fair Bipartite Matching Problem was implemented, as detailed in Chapter~\ref{chap:fair_matching}.  
This implementation allows the framework to address scenarios in which minimum quotas must be taken into account during grouping or resource allocation.

Script~\ref{script:fair_bipartite_matching} illustrates an example similar to Script~\ref{script:job_assignement}, with the main difference being the inclusion of minimum quotas.
Notably, the example from Script~\ref{script:fair_bipartite_matching} would be assigned to the \textit{HungarianAlgorithm} solver, in case the quotas were not defined.
This example corresponds to the scenario represented in Section~\ref{sec:workers_jobs_example}.

\begin{lstlisting}[language=Python, caption={Creating and solving a Fair Matching Problem using this framework.}, label={script:fair_bipartite_matching}]
from src.group import GroupRule
import random
from typing import List, Type
from src.solve import solve

class Worker:
    def __init__(self, ID, Disabilities):
        self.ID = ID
        self.skills = random.randint(1, int(1e6)) # Simulating skills
        self.Disabilities = Disabilities
    def __repr__(self):
        return f"Worker(ID={self.ID}, Disabilities='{self.Disabilities}')"
class Job:
    def __init__(self, ID):
        self.ID = ID
        self.skills = random.randint(1, int(1e6)) # Simulating skills required
    def __repr__(self):
        return f"Job({self.ID})"

# Define the Statistic Function
def skill_allignment(members: dict[Type, List]):
    workers = members.get(Worker, [None])[0]
    jobs = members.get(Job, [None])[0]
    return abs(workers.skills - jobs.skills)

# Create instances for the problem
def gen_jobs(qnt):
    # Generate a list of Job instances
    list = []
    for i in range(qnt):
        list.append(Job(ID=i))
    return list

def gen_workers(qnt):
    # Generate a list of Worker instances, with some having disabilities
    list = []
    disabilities = ["Low Visibility", "Low Mobility", "None"]
    for i in range(0, int(qnt*0.8)):
        list.append(Worker(
            ID=i,
            Disabilities=disabilities[2]
        ))
    for i in range(int(qnt*0.8), qnt):
        list.append(Worker(
            ID=i,
            Disabilities=disabilities[random.randint(0,len(disabilities)-1)]
        ))
    return list

instances = gen_jobs(3) + gen_workers(6)

# Define the problem rules
gr = GroupRule()
# Two slots for both low mobility and low visibility
gr.quotas = {"Disabilities":[["Low Visibility", 2],["Low Mobility", 2]]}
gr.set_cardinality(Job, 1, 1)           # Each group must have exactly 1 job
gr.set_cardinality(Worker, 1, 1)        # Each group must have exactly 1 worker
gr.add_statistic(skill_allignment)      # Add the skill alignment statistic
# Use the sum of the statistic as the objective function
gr.set_objective_function("minimize_sum_of_single_statistic")  

# solve from a list of instances
print("Answer: ", solve(gr, instances))
\end{lstlisting}
\chapter{Conclusion} \label{chap:conclusion}

This work presented a framework designed to address a wide range of matching and grouping problems, with a focus on flexibility, accessibility, and practical applications for industry. Through a modular design and support for various optimization techniques, including both exact and heuristic methods, the framework provides an effective tool for developers tackling complex matching problems across diverse contexts.

\section{Contributions}
    This work makes several contributions to the field of matching and grouping problems, each designed to enhance the flexibility, applicability, and efficiency of solving these problems in real-world scenarios. The contributions outlined below highlight the definition of a metaheuritics to solve grouping problem, the development of a framework that integrates various optimization techniques, a new algorithm for the minimum quota bipartite matching problem and gathering information about the current state of matching problems articles.
    \subsection{Metaheuristic}
    A contribution of this work is the mapping of grouping problems to a genetic algorithm (GA) approach. By developing a structured method for representing grouping problems within the genetic algorithm framework, this work provides a foundation for solving complex, NP-hard problems that may not be feasible with exact algorithms. This contribution enables approximate solutions through GA for various types of grouping problems, expanding the framework's capacity to handle cases where computational efficiency and flexibility are prioritized over exact solutions.


    \subsection{Framework}
        The proposed framework represents a flexible and extensible tool designed for a variety of matching problems. By allowing users to define problem-specific constraints, costs, and preferences, the framework accommodates both standard and customized matching configurations. This adaptability enables its application across multiple domains, from resource allocation to scheduling, and makes it suitable for developers with varying levels of expertise in optimization.
    
        \subsubsection{Limitations}
            Despite its versatility, the framework has some limitations. One significant limitation is the requirement for users to implement their own fitness function to define problem-specific optimization criteria. This requires familiarity with both the framework and the underlying optimization principles, which may be a barrier for some users.
            
            Additionally, the framework currently provides only three solvers. Although these cover a range of matching problems, users must add new solvers if they require support for problems with unique "statistics" or specific geometries not handled by the existing options. Expanding the solver library could enhance the framework's applicability to a broader array of problems.

    \subsection{Fair Matching Algorithm}
        The present work introduced an approach to the bipartite matching problem with the inclusion of minimum quotas, aiming to promote fairness in resource allocation. The proposed mapping for the Minimum Cost Maximum Flow (MCMF) problem, adapted to consider quotas, proved valid in the experiments conducted.

        In this work, we focus on the fair allocation of resources in scenarios where supply ($U$) exceeds demand ($V$).
        Without loss of generality, swapping the sets $U$ and $V$ allows deploying our strategy in cases where demand exceeds supply.
        
        Nevertheless, it is important to highlight that there are several areas to be explored in future work. Verifying the effectiveness and significance of the quotas in different contexts is an open field, requiring more in-depth analyses and broader experiments. Additionally, the implementation must be subjected to rigorous testing to ensure the correctness of the code, as well as performance optimization.
        
        Other research directions include investigating alternative approaches for the definition and application of quotas, as well as exploring machine learning techniques to optimize quota selection in different scenarios. Refining the proposed model and analyzing its applicability across various domains also represent valuable opportunities for future research.
        
        In summary, this work is an initial step toward understanding and applying quotas in the context of bipartite matching, but there is a vast area to be explored in pursuit of a more comprehensive understanding and improvement of the proposed approach.
        
        \subsubsection{Advantages}
        
        One of the fundamental advantages of this approach is the ability to employ any MCMF algorithm, even those that are not exact. This flexibility is particularly beneficial, as the different existing implementation methods place varying importance on the trade-off between execution time and solution accuracy.
        
        Furthermore, there is the opportunity to use distributed techniques to solve the MCMF problem, broadening the available options for addressing it.
        Depending on the chosen algorithm, it is not necessary to have all relationships between the elements of each set precalculated, which can reduce memory usage and processing time.
        
        In summary, the approach offers great flexibility, allowing it to be computed by various algorithms, thus providing a wide range of options for solving the problem.
        
        \subsubsection{Future Work}
        
        In future work, we will explore the concept of fairness in relation to more than two sets. 
        For example, in the timetable problem, one may be required to match rooms, modules, lectures, and students' timetables subjected to constraints based on fair allocation principles.
        In another future work, we will study the application of fairness in bipartite graphs based on maximum quota principles.


\subsection{Review of Recent Developments in Matching Problems and Algorithms}

    This work also conducted a preliminary review of recent advancements in matching problems, focusing on contributions made since 2021. Through a comprehensive search of relevant studies, we identified developments in matching algorithms and problem types. After filtering the studies by language and relevance, we examined the most prominent topics that emerged from the remaining works. This review highlights the continued evolution of the field and contributes to the understanding of cutting-edge solutions and their potential applications in industry.


\section{Future Work}
    Future work could focus on several areas to enhance the framework's capabilities and usability. 
    One priority is to expand the library of built-in solvers to cover a wider variety of matching problems, reducing the need for users to implement custom solvers for specific configurations. 
    Another area for improvement is the automation of fitness functions or the development of built-in options that cover common optimization criteria, making the framework more accessible to non-expert users. 
    Furthermore, optimizing the metaheuristic solver to better handle large datasets and incorporating support for multi-criteria optimization in fair matching could make the framework even more robust and versatile. 
    Additionally, the Set Cover Framework (SCF) could be extended to address clustering problems by mapping them to grouping problems, enabling the framework to handle scenarios where elements need to be grouped based on similarity or other clustering criteria.


% -----------------------------------------------------------------
% ELEMENTOS PÓS-TEXTUAIS
% -----------------------------------------------------------------
\postextual

% Você pode comentar os elementos que não deseja em seu trabalho;

% Referências bibliográficas

\bibliography{abntex2-ref_UDESC_2020}	% Elemento Obrigatório

% ----------------------------------------------------------
% Glossário
% ----------------------------------------------------------

%Consulte o manual da classe abntex2 para orientações sobre o glossário.

%\glossary




% ----------------------------------------------------------
% Glossário (Formatado Manualmente)
% ----------------------------------------------------------

% \chapter*{GLOSSÁRIO}
% \addcontentsline{toc}{chapter}{GLOSSÁRIO}

% { \setlength{\parindent}{0pt} % ambiente sem indentação


% } % fim ambiente sem indentação


				% Elemento Opcional

% ----------------------------------------------------------
% Apêndices
% ----------------------------------------------------------

% ---
% Inicia os apêndices
% ---
% \begin{apendicesenv}

% % Imprime uma página indicando o início dos apêndices
% %\partapendices

% % ----------------------------------------------------------
% \chapter{TÍTULO}
% % ----------------------------------------------------------




% \end{apendicesenv}
% % ---				% Elemento Opcional

% ----------------------------------------------------------
% Anexos
% ----------------------------------------------------------
%
% ---
% Inicia os anexos
% ---
% \begin{anexosenv}

% % Imprime uma página indicando o início dos anexos
% %\partanexos

% % ---
% \chapter{TÍTULO}
% % ---



% \end{anexosenv}
				% Elemento Opcional

%%---------------------------------------------------------------------
%% INDICE REMISSIVO
%%---------------------------------------------------------------------

%\phantompart
%\printindex

%---------------------------------------------------------------------

%%---------------------------------------------------------------------
%% INDICE REMISSIVO (Formatado Manualmente)
%%---------------------------------------------------------------------

% \chapter*{ÍNDICE}
% \addcontentsline{toc}{chapter}{ÍNDICE}

% { \setlength{\parindent}{0pt}  % ambiente sem indentação
	
% Andesito, 22, 50, 73

% Argila, 52, 75, 121

% Basalto, 25, 230, 235

	
	
	
	
% } % fim ambiente sem indentação


		% Elemento Opcional

% ----------------------------------------------------------
% Apêndices
% ----------------------------------------------------------

% ---
% Inicia os apêndices
% ---
% \begin{apendicesenv}

% % Imprime uma página indicando o início dos apêndices
% %\partapendices

% % ----------------------------------------------------------
% \chapter{TÍTULO}
% % ----------------------------------------------------------




% \end{apendicesenv}
% % ---				% Elemento Opcional

\end{document}

% -----------------------------------------------------------------
% Fim do Documento
% -----------------------------------------------------------------	